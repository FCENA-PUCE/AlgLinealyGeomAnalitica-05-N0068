\documentclass[a4,11pt]{aleph-notas}

% -- Paquetes adicionales
\usepackage{enumitem}
\usepackage{aleph-comandos}
\hypersetup{urlcolor=blue}

% -- Datos  
\institucion{Facultad de Ciencias Exactas, Naturales y Ambientale}
\carrera{Catálogo STEM}
\asignatura{Álgebra Lineal y Geometría Analítica}
\tema{Clase invertida no. 3: Valores propios}
\autor[A. Merino]{Andrés Merino}
\fecha{Periodo 2025-1}

\logouno[0.14\textwidth]{Logos/logoPUCE_04_ac}
\definecolor{colortext}{HTML}{0030A1}
\definecolor{colordef}{HTML}{0030A1}
\fuente{montserrat}

% -- Comandos adicionales
\begin{document}

\encabezado

\vspace*{-10mm}
\section*{Introducción}

\begin{itemize}
    \item \textbf{Tema:} Valores propios
    \item \textbf{Resultado de Aprendizaje:} Calcula valores propios de matrices.
\end{itemize}

\section*{Lección en casa}

\subsection*{Actividades}

\begin{enumerate}[leftmargin=*]
    \item Interactuar con ChatGPT mediante los siguientes \textit{prompts}, leyendo detenidamente el \textit{prompt} y su respuesta:
    \begin{enumerate}[label=\textit{Prompt \arabic*.},leftmargin=2.1cm]
        \item Vas a ser mi profesor de la asignatura de Álgebra Lineal, te iré dando indicaciones y me irás explicando de manera formal lo que te pida. Vas a tener mucho cuidado al escribir la parte matemática para que se visualice bien. Sé divertido. ¿Entendido?
        \item ¿Cómo se calcula un valor propio de una matriz? No me des un ejemplo numérico aún.
        \item Dame un ejemplo del cálculo de valores propios con una matriz de 2 por 2.
    \end{enumerate}
    \item Visualiza el siguiente video: \href{https://youtu.be/HET8XcIX-n4?si=t4lUbTmWaPOTbtAM}{Obteniendo los valores propios de una matriz de 2$\times$2}.
    \item Continúa la interacción con ChatGPT mediante los siguientes \textit{prompts}, leyendo detenidamente el \textit{prompt} y su respuesta:
    \begin{enumerate}[label=\textit{Prompt \arabic*.},leftmargin=2.1cm,start=4]
        \item Dame un ejemplo del cálculo de valores propios con una matriz de 3 por 3, que el ejemplo sea en una matriz triangular. Realízalo paso a paso con el cálculo de determinante.
        \item Plantéame un ejercicio de cálculo de valores propios en matrices de 2 por 2.
    \end{enumerate}
    \item Continúa la interacción con ChatGPT con las preguntas sobre este tema.
    \item Realiza el cuestionario del aula virtual.
\end{enumerate}

\end{document}