\documentclass[a4,11pt]{aleph-notas}

% -- Paquetes adicionales
\usepackage{enumitem}
\usepackage{aleph-comandos}
\usepackage{systeme}

% -- Datos  
\institucion{Facultad de Ciencias Exactas, Naturales y Ambientale}
\carrera{Catálogo STEM}
\asignatura{Álgebra Lineal y Geometría Analítica}
\tema{Ejercicios resueltos}
\autor{Andrés Merino}
\fecha{Semestre 2025-1}

\logouno[0.14\textwidth]{Logos/logoPUCE_04_ac}
\definecolor{colortext}{HTML}{0030A1}
\definecolor{colordef}{HTML}{0030A1}
\fuente{montserrat}

% -- Comandos adicionales
\setlist[enumerate]{label=\roman*.}

\begin{document}

\encabezado

\section{Cálculo del rango de una matriz}


%%%%%%%%%%%%%%%%%%%%%%%%%%%%%%%%%%%%%%%%
\begin{ejer}
    Dadas las siguientes matrices:
    \[
        A=
        \begin{pmatrix}
        3 & 1 & 3\\
        1 & 2 & 3\\
        -1 & -3 & 4\\
        3 & 9 & -12
        \end{pmatrix}
        , \hspace{0.1cm}
       B=
        \begin{pmatrix}
        1 & -1 & 1 & 4\\
        3 & 2 & 1 & 2\\
        4 & 2 & 2 & 8
        \end{pmatrix}
        \texty
        C=
        \begin{pmatrix}
        -2 & 3 & 1 &0\\
        1 & 3 & -1 & -2
        \end{pmatrix},
    \]
    hallar su rango.
\end{ejer}

\begin{proof}[Solución]\hspace{0pt}
    \begin{enumerate}
    \item 
        Vamos a hallar el rango de $A$, para ello reduciremos la matriz $A$ por filas.
        \begin{align*}
            \begin{pmatrix}
            3 & 1 & 3\\
            1 & 2 & 3\\
            -1 & -3 & 4\\
            3 & 9 & -12
            \end{pmatrix} & \sim 
            \begin{pmatrix}
            1 & 2 & 3\\
            3 & 1 & 3\\
            -1 & -3 & 4\\
            3 & 9 & -12
            \end{pmatrix} &&
            F_1 \leftrightarrow F_2\\
            & \sim 
            \begin{pmatrix}
            1 & 2 & 3\\
            0 & -5 & -6\\
            -1 & -3 & 4\\
            3 & 9 & -12
            \end{pmatrix} &&
            -3F_1 + F_2 \to F_2\\ 
            & \sim 
            \begin{pmatrix}
            1 & 2 & 3\\
            0 & -5 & -6\\
            0 & -1 & 7\\
            3 & 9 & -12
            \end{pmatrix} &&
            F_1 + F_3 \to F_3\\  
            & \sim 
            \begin{pmatrix}
            1 & 2 & 3\\
            0 & -5 & -6\\
            0 & -1 & 7\\
            0 & 3 & -21
            \end{pmatrix} &&
            -3F_1 + F_4 \to F_4\\
            & \sim 
            \begin{pmatrix}
            1 & 2 & 3\\
            0 & 1 & -7\\
            0 & -5 & -6\\
            0 & 3 & -21
            \end{pmatrix} &&
            -F_3 \leftrightarrow F_2\\
            & \sim 
            \begin{pmatrix}
            1 & 2 & 3\\
            0 & 1 & -7\\
            0 & 0 & -41\\
            0 & 3 & -21
            \end{pmatrix} &&
            5F_2 + F_3 \to F_3 \\     
            & \sim 
            \begin{pmatrix}
            1 & 2 & 3\\
            0 & 1 & -7\\
            0 & 0 & -41\\
            0 & 0 & 0
            \end{pmatrix} &&
            -3F_2 + F_4 \to F_4 \\  
            & \sim 
            \begin{pmatrix}
            1 & 2 & 3\\
            0 & 1 & -7\\
            0 & 0 & 1\\
            0 & 0 & 0
            \end{pmatrix} &&
            -\frac{1}{41}F_3 \to F_3\\
            & \sim
             \begin{pmatrix}
            1 & 2 & 3\\
            0 & 1 & 0\\
            0 & 0 & 1\\
            0 & 0 & 0
            \end{pmatrix} &&
            7F_3+F_2 \to F_2\\
            & \sim 
            \begin{pmatrix}
            1 & 2 & 0\\
            0 & 1 & 0\\
            0 & 0 & 1\\
            0 & 0 & 0
            \end{pmatrix} &&
            -3F_3+F_1 \to F_1\\
            & \sim \begin{pmatrix}
            1 & 0 & 0\\
            0 & 1 & 0\\
            0 & 0 & 1\\
            0 & 0 & 0
            \end{pmatrix} &&
            -2F_2+F_1 \to F_1\\                
        \end{align*}
        Así la matriz escalonada reducida por filas, equivalente a $A$ es 
        \[
            \begin{pmatrix}
            1 & 0 & 0\\
            0 & 1 & 0\\
            0 & 0 & 1\\
            0 & 0 & 0
            \end{pmatrix}           
        \]
        Por lo tanto $\rang(A)=3$.
    \item 
        Para encontrar el rango de $B$ hallaremos la matriz escalonada reducida por filas equivalente a $B$.
        \begin{align*}
            \begin{pmatrix}
            1 & -1 & 1 & 4\\
            3 & 2 & 1 & 2\\
            4 & 2 & 2 & 8
            \end{pmatrix}
            & \sim 
            \begin{pmatrix}
            1 & -1 & 1 & 4\\
            0 & -5 & 2 & 10\\
            4 & 2 & 2 & 8
            \end{pmatrix} &&
            3F_1 - F_2 \to F_2\\
            & \sim 
            \begin{pmatrix}
            1 & -1 & 1 & 4\\
            0 & -5 & 2 & 10\\
            0 & -6 & 2 & 8
            \end{pmatrix} &&
            4F_1 - F_3 \to F_3\\
            & \sim 
            \begin{pmatrix}
            1 & -1 & 1 & 4\\
            0 & 1 & -\frac{2}{5} & -2\\
            0 & -6 & 2 & 8
            \end{pmatrix} &&
            \frac{1}{5}F_2 \to F_2\\    
            & \sim 
            \begin{pmatrix}
            1 & -1 & 1 & 4\\
            0 & 1 & -\frac{2}{5} & -2\\
            0 & 0 & -\frac{2}{5} & -4
            \end{pmatrix} &&
            6F_2 + F_3 \to F_3\\       
            & \sim 
            \begin{pmatrix}
            1 & -1 & 1 & 4\\
            0 & 1 & -\frac{2}{5} & -2\\
            0 & 0 & 1 & 10
            \end{pmatrix} &&
            -\frac{5}{2}F_3 \to F_3\\    
            & \sim 
            \begin{pmatrix}
            1 & -1 & 1 & 4\\
            0 & 1 & 0 & 2\\
            0 & 0 & 1 & 10
            \end{pmatrix} &&
            \frac{2}{5}F_3  + F_2 \to F_2\\     
            & \sim 
            \begin{pmatrix}
            1 & -1 & 0 & -6\\
            0 & 1 & 0 & 2\\
            0 & 0 & 1 & 10
            \end{pmatrix} &&
            F_1  - F_3 \to F_1\\
            & \sim 
            \begin{pmatrix}
            1 & 0 & 0 & -4\\
            0 & 1 & 0 & 2\\
            0 & 0 & 1 & 10
            \end{pmatrix} &&
            F_1  + F_2 \to F_1\\
        \end{align*}
    Así la matriz escalonada reducida por filas, equivalente a $B$ es
        \[
            \begin{pmatrix}
            1 & 0 & 0 & -4\\
            0 & 1 & 0 & 2\\
            0 & 0 & 1 & 10
            \end{pmatrix} .   
        \]
    Por lo tanto $\rang(B)=3$.
    \item Para encontrar el rango de $C$ hallaremos la matriz escalonada reducida por filas equivalente a $C$.
        \begin{align*}
            \begin{pmatrix}
            -2 & 3 & 1 &0\\
            1 & 3 & -1 & -2
            \end{pmatrix} 
            & \sim
            \begin{pmatrix}
            1 & 3 & -1 & -2\\
            -2 & 3 & 1 &0
            \end{pmatrix} &&
            F_1 \leftrightarrow F_2\\
            & \sim
            \begin{pmatrix}
            1 & 3 & -1 & -2\\
            0 & 9 & -1 & -4
            \end{pmatrix} &&
            2F_1 + F_2 \to F_2\\     
            & \sim
            \begin{pmatrix}
            1 & 3 & -1 & -2\\
            0 & 1 & -\frac{1}{9} & -\frac{4}{9}
            \end{pmatrix} &&
            \frac{1}{9}F_2 \to F_2\\ 
            & \sim
            \begin{pmatrix}
            1 & 0 & -\frac{2}{3} & -\frac{2}{3}\\
            0 & 1 & -\frac{1}{9} & -\frac{4}{9}
            \end{pmatrix} &&
            F_1 - 3F_2 \to F_1\\   
         \end{align*}
    Así la matriz escalonada reducida por filas equivalente a $C$ es
        \[
            \begin{pmatrix}
            1 & 0 & -\frac{2}{3} & -\frac{2}{3}\\
            0 & 1 & -\frac{1}{9} & -\frac{4}{9}
            \end{pmatrix} .           
        \]
    Por lo tanto $\rang(C)=2$.
    \end{enumerate}
\end{proof}

\section{Soluciones de sistemas de ecuaciones lineales}


%%%%%%%%%%%%%%%%%%%%%%%%%%%%%%%%%%%%%%%%
\begin{ejer}
    Dado el sistema lineal de ecuaciones 
    \[
    \systeme{x+5y-4z=0, x-2y+z=0, 3x+y-2z=0}
    \]
    utilizar la eliminación de Gauss-Jordan para determinar el conjunto de soluciones del sistema.
\end{ejer}

\begin{proof}[Solución]\hspace{0pt}
    La forma matricial del sistema es:
    \[
    \begin{pmatrix}
    1&5&-4\\1&-2&1\\3&1&-2
    \end{pmatrix}
    \begin{pmatrix}
    x \\ y\\ z
    \end{pmatrix}
    =
    \begin{pmatrix}
    0 \\ 0\\ 0
    \end{pmatrix}.
    \]
    La matriz ampliada correspondiente es:
    \[
    \begin{pmatrix}
    1&5&-4&|&0\\
    1&-2&1&|&0\\
    3&1&-2&|&0
    \end{pmatrix}.
    \]
    Ahora, podemos aplicar la aliminación de Gauss-Jordan a la matriz ampliada:
    \begin{align*}
      \begin{pmatrix}
        1&5&-4&|&0\\
        1&-2&1&|&0\\
        3&1&-2&|&0
      \end{pmatrix}
      & \sim 
      \begin{pmatrix}
        1&5&-4&|&0\\
        0&-7&5&|&0\\
        0&-14&10&|&0
      \end{pmatrix}
      && -1F_1+F_2\to F_2\\
      &&&-3F_1+F_3\to F_3\\
      & \sim 
      \begin{pmatrix}
        1&5&-4&|&0\\
        0&-7&5&|&0\\
        0&0&0&|&0
      \end{pmatrix}
      && -2F_2+F_3\to F_3\\
      & \sim 
      \begin{pmatrix}
        1&5&-4&|&0\\
        0&1&-\frac{5}{7}&|&0\\
        0&0&0&|&0
      \end{pmatrix}
      && -\frac{1}{7}F_2\to F_2\\
      & \sim 
      \begin{pmatrix}
        1&0&-\frac{3}{7}&|&0\\
        0&1&-\frac{5}{7}&|&0\\
        0&0&0&|&0
      \end{pmatrix}
      && -5F_2+F_1\to F_1\\
      \end{align*}
      Como la matriz equivalente a la matriz de los coeficientes ya es escalonada reducida por filas, comparamos los rangos $\rang (A) = \rang (A|b) = 2$ y es menor que el número de incógnitas $(3)$ entonces, el sistema tiene infinitas soluciones. Como los sistemas
      \[
      \systeme{x+5y-4z=0, x-2y+z=0, 3x+y-2z=0}
      \]
      y
      \[
      \systeme{x-\frac{3}{7}z=0, y-\frac{5}{7}z=0}
      \]
      son equivalentes, por ende, tienen las mismas soluciones
      \[
      \systeme{x=\frac{3}{7}t, y=\frac{5}{7}t, z=t}
      \]
      Es decir, el conjunto de soluciones del sistema es
      \[
      \left\{ \left( \frac{3}{7}t,\frac{5}{7}t\right),t:t \in \R \right\}.\qedhere
      \]
\end{proof}


%%%%%%%%%%%%%%%%%%%%%%%%%%%%%%%%%%%%%%%%
\begin{ejer}
    Dado el sistema de ecuaciones lineales:
    \[
        \systeme{x+y-3z=-1, 2x+y-2z=1, x+y+z=3, x+2y-3z=1}
    \]
    utilizar la eliminación de Gauss-Jordan para determinar si el sistema es consistente.
\end{ejer}

\begin{proof}[Solución]\hspace{0pt}
    Tenemos las matrices
    \[
        A = \begin{pmatrix}
            1&1&-3\\2&1&-2\\1&1&1\\1&2&-3
        \end{pmatrix}
        \texty
        b = \begin{pmatrix}
             -1 \\ 1\\ 3 \\ 1
        \end{pmatrix}.
    \]
    Así, el sistema en forma matricial es
    \[
        \begin{pmatrix}
            1&1&-3\\2&1&-2\\1&1&1\\1&2&-3
        \end{pmatrix}
        \begin{pmatrix}
            x \\ y\\ z
        \end{pmatrix}
        =
        \begin{pmatrix}
            -1 \\ 1\\ 3 \\ 1
        \end{pmatrix}.
    \]
    De donde, la matriz ampliada es
    \[
        \begin{pmatrix}
            1&1&-3&|&-1\\
            2&1&-2&|&1\\
            1&1&1&|&3\\
            1&2&-3&|&1
         \end{pmatrix}.
    \]
    Realizamos la eliminación de Gauss-Jordan sobre la matriz ampliada, así:
    \begin{align*}
        \begin{pmatrix}
            1&1&-3&|&-1\\
            2&1&-2&|&1\\
            1&1&1&|&3\\
            1&2&-3&|&1
        \end{pmatrix}
        & \sim 
        \begin{pmatrix}
            1&1&-3&|&-1\\
            0&-1&4&|&3\\
            0&0&4&|&4\\
            0&1&0&|&2
        \end{pmatrix}
        && -2F_1+F_2\to F_2\\
        &&&-1F_1+F_3\to F_3\\
        &&&-1F_1+F_4\to F_4\\
        & \sim 
        \begin{pmatrix}
            1&1&-3&|&-1\\
            0&0&4&|&5\\
            0&0&4&|&4\\
            0&1&0&|&2
        \end{pmatrix}
        && F_4+F_2\to F_2\\
        & \sim 
        \begin{pmatrix}
            1&1&-3&|&-1\\
            0&1&0&|&2\\
            0&0&4&|&4\\
            0&0&4&|&5
        \end{pmatrix}
        && F_2 \leftrightarrow F_4\\
        & \sim 
        \begin{pmatrix}
            1&1&-3&|&-1\\
            0&1&0&|&2\\
            0&0&4&|&4\\
            0&0&0&|&1
        \end{pmatrix}
        && -1F_3+F_4\to F_4\\
        & \sim 
        \begin{pmatrix}
            1&1&-3&|&-1\\
            0&1&0&|&2\\
            0&0&1&|&1\\
            0&0&0&|&1
        \end{pmatrix}
        && \frac{1}{4}F_4\to F_4\\
         & \sim 
        \begin{pmatrix}
            1&0&-3&|&-3\\
            0&1&0&|&2\\
            0&0&1&|&1\\
            0&0&0&|&1
        \end{pmatrix}
        && -1F_2+F_1\to F_1\\
        & \sim 
        \begin{pmatrix}
            1&0&0&|&0\\
            0&1&0&|&2\\
            0&0&1&|&1\\
            0&0&0&|&1
        \end{pmatrix}
        && 3F_3+F_1\to F_1\\
    \end{align*}
    Con esto, tenemos que $\rang(A)=3 < \rang(A|b) = 4$, entonces el sistema es inconsistente; es decir, no tiene solución.
\end{proof}


%%%%%%%%%%%%%%%%%%%%%%%%%%%%%%%%%%%%%%%%
\begin{ejer}
    Sea $\alpha \in \R$; y considere el sistema lineal
    \[
        \systeme{x+y+\alpha z=2, 3x+4y+2z=\alpha, 2x+3y-z=1}
    \]
    \begin{enumerate}
        \item Utilizando la eliminación de Gauss-Jordan, determine las condiciones sobre $\alpha$ tales que el sistema tenga solución.
        \item Para las condiciones sobre $\alpha$ en que el sistema tiene solución, escriba el conjunto de soluciones del sistema.
    \end{enumerate}
\end{ejer}

\begin{proof}[Solución]\hspace{0pt}
    \begin{enumerate}
    \item 
        Tenemos las matrices
        \[
            A = \begin{pmatrix}
            1&1&\alpha\\3&4&2\\2&3&-1
            \end{pmatrix}
            \texty
            b = \begin{pmatrix}
            2 \\ \alpha\\ 1
            \end{pmatrix}.
        \]

    Así, el sistema en forma matricial es
    \[
        \begin{pmatrix}
        1&1&\alpha\\3&4&2\\2&3&-1
        \end{pmatrix}
        \begin{pmatrix}
        x \\ y\\ z
        \end{pmatrix}
        =
        \begin{pmatrix}
        2 \\ \alpha\\ 1
        \end{pmatrix}.
    \]
    De donde, la matriz ampliada $(A|b)$ es
    \[
        \begin{pmatrix}
        1&1&\alpha&|&2\\
        3&4&2&|&\alpha\\
        2&3&-1&|&1
        \end{pmatrix}.
    \]
    Realizamos la eliminación de Gauss-Jordan sobre la matriz ampliada, así:
        \begin{align*}
        \begin{pmatrix}
        1&1&\alpha&|&2\\
        3&4&2&|&\alpha\\
        2&3&-1&|&1
        \end{pmatrix}
        & \sim 
        \begin{pmatrix}
        1&1&\alpha&|&2\\
        0&1&2-3\alpha&|&\alpha - 6\\
        0&1&-1-2 \alpha&|&-3
        \end{pmatrix}
        && -3F_1+F_2\to F_2\\
        &&&-2F_1+F_3\to F_3\\
        & \sim 
        \begin{pmatrix}
        1&1&\alpha&|&2\\
        0&1&2-3\alpha&|&\alpha - 6\\
        0&0&\alpha - 3&|& 3 - \alpha
        \end{pmatrix}
        && -1F_2+F_3\to F_3\\
        \end{align*}
    Si $\alpha \neq 3$, entonces 
    \begin{align*}
        & \sim 
        \begin{pmatrix}
        1&1&\alpha&|&2\\
        0&1&2-3\alpha&|&\alpha - 6\\
        0&0&1&|& \frac{3 - \alpha}{\alpha - 3}
        \end{pmatrix}
        && \frac{1}{\alpha - 3}F_3\to F_3\\
        & \sim 
        \begin{pmatrix}
        1&1&\alpha&|&2\\
        0&1&0&|&-2\alpha - 4\\
        0&0&1&|& -1
        \end{pmatrix}
        && -(2-3\alpha)F_3+F_2\to F_2\\
        & \sim 
        \begin{pmatrix}
        1&1&0&|&\alpha+2\\
        0&1&0&|&-2\alpha - 4\\
        0&0&1&|& -1
        \end{pmatrix}
        && -\alpha F_3+F_1\to F_1\\
        & \sim 
        \begin{pmatrix}
        1&0&0&|& 3 \alpha+6\\
        0&1&0&|&-2\alpha - 4\\
        0&0&1&|& -1
        \end{pmatrix}
        && -1F_2+F_1\to F_1\\
        \end{align*}
    De donde, tenemos que el $\rang(A)=3 = \rang(A|b)$. Entonces el sistema tiene una solución única si $\alpha \neq 3$.
    Ahora, analizamos el sistema lineal cuando $\alpha = 3$;es decir, regresamos a la matriz ampliada
    \[
       \begin{pmatrix}
        1&1&\alpha&|&2\\
        0&1&2-3\alpha&|&\alpha - 6\\
        0&0&\alpha - 3&|& 3 - \alpha
        \end{pmatrix}.
    \]
    Reemplazamos $\alpha = 3$, realizamos la eliminación de Gauss-Jordan y obtenemos
    \[
        \begin{pmatrix}
        1&0&10&|&5\\
        0&1&-7&|&-3\\
        0&0&0&|&0
        \end{pmatrix}.
    \]
    Si comparamos los rangos, tenemos que, $\rang(A)=2= \rang(A|b)$, pero es menor que el número de incógnitas $(3)$, entonces el sistema tiene infinitas soluciones.  \item Si $\alpha \neq 3 $, la matriz ampliada asociada al sistema es
    \[
        \begin{pmatrix}
        1&0&0&|& 3 \alpha+6\\
        0&1&0&|&-2\alpha - 4\\
        0&0&1&|& -1
        \end{pmatrix}               .
    \]
    entonces el conjunto de soluciones del sistema es 
    \[
        \{(3 \alpha+6,-2\alpha - 4,-1):\alpha \neq 3\}.\qedhere
    \]
    Si $\alpha = 3 $, la matriz ampliada asociada al sistema es
    \[
        \begin{pmatrix}
        1&0&10&|&5\\
        0&1&-7&|&-3\\
        0&0&0&|&0
        \end{pmatrix}.
    \]
    Así, tenemos que
    \[
    \systeme{x = 5 - 10r, y= -3 +7r, z= r}
    \]
    es una solución del sistema lineal para todo $r \in \R$. Entonces el conjunto de soluciones del sistema, cuando $\alpha = 3$, es
    \[
        \{(5 - 10r,-3 +7r,r):r \in \R\}.\qedhere
    \]
    
 \end{enumerate}
\end{proof}


%%%%%%%%%%%%%%%%%%%%%%%%%%%%%%%%%%%%%%%%
\begin{ejer}
    Sea $p \in \R$ y $q \in \R$; y considere el sistema lineal
    \[
    \systeme{x+z=q, y+2w=0, x+2z+3w=0, 2y+3z+pw=3}
    \]
    Determine las condiciones sobre $p$ y $q$ para que el sistema tenga una única solución.
\end{ejer}

\begin{proof}[Solución]\hspace{0pt}
    La representación matricial del sistema es
    \[
    \begin{pmatrix}
                1&0&1&0\\0&1&0&2\\1&0&2&3\\0&2&3&p
                \end{pmatrix}
                \begin{pmatrix}
                x \\ y\\ z \\ w
                \end{pmatrix}
                =
                \begin{pmatrix}
                q \\ 0\\ 0\\ 3
                \end{pmatrix}.
    \]
    de donde, la matriz ampliada asociada al sistema es
    \[
                \begin{pmatrix}
                1&0&1&0&|&q\\
                0&1&0&2&|& 0\\
                1&0&2&3&|& 0\\
                0&2&3&p&|& 3
                \end{pmatrix} 
    \]
    Luego de aplicar la eliminación de Gauss-Jordan a la matriz ampliada, tenemos 
    \[
                \begin{pmatrix}
                1&0&0&0&|&\frac{2pq-17q+9}{p-13}\\
                0&1&0&0&|& \frac{-6q-6}{p-13}\\
                0&0&1&0&|& \frac{-pq+4q-9}{p-13}\\
                0&0&0&1&|& \frac{3q+3}{p-13}
                \end{pmatrix} 
    \]
    De donde, tenemos que, si $p \neq 13$ y para todo $q \in \R$, $\rang(A)=4 = \rang(A|b)$; y ademas es igual al número de incógnitas $(4)$. Entonces, el sistema es consistente y tiene una única solución si $p \neq 13$ y para todo $q \in \R$. Es fácil verificar que cuando $p=13$ el sistema o bien no tiene solución o tiene infinitas soluciones, por lo que este caso queda excluido de este análisis.
\end{proof}


\end{document}