\documentclass[a4,11pt]{aleph-notas}

% -- Paquetes adicionales 
\usepackage{enumitem}
\usepackage{aleph-comandos}
\usepackage{systeme}

% -- Datos  
\institucion{Facultad de Ciencias Exactas, Naturales y Ambientale}
\carrera{Catálogo STEM}
\asignatura{Álgebra Lineal y Geometría Analítica}
\tema{Ejercicios resueltos 3: Determinantes y Matriz inversa}
\autor{Andrés Merino}
\fecha{Semestre 2025-1}

\logouno[0.14\textwidth]{Logos/logoPUCE_04_ac}
\definecolor{colortext}{HTML}{0030A1}
\definecolor{colordef}{HTML}{0030A1}
\fuente{montserrat}

% -- Comandos adicionales
\setlist[enumerate]{label=\roman*.}

\begin{document}

\encabezado

\section{Determinantes}

%%%%%%%%%%%%%%%%%%%%%%%%%%%%%%%%%%%%%%%%
\begin{ejer}
    Sea la matriz:
    \[
        A= \begin{pmatrix}
            \alpha & -1 & 0 \\
            -2 & \alpha & -2 \\
            0 & -1 & \alpha
        \end{pmatrix},
    \]
    donde $\alpha \in \mathbb{R}$.
    \begin{enumerate}
    \item 
        ¿Para qu\'{e} valores de $\alpha$ la matriz $A$ es invertible? 
    \item 
        Calcule la inversa de $A$ cuando sea posible.
    \end{enumerate}
\end{ejer}

\begin{proof}[Solución]\hspace{0pt}
    \begin{enumerate}
    \item 
    Para saber si A es invertible, se debe calcular el determinante: 
    \[
    det(A)=\begin{vmatrix}
                \alpha & -1 & 0 \\
                -2 & \alpha & -2 \\
                0 & -1 & \alpha
                \end{vmatrix}=\alpha \begin{vmatrix}
                \alpha & -2 \\
                -1 & \alpha
                \end{vmatrix}-\begin{vmatrix}
                -2 & -2 \\
                0 & \alpha
                \end{vmatrix}=\alpha^3-4\alpha.\\
    \]
    Por lo tanto, $A$ es invertible si y solo si $\alpha \neq-2$, $\alpha\neq0$ y $\alpha\neq2$.
    
    \item Supongamos que $\alpha \in \mathbb{R}-\{-2, 0, 2\}$. Entonces, $A^{-1}$ existe y tenemos:
    \[
    A^{-1}=\frac{1}{det(A)}Adj(A)=\frac{1}{\alpha^3-4\alpha}   \begin{pmatrix}
                \alpha^2-2 & \alpha & 2 \\
                2\alpha & \alpha^2 & 2\alpha \\
                2 & \alpha & \alpha^2-2
            \end{pmatrix}.
    \]
    \end{enumerate}
\end{proof}

\section{Inversa de una matriz}

%%%%%%%%%%%%%%%%%%%%%%%%%%%%%%%%%%%%%%%%
\begin{ejer}
    En cada caso, suponga que la matriz $A$ es invertible, utilizar operaciones por filas para determinar su matriz inversa.
    \begin{enumerate}
        \item $A = \begin{pmatrix}
                2 & -1\\ 3 & 0
            \end{pmatrix}$
        \item $A = \begin{pmatrix}
                1 & 0 & 3\\ 0 & -2 &  1 \\ 2 & 2 & 4
            \end{pmatrix}$
    \end{enumerate}
\end{ejer}

\begin{proof}[Solución]\hspace{0pt}
    % \comentario{Utilizar la propiedad de que si $A$ es invertible, entonces $(A|I_n)\sim (I_n|A^{-1})$ y operaciones por filas para determinar $A^{-1}$.}
    Puesto que $A$ es invertible, se tiene que 
    \[
        (A|I_2)\sim (I_2|A^{-1}).
    \]
    Utilicemos esto en cada caso:
    \begin{enumerate}
        \item Notemos que 
        \[
           (A|I_2) =  \begin{pmatrix}
                2 & -1 & | & 1 & 0\\ 3 & 0 & | & 0 & 1
            \end{pmatrix}.
        \]
            Así, aplicando operaciones de fila, tenemos que 
        \begin{align*}
            \begin{pmatrix}
                2 & -1 & | & 1 & 0\\[0.75em]
                3 & 0 & | & 0 & 1
            \end{pmatrix}& \sim \begin{pmatrix}
                3 & 0 & | & 0 & 1 \\[0.75em]
                2 & -1 & | & 1 & 0
            \end{pmatrix} && F_1 \leftrightarrow F_2,  \\[0.25em]
            &\sim \begin{pmatrix}
                1 & 0 & | & 0 & \dfrac{1}{3} \\[0.75em]
                2 & -1 & | & 1 & 0
            \end{pmatrix} && \dfrac{1}{3}F_1 \rightarrow F_1, \\[0.25em]
            & \sim \begin{pmatrix}
                1 & 0 & | & 0 & \dfrac{1}{3} \\[0.75em]
                0 & -1 & | & 1 & -\dfrac{2}{3}
            \end{pmatrix} && F_2 - 2 F_1 \rightarrow F_2, \\[0.25em]
            & \sim \begin{pmatrix}
                1 & 0 & | & 0 & \dfrac{1}{3} \\[0.75em]
                0 & 1 & | & -1 & \dfrac{2}{3}
            \end{pmatrix} && -F_2 \rightarrow F_2.
        \end{align*}
        Así, 
        \[
            (I_2|A^{-1}) = \begin{pmatrix}
                1 & 0 & | & 0 & \dfrac{1}{3} \\[0.75em]
                0 & 1 & | & -1 & \dfrac{2}{3}
            \end{pmatrix},
        \]
        y, por lo tanto 
        \[
            A^{-1} = \begin{pmatrix}
                0 & \dfrac{1}{3} \\[0.75em]
                -1 & \dfrac{2}{3}
            \end{pmatrix}.
        \]
    \item Notemos que 
        \[
           (A|I_3) =  \begin{pmatrix}
                1 & 0 & 3 & | & 1 & 0 & 0 \\
                0 & -2 & 1 & | & 0 & 1 & 0 \\
                2 & 2 & 4 & | & 0 & 0 & 1 
            \end{pmatrix}.
        \]
        Así, aplicando operaciones de fila, tenemos que 
        \begin{align*}
            \begin{pmatrix}
                1 & 0 & 3 & | & 1 & 0 & 0 \\
                0 & -2 & 1 & | & 0 & 1 & 0 \\
                2 & 2 & 4 & | & 0 & 0 & 1 
            \end{pmatrix} & \sim \begin{pmatrix}
                1 & 0 & 3 & | & 1 & 0 & 0 \\
                0 & -2 & 1 & | & 0 & 1 & 0 \\
                0 & 2 & -2 & | & -2 & 0 & 1 
            \end{pmatrix} && F_3 - 2 F_1 \rightarrow F_3 \\
            & \sim \begin{pmatrix}
                1 & 0 & 3 & | & 1 & 0 & 0 \\
                0 & -2 & 1 & | & 0 & 1 & 0 \\
                0 & 0 & -1 & | & -2 & 1 & 1 
            \end{pmatrix} && F_3 + F_2 \rightarrow F_3 \\
            & \sim \begin{pmatrix}
                1 & 0 & 3 & | & 1 & 0 & 0 \\
                0 & 1 & -\dfrac{1}{2} & | & 0 & -\dfrac{1}{2} & 0 \\
                0 & 0 & -1 & | & -2 & 1 & 1 
            \end{pmatrix} && -\dfrac{1}{2} F_2 \rightarrow F_2 \\
            & \sim \begin{pmatrix}
                1 & 0 & 3 & | & 1 & 0 & 0 \\
                0 & 1 & -\dfrac{1}{2} & | & 0 & -\dfrac{1}{2} & 0 \\
                0 & 0 & 1 & | & 2 & -1 & -1 
            \end{pmatrix} && - F_3 \rightarrow F_3 \\
            & \sim \begin{pmatrix}
                1 & 0 & 0 & | & -5 & 3 & 3 \\
                0 & 1 & -\dfrac{1}{2} & | & 0 & -\dfrac{1}{2} & 0 \\
                0 & 0 & 1 & | & 2 & -1 & -1 
            \end{pmatrix} && F_1 - 3 F_3 \rightarrow F_1 \\
            & \sim \begin{pmatrix}
                1 & 0 & 0 & | & -5 & 3 & 3 \\
                0 & 1 & 0 & | & 1 & -1 & -\dfrac{1}{2} \\
                0 & 0 & 1 & | & 2 & -1 & -1 
            \end{pmatrix} && F_2 + \dfrac{1}{2} F_3 \rightarrow F_2 \\
        \end{align*}
        Así, 
        \[
            (I_3|A^{-1}) = \begin{pmatrix}
                -5 & 3 & 3 \\
                 1 & -1 & -\dfrac{1}{2} \\
                 2 & -1 & -1 
            \end{pmatrix},
        \]
        y, por lo tanto, 
        \[
            A^{-1} = \begin{pmatrix}
                -5 & 3 & 3 \\
                 1 & -1 & -\dfrac{1}{2} \\
                 2 & -1 & -1 
            \end{pmatrix}. \qedhere
        \]
    \end{enumerate}
\end{proof}

%%%%%%%%%%%%%%%%%%%%%%%%%%%%%%%%%%%%%%%%
\begin{ejer}
    Sea la matriz:
    \[
        A= \begin{pmatrix}
            \alpha & -1 & 0 \\
            -2 & \alpha & -2 \\
            0 & -1 & \alpha
        \end{pmatrix},
    \]
    donde $\alpha \in \mathbb{R}$.
    \begin{enumerate}
    \item 
        ¿Para qu\'{e} valores de $\alpha$ la matriz $A$ es invertible? 
    \item 
        Calcule la inversa de $A$ cuando sea posible.
    \end{enumerate}
\end{ejer}

\begin{proof}[Solución]\hspace{0pt}
    \begin{enumerate}
        \item Supongamos que la matriz es invertible, por lo tanto como $A\in\R^{3\times 3}$, se tiene que es invertible si y solo si $\rang(A)=3$. Así, calculemos la matriz reducida por filas equivalente.
        \begin{align*}
            \begin{pmatrix}
                \alpha & -1 & 0 \\
                -2 & \alpha & -2 \\
                0 & -1 & \alpha
            \end{pmatrix}
         & \sim \begin{pmatrix}
                \alpha & -1 & 0 \\
                0 & \dfrac{\alpha^2-2}{\alpha} & -2 \\
                0 & -1 & \alpha 
            \end{pmatrix} && F_2 + \dfrac{2}{\alpha} F_1 \rightarrow F_2 \\
        &\sim \begin{pmatrix}
                \alpha & -1 & 0 \\
                0 & \dfrac{\alpha^2-2}{\alpha} & -2 \\
                0 & 0 & \dfrac{\alpha^3-4\alpha}{\alpha^2-2}
            \end{pmatrix} && F_3 + \dfrac{\alpha}{\alpha^2-2} F_2 \rightarrow F_3.
        \end{align*}
        Así, para que $\rang(A)=3$, se necesita que 
        \[
            \dfrac{\alpha^3-4\alpha}{\alpha^2-2} \neq 0.
        \]
        De donde, 
        \[
            \alpha^3-4\alpha = \alpha ( \alpha^2-4) = \alpha (\alpha-2)(\alpha+2)\neq 0.
        \]
        Por lo tanto, $A$ es invertible si y solo si $\alpha \neq-2$, $\alpha\neq0$ y $\alpha\neq2$.
    % \comentario{Igual que en el ejercicio anterior, utilizar el la propiedad del rango para determinar si la matriz es invertible y calcularla por operaciones por filas. Dejo la solución original para ver la respuesta.}
    % \begin{enumerate}
    % \item 
    % Para saber si A es invertible, se debe calcular el determinante: 
    % \[
    % det(A)=\begin{vmatrix}
    %             \alpha & -1 & 0 \\
    %             -2 & \alpha & -2 \\
    %             0 & -1 & \alpha
    %             \end{vmatrix}=\alpha \begin{vmatrix}
    %             \alpha & -2 \\
    %             -1 & \alpha
    %             \end{vmatrix}-\begin{vmatrix}
    %             -2 & -2 \\
    %             0 & \alpha
    %             \end{vmatrix}=\alpha^3-4\alpha.\\
    % \]
    % Por lo tanto, $A$ es invertible si y solo si $\alpha \neq-2$, $\alpha\neq0$ y $\alpha\neq2$.
    
    \item Supongamos que $\alpha \in \mathbb{R}-\{-2, 0, 2\}$. Entonces, $A^{-1}$ existe. Así, como $A$ es invertible, se tiene que 
    \[
        (A|I_3)\sim (I_3|A^{-1}).
    \]
    Es decir, 
    \[
        (A|I_3) =  \begin{pmatrix}
                \alpha & -1 & 0 & | & 1 & 0 & 0 \\
                -2 & \alpha & -2 & | & 0 & 1 & 0\\
                0 & -1 & \alpha & | & 0 & 0 & 1
            \end{pmatrix}
    \]
    Por lo tanto, aplicando operaciones por filas
    \begin{align*}
         \begin{pmatrix}
                \alpha & -1 & 0 & | & 1 & 0 & 0 \\
                -2 & \alpha & -2 & | & 0 & 1 & 0\\
                0 & -1 & \alpha & | & 0 & 0 & 1
            \end{pmatrix} &\sim  \begin{pmatrix}
                \alpha & -1 & 0 & | & 1 & 0 & 0 \\
                0 & \dfrac{\alpha^2-2}{\alpha} & -2 & | & \dfrac{2}{\alpha} & 1 & 0\\
                0 & -1 & \alpha & | & 0 & 0 & 1
            \end{pmatrix} && F_2 + \dfrac{2}{\alpha} F_1 \rightarrow F_2, \\
            & \sim \begin{pmatrix}
                \alpha & -1 & 0 & | & 1 & 0 & 0 \\
                0 & \dfrac{\alpha^2-2}{\alpha} & -2 & | & \dfrac{2}{\alpha} & 1 & 0\\
                0 & 0 & \dfrac{\alpha^3-4\alpha}{\alpha^2-2} & | & \dfrac{2}{\alpha^2-2} & \dfrac{\alpha}{\alpha^2-2} & 1
            \end{pmatrix} && F_3 + \dfrac{\alpha}{\alpha^2-2} F_2 \rightarrow F_3, \\
            & \sim \begin{pmatrix}
                \alpha & -1 & 0 & | & 1 & 0 & 0 \\
                0 & \dfrac{\alpha^2-2}{\alpha} & -2 & | & \dfrac{2}{\alpha} & 1 & 0\\
                0 & 0 & 1 & | & \dfrac{2}{\alpha^3-4\alpha} & \dfrac{\alpha}{\alpha^3-4\alpha} & \dfrac{\alpha^2-2}{\alpha^3-4\alpha}
            \end{pmatrix} && \dfrac{\alpha^2-2}{\alpha^3-4\alpha} F_3 \rightarrow F_3, \\
            & \sim \begin{pmatrix}
                \alpha & -1 & 0 & | & 1 & 0 & 0 \\
                0 & \dfrac{\alpha^2-2}{\alpha} & 0 & | & \dfrac{2(\alpha^2-2)}{\alpha(\alpha^2-4)} & \dfrac{\alpha^2-2}{\alpha^2-4} & \dfrac{2(\alpha^2-2)}{\alpha^3-4\alpha}\\
                0 & 0 & 1 & | & \dfrac{2}{\alpha^3-4\alpha} & \dfrac{\alpha}{\alpha^3-4\alpha} & \dfrac{\alpha^2-2}{\alpha^3-4\alpha}
            \end{pmatrix} &&  F_2 + 2 F_3 \rightarrow F_2, \\
            & \sim \begin{pmatrix}
                \alpha & -1 & 0 & | & 1 & 0 & 0 \\
                0 & 1 & 0 & | & \dfrac{2}{\alpha^2-4} & \dfrac{\alpha}{\alpha^2-4} & \dfrac{2\alpha}{\alpha^3-4\alpha}\\
                0 & 0 & 1 & | & \dfrac{2}{\alpha^3-4\alpha} & \dfrac{\alpha}{\alpha^3-4\alpha} & \dfrac{\alpha^2-2}{\alpha^3-4\alpha}
            \end{pmatrix} && \dfrac{\alpha}{\alpha^2-2}F_2 \rightarrow F_2, \\
            & \sim \begin{pmatrix}
                \alpha & 0 & 0 & | & \dfrac{\alpha^2-2}{\alpha^2-4} & \dfrac{\alpha}{\alpha^2-4} & \dfrac{2\alpha}{\alpha^3-4\alpha}\\
                0 & 1 & 0 & | & \dfrac{2}{\alpha^2-2} & \dfrac{\alpha}{\alpha^2-4} & \dfrac{2}{\alpha^2-4}\\
                0 & 0 & 1 & | & \dfrac{2}{\alpha^3-4\alpha} & \dfrac{1}{\alpha^2-4} & \dfrac{\alpha^2-2}{\alpha^3-4\alpha}
            \end{pmatrix} && F_1 + F_2 \rightarrow F_1, \\
            & \sim \begin{pmatrix}
                1 & 0 & 0 & | & \dfrac{\alpha^2-2}{\alpha(\alpha^2-4)} & \dfrac{1}{\alpha^2-4} & \dfrac{2}{\alpha^3-4\alpha}\\
                0 & 1 & 0 & | & \dfrac{2}{\alpha^2-2} & \dfrac{\alpha}{\alpha^2-4} & \dfrac{2}{\alpha^2-4}\\
                0 & 0 & 1 & | & \dfrac{2}{\alpha^3-4\alpha} & \dfrac{1}{\alpha^2-4} & \dfrac{\alpha^2-2}{\alpha^3-4\alpha}
            \end{pmatrix} && \dfrac{1}{\alpha} F_1 \rightarrow F_1.
    \end{align*}
    Así, 
    \[
        (I_3|A^{-1}) = \begin{pmatrix}
                1 & 0 & 0 & | & \dfrac{\alpha^2-2}{\alpha(\alpha^2-4)} & \dfrac{1}{\alpha^2-4} & \dfrac{2}{\alpha^3-4\alpha}\\
                0 & 1 & 0 & | & \dfrac{2}{\alpha^2-2} & \dfrac{\alpha}{\alpha^2-4} & \dfrac{2}{\alpha^2-4}\\
                0 & 0 & 1 & | & \dfrac{2}{\alpha^3-4\alpha} & \dfrac{1}{\alpha^2-4} & \dfrac{\alpha^2-2}{\alpha^3-4\alpha}
            \end{pmatrix},
    \]
    y, por lo tanto, 
    \[
        A^{-1} = \begin{pmatrix}
                \dfrac{\alpha^2-2}{\alpha(\alpha^2-4)} & \dfrac{1}{\alpha^2-4} & \dfrac{2}{\alpha^3-4\alpha}\\
                \dfrac{2}{\alpha^2-2} & \dfrac{\alpha}{\alpha^2-4} & \dfrac{2}{\alpha^2-4}\\
                \dfrac{2}{\alpha^3-4\alpha} & \dfrac{1}{\alpha^2-4} & \dfrac{\alpha^2-2}{\alpha^3-4\alpha}
            \end{pmatrix}. \qedhere
    \]
    % tenemos:
    % \[
    % A^{-1}=\frac{1}{det(A)}Adj(A)=\frac{1}{\alpha^3-4\alpha}   \begin{pmatrix}
    %             \alpha^2-2 & \alpha & 2 \\
    %             2\alpha & \alpha^2 & 2\alpha \\
    %             2 & \alpha & \alpha^2-2
    %         \end{pmatrix}
    % \]
    \end{enumerate}
\end{proof}


\end{document}