\documentclass[a4,11pt]{aleph-notas}

% -- Paquetes adicionales 
\usepackage{enumitem}
\usepackage{aleph-comandos}
\usepackage{systeme}

% -- Datos  
\institucion{Facultad de Ciencias Exactas, Naturales y Ambientales}
\carrera{Catálogo STEM}
\asignatura{Álgebra Lineal y Geometría Analítica}
\tema[Ejercicios resueltos no. 6: Indep. Lineal y Conj. Generador]{Ejercicios resueltos no. 6: Independencia Lineal y Conjunto Generador}
\autor{Andrés Merino}
\fecha{Periodo 2025-1}

\logouno[0.14\textwidth]{Logos/logoPUCE_04_ac}
\definecolor{colortext}{HTML}{0030A1}
\definecolor{colordef}{HTML}{0030A1}
\fuente{montserrat}

% -- Comandos adicionales
\setlist[enumerate]{label=\roman*.}

\begin{document}

\encabezado

%%%%%%%%%%%%%%%%%%%%%%%%%%%%%%%%%%%%%%%%
\begin{ejer}
    En el espacio vectorial $\R^{2}$, sean:
    \[
        \begin{array}{cccc}
            v_1 = \begin{pmatrix} 1\\ 3 \end{pmatrix}, &
            v_2 = \begin{pmatrix} 2\\ -3 \end{pmatrix} &
            \text{ y } &
            v_3 = \begin{pmatrix} 0\\ 2 \end{pmatrix}.
        \end{array}
    \]
    ¿Son los vectores $v_1$, $v_2$ y $v_3$ linealmente independientes?
\end{ejer}

\begin{proof}[Solución]\hspace{0pt}
    Tomemos $\alpha_1, \alpha_2$ y $\alpha_3\in\R$ y planteamos el siguiente sistema lineal homogéneo
    \[
        \alpha_1 v_1 + \alpha_2 v_2 + \alpha_3 v_3 = 0_V,
    \]
    o equivalentemente
    \[
        \begin{array}{ccccccc}
            \alpha_1 & + & 2\alpha_2 & & & = & 0;\\
            3\alpha_1 & - & 3\alpha_2 & + & 2\alpha_3& = & 0;
        \end{array}
    \]
    cuya matriz adjunta en forma escalonada por filas es
    \[
        \begin{pmatrix}
            1 & 2 & 0 & | & 0\\
            0 & -9 & 2 & | & 0
        \end{pmatrix}.
    \]
    Entonces, $\alpha_1 = -2\alpha_2$ y $\alpha_3 = \dfrac{9}{2}\alpha_2$, donde $\alpha_2\in\R$. Escogiendo $\alpha_{2} = 2$, encontramos la solución no trivial $\alpha_{1}=-4$, $\alpha_2=2$ y $\alpha_3 = 9$. Por lo tanto, $v_1$, $v_2$ y $v_3$ son linealmente dependientes.
\end{proof}

%%%%%%%%%%%%%%%%%%%%%%%%%%%%%%%%%%%%%%%%
\begin{ejer}
    En el espacio vectorial $\R^{3}$, sean:
    \[
        \begin{array}{cccc}
            v_1 = \begin{pmatrix} 2\\ 2\\ 3 \end{pmatrix}, &
        v_2 = \begin{pmatrix} -1\\ -2\\ 1 \end{pmatrix} &
        \text{ y } &
        v_3 = \begin{pmatrix} 0\\ 1\\ 0 \end{pmatrix}.
        \end{array}
    \]
    ¿Son los vectores $v_1$, $v_2$ y $v_3$ linealmente independientes?
\end{ejer}

\begin{proof}[Solución]\hspace{0pt}
    Tomemos $\alpha_1, \alpha_2$ y $\alpha_3\in\R$ y planteamos el siguiente sistema lineal homogéneo
    \[
        \alpha_1 v_1 + \alpha_2 v_2 + \alpha_3 v_3 = 0_V,
    \]
    o equivalentemente
    \[
        \begin{array}{ccccccc}
            2\alpha_1 & - & \alpha_2 &  & & = & 0;\\
            2\alpha_1 & - & 2\alpha_2 & + & \alpha_3& = & 0;\\
            3\alpha_1 & + & \alpha_2 &  & & = & 0;
        \end{array}
    \]
    cuya matriz adjunta en forma escalonada reducida por filas es
    \[
        \begin{pmatrix}
            1 & 0 & 0 & | & 0\\
            0 & 1 & 0 & | & 0\\
            0 & 0 & 1 & | & 0
        \end{pmatrix}.
    \]
    Entonces, el sistema posee solución única, es decir, $\alpha_1 = \alpha_2 = \alpha_3 = 0$. Por lo tanto, $v_1$, $v_2$ y $v_3$ son linealmente independientes.
\end{proof}

%%%%%%%%%%%%%%%%%%%%%%%%%%%%%%%%%%%%%%%%
\begin{ejer}
    En el espacio vectorial $\R_2[t]$, sean:
    \[
        \begin{array}{cccc}
            p_1(t) = t^2 + 1, & 
            p_2(t) = t - 2 &
            \text{ y } &
            p_3(t) = t + 3.
        \end{array}
    \]
    ¿Son los vectores $p_1(t)$, $p_2(t)$ y $p_3(t)$ linealmente independientes?
\end{ejer}

\begin{proof}[Solución]\hspace{0pt}
    Tomemos $\alpha_1, \alpha_2$ y $\alpha_3\in\R$ y planteamos la combinación lineal nula
    \[
        \alpha_1 p_1(t) + \alpha_2 p_2(t) + \alpha_3 p_3(t) = 0t^2 + 0t + 0,
    \]
    a partir de lo cual se tiene que
    \[
        \alpha_1(t^2 + 1) + \alpha_2(t - 2) + \alpha_3(t + 3) = 0t^2 + 0t + 0,
    \]
    agrupando términos, se obtiene
    \[
        \alpha_1t^2 + (\alpha_2 + \alpha_3)t + (\alpha_1 - 2\alpha_2 + 3\alpha_3) = 0t^2 + 0t + 0,
    \]
    es decir, se obtiene el sistema lineal homogéneo
    \[
        \begin{array}{ccccccc}
            \alpha_1 &  &  &  & & = & 0;\\
             &  & \alpha_2 & + & \alpha_3& = & 0;\\
            \alpha_1 & - & 2\alpha_2 & + & 3\alpha_3& = & 0;
        \end{array}
    \]
    cuya matriz adjunta en forma escalonada reducida por filas es
    \[
        \begin{pmatrix}
            1 & 0 & 0 & | & 0\\
            0 & 1 & 0 & | & 0\\
            0 & 0 & 1 & | & 0
        \end{pmatrix}.
    \]
    Entonces, el sistema posee solución única, es decir, $\alpha_1 = \alpha_2 = \alpha_3 = 0$. Por lo tanto, $p_1(t)$, $p_2(t)$ y $p_3(t)$ son linealmente independientes.
\end{proof}

%%%%%%%%%%%%%%%%%%%%%%%%%%%%%%%%%%%%%%%%
\begin{ejer}
    ¿Cuáles de los siguientes conjuntos de vectores generan a $\R^3$?
\begin{enumerate}
        \item $S=\{(1, 1, 0), (3, 4, 2)\}$
        \item $S=\{(1, 1, 0), (0, 1, 0), (2, 2, 2)\}$
\end{enumerate}
\end{ejer}

\begin{proof}[Solución]\hspace{0pt}
    \begin{enumerate}
        \item Sea $(a, b, c) \in \R^3$, queremos examinar si existen $\alpha_1$, $\alpha_2$ $\in \R$ tales que  
        \[
        \alpha_{1} (1, 1, 0) + \alpha_{2} (3, 4, 2) =(a, b, c).
        \]
        La ecuación anterior conduce al sistema lineal
        \[
        \begin{array}{rrrrrrr}
         \alpha _{1} & + & 3\alpha_{2} & = & a; \\
         \alpha _{1} & + & 4\alpha_{2} & = & b;\\
         & & 2\alpha_{2} & = & c;
         \end{array}
         \]
         de donde, resolviendo obtenemos
         \begin{align*}
        \begin{pmatrix}
            1&3&|&a\\
            1&4&|&b\\
            0&2&|&c\\
        \end{pmatrix}
        & \sim 
        \begin{pmatrix}
            1&3&|&a\\
            0&1&|&-a+b\\
            0&0&|&2a - 2b +c\\
        \end{pmatrix},
        \end{align*}
        con lo cual, notamos que, si $2a - 2b + c \neq 0$, el sistema no tiene solución. Por lo tanto no existe solución para cualquier elección de $a, b, c$, y se concluye que $S$ no genera a $\R^3$.
        
        \item Sea $(a, b, c) \in \R^3$, queremos examinar si existen $\alpha_1$, $\alpha_2$, $\alpha_3$  $\in \R$ tales que  
        \[
        \alpha_{1} (1, 1, 0) + \alpha_{2} (0, 1, 0) + \alpha_{3} (2, 2, 2) =(a, b, c).
        \]
        La ecuación anterior conduce al sistema lineal
        \[
        \begin{array}{rrrrrrr}
         \alpha _{1} & + & & + & 2\alpha_{3} = & a, \\
         \alpha _{1} & + &\alpha_{2}& + &2\alpha_{3}= & b,\\
         & & & &2\alpha_{3}= & c,
         \end{array}
         \]
         de donde, resolviendo obtenemos
        \begin{align*}
        \begin{pmatrix}
            1&0&2&|&a\\
            1&1&2&|&b\\
            0&0&2&|&c\\
        \end{pmatrix}
        & \sim 
        \begin{pmatrix}
            1&0&0&|&a-c\\
            0&1&0&|&-a+b\\
            0&0&1&|&\frac{c}{2}\\
        \end{pmatrix},
        \end{align*}
        con lo cual, notamos que, existe solución para cualquier elección de $a, b, c$, y por lo tanto $\R^2 = gen (S)$. \qedhere
    \end{enumerate}
\end{proof}

\end{document}