\documentclass[a4,11pt]{aleph-notas}
% Se puede ver la documentación aquí: 
% https://github.com/alephsub0/LaTeX_aleph-notas 

% -- Paquetes adicionales 
\usepackage{enumitem}
\usepackage{array,booktabs,multirow,makecell}
\usepackage{colortbl}
\usepackage{longtable}
\usepackage{url}
\usepackage{pdflscape}
\usepackage{amsmath,amssymb}

% -- Datos 
\institucion{Facultad de Ciencias Exactas, Naturales y Ambientales}
\carrera{Catálogo STEM}
\asignatura{Álgebra Lineal y Geometría Analítica}
\tema{Cronograma y actividades}
\autor{Andrés Merino}
\fecha{Semestre 2025-1}

\logouno[0.14\textwidth]{Logos/logoPUCE_04_ac}
\definecolor{colortext}{HTML}{0030A1}
\definecolor{colordef}{HTML}{0030A1}
\fuente{montserrat}


% -- Comandos para tablas
\newcolumntype{C}[1]{>{\hspace{0pt}\centering\arraybackslash}p{#1}}
\newcolumntype{L}[1]{>{\raggedright\arraybackslash}p{#1}}

\definecolor{verde}{RGB}{0, 255, 127}
\definecolor{celeste}{RGB}{68,195,218}

\begin{document}
\addtolength{\headheight}{1.8\baselineskip}
\addtolength{\voffset}{-1.5\baselineskip}

\encabezado

%%%%%%%%%%%%%%%%%%%%%%%%%%%%%%%%%%%%%%%%
\section{Resultados de aprendizaje} 
%%%%%%%%%%%%%%%%%%%%%%%%%%%%%%%%%%%%%%%%

\begin{itemize}[leftmargin=*]
\item 
    \textbf{RdA 1:} Comprender los conceptos básicos del Álgebra Lineal y Geometría Analítica en el campo de la Ingeniería.
    \begin{itemize}[leftmargin=*]
        \item \textbf{Criterio 1.1:}  Identifica cuándo un Sistema de Ecuaciones Lineales es compatible o no, y cómo se relacionan con la representación gráfica de las ecuaciones.
        \item \textbf{Criterio 1.2:} Distingue los conceptos básicos relacionados con aplicaciones lineales.
        \item \textbf{Criterio 1.3:} Describe los conceptos de la Geometría Analítica en $R^2$.
    \end{itemize}
\item 
    \textbf{RdA 2:} Analizar los problemas relacionados al Álgebra Lineal y Geometría Analítica en el campo de la Ingeniería.
    \begin{itemize}[leftmargin=*]
        \item \textbf{Criterio 2.1:} Resuelve Sistemas de Ecuaciones Lineales a través de matrices y determinantes.
        \item \textbf{Criterio 2.2:} Resuelve operaciones con aplicaciones lineales.
        \item \textbf{Criterio 2.3:} Resuelve problemas asociados a la Geometría Analítica.
    \end{itemize}
\item
    \textbf{RdA 3:} Aplicar distintos tópicos del Álgebra Lineal y la Geometría Analítica en el campo de la Ingeniería.
    \begin{itemize}[leftmargin=*]
        \item \textbf{Criterio 3.1:} Modela distintas situaciones a través de Sistemas de Ecuaciones Lineales.
        \item \textbf{Criterio 3.2:} Aplica los conceptos de transformaciones lineales al campo de la Ingeniería.
        \item \textbf{Criterio 3.3:} Aplica los conceptos de la Geometría Analítica al campo de la Ingeniería.
    \end{itemize}
\end{itemize}

%%%%%%%%%%%%%%%%%%%%%%%%%%%%%%%%%%%%%%%%
\section{Contenidos generales} 
%%%%%%%%%%%%%%%%%%%%%%%%%%%%%%%%%%%%%%%%

\begin{itemize}
\item 
    Matrices
\item 
    Sistemas de ecuaciones
\item 
    Determinantes
\item 
    Espacios vectoriales
\item 
    Aplicaciones lineales
\item 
    Valores y vectores propios
\item 
    Cónicas
\end{itemize}

%%%%%%%%%%%%%%%%%%%%%%%%%%%%%%%%%%%%%%%%
\section{Actividades de evaluación} 
%%%%%%%%%%%%%%%%%%%%%%%%%%%%%%%%%%%%%%%%

\begin{itemize}[leftmargin=*]
    \item \textbf{Criterio 1.1}
        \begin{itemize}[leftmargin=*]
            \item \textbf{Cuestionario en línea 1 (100\%):} Evaluará la capacidad para identificar cuándo un sistema de ecuaciones lineales es compatible o no, mediante preguntas de opción múltiple y ejercicios gráficos breves.
        \end{itemize}
        
    \item \textbf{Criterio 1.2}
        \begin{itemize}[leftmargin=*]
            \item \textbf{Cuestionario en línea 2 (100\%):} Medirá la habilidad para distinguir los conceptos básicos relacionados con aplicaciones lineales mediante preguntas conceptuales y análisis de afirmaciones breves.
        \end{itemize}
        
    \item \textbf{Criterio 1.3}
        \begin{itemize}[leftmargin=*]
            \item \textbf{Cuestionario en línea 3 (100\%):} Evaluará la capacidad para describir los conceptos básicos de la Geometría Analítica en \( \mathbb{R}^2 \), mediante preguntas cortas de desarrollo y ejercicios gráficos breves.
        \end{itemize}

    \item \textbf{Criterio 2.1}
        \begin{itemize}[leftmargin=*]
            \item \textbf{Examen 1 (100\%):} Evaluará la capacidad para resolver sistemas de ecuaciones lineales utilizando matrices y determinantes mediante ejercicios prácticos de resolución y análisis.
        \end{itemize}

    \item \textbf{Criterio 2.2}
        \begin{itemize}[leftmargin=*]
            \item \textbf{Examen 2 - parte 1 (100\%):} Medirá la capacidad para resolver operaciones relacionadas con aplicaciones lineales, incluyendo análisis del núcleo e imagen, mediante problemas prácticos y breves justificaciones teóricas.
        \end{itemize}
    
    \item \textbf{Criterio 2.3}
        \begin{itemize}[leftmargin=*]
            \item \textbf{Examen 2 - parte 2 (100\%):} Evaluará la capacidad para resolver problemas específicos asociados a la Geometría Analítica, especialmente cónicas, mediante ejercicios analíticos y gráficos.
        \end{itemize}

    \item \textbf{Criterio 3.1}
        \begin{itemize}[leftmargin=*]
            \item \textbf{Reto 1 (100\%):} Consistirá en la modelación de situaciones reales a través de sistemas de ecuaciones lineales, utilizando herramientas computacionales y análisis crítico de resultados.
        \end{itemize}

    \item \textbf{Criterio 3.2}
        \begin{itemize}[leftmargin=*]
            \item \textbf{Reto 2 - parte 1 (100\%):} Los estudiantes aplicarán transformaciones lineales para resolver problemas concretos de ingeniería, enfatizando el análisis y discusión de resultados obtenidos.
        \end{itemize}

    \item \textbf{Criterio 3.3}
        \begin{itemize}[leftmargin=*]
            \item \textbf{Reto 2 - parte 2 (100\%):} Evaluará la aplicación práctica de los conceptos de Geometría Analítica en situaciones reales del campo ingenieril, mediante ejercicios integradores donde se usarán técnicas matriciales y vectoriales.
        \end{itemize}
\end{itemize}



\begin{landscape}
%%%%%%%%%%%%%%%%%%%%%%%%%%%%%%%%%%%%%%%%
\section{Cronograma de Desarrollo del Curso} 
%%%%%%%%%%%%%%%%%%%%%%%%%%%%%%%%%%%%%%%%

\begin{center}\small
\setlength{\extrarowheight}{0ex}
\setlength{\belowrulesep}{.6ex}
\begin{longtable}{cccL{12cm}L{7.5cm}}
    \toprule
    &&\thead{Fecha}&\thead{Detalle de contenido} & \thead{Observación} \\
    \midrule
  \endfirsthead
    \multicolumn{5}{l}{\footnotesize \ldots viene de la página anterior}\\
    \toprule
    &&\thead{Fecha}&\thead{Detalle de contenido} & \thead{Observación} \\
    \midrule
  \endhead
        \bottomrule  \multicolumn{5}{r}{\footnotesize Continúa en la siguiente página\ldots}
  \endfoot
        \bottomrule
  \endlastfoot
1	&	1	&	1-ene	&	Matrices: Definiciones, operaciones y propiedades fundamentales	&		\\ \midrule	
2	&	2	&	1-abr	&	Sistemas lineales: conceptos, definiciones y métodos de resolución	&		\\ \midrule	
3	&	3	&	2-abr	&	Determinantes: definición, propiedades y resolución de sistemas	&	Cuestionario en línea 1	\\ \midrule	
4	&	4	&	3-abr	&	Vectores en el plano y en $\mathbb{R}$: operaciones básicas, interpretación geométrica	&	Entrega Reto 1	\\ \midrule	
5	&	5	&	7-abr	&	Ecuación vectorial y paramétrica de rectas y planos	&		\\ \midrule	
6	&	6	&	8-abr	&	Espacios vectoriales y subespacios: conceptos básicos y ejemplos	&		\\ \midrule	
7	&	7	&	9-abr	&	Independencia lineal, conjuntos generadores y bases	&		\\ \midrule	\rowcolor{celeste!50}
8	&	8	&	10-abr	&	Evaluación	&	Examen 2	\\ \midrule	
9	&	9	&	14-abr	&	Aplicaciones lineales: conceptos fundamentales, definición formal	&		\\ \midrule	
10	&	10	&	15-abr	&	Núcleo, imagen y matriz asociada a una aplicación lineal	&		\\ \midrule	
11	&	11	&	16-abr	&	Valores y vectores propios: concepto, métodos y análisis	&	Cuestionario en línea 2	\\ \midrule	
12	&	12	&	17-abr	&	Diagonalización de matrices: procedimiento y aplicaciones	&		\\ \midrule	
13	&	13	&	21-abr	&	Cónicas, definición matricial, clasificación 	&		\\ \midrule	
14	&	14	&	22-abr	&	Producto interno, ortogonalidad y proyecciones	&		\\ \midrule	
15	&	15	&	23-abr	&	Resolución de problemas con cónicas aplicando métodos vectoriales y matriciales	&	Cuestionario en línea 3	\\ \midrule	\rowcolor{celeste!50}
16	&	16	&	24-abr	&	Evaluación	&	Examen 2; Entrega Reto 2	\\ 
\end{longtable}
\end{center}
\end{landscape}

\end{document} 