\documentclass[a4,11pt]{aleph-notas}

% -- Datos 
\tema{Clase 00: Nombre de la clase}


\begin{document}

\encabezado
% En todo el documento, las indicaciones deben ser simples y directar, con una sola oración.

%%%%%%%%%%%%%%%%%%%%%%%%%%%%%%%%%%%%%%%%
\section*{Resultado de Aprendizaje}
%%%%%%%%%%%%%%%%%%%%%%%%%%%%%%%%%%%%%%%%

%%%%%%%%%%%%%%%%%%%%%%%%%%%%%%%%%%%%%%%%
\subsection*{RdA de la asignatura:}
% Se toma uno o más de los siguientes
\begin{itemize}[leftmargin=*]
    \item \textbf{RdA 1:} Comprender los conceptos básicos del Álgebra Lineal y Geometría Analítica en el campo de la Ingeniería.
    \item \textbf{RdA 2:} Analizar los problemas relacionados al Álgebra Lineal y Geometría Analítica en el campo de la Ingeniería.
    \item \textbf{RdA 3:} Aplicar distintos tópicos del Álgebra Lineal y la Geometría Analítica en el campo de la Ingeniería.
\end{itemize}

%%%%%%%%%%%%%%%%%%%%%%%%%%%%%%%%%%%%%%%%
\subsection*{RdA de la actividad:}
% Máximo 3 resultados
\begin{itemize}[leftmargin=*]
    \item ...
\end{itemize}

%%%%%%%%%%%%%%%%%%%%%%%%%%%%%%%%%%%%%%%%
\section*{Introducción}
%%%%%%%%%%%%%%%%%%%%%%%%%%%%%%%%%%%%%%%%

%%%%%%%%%%%%%%%%%%%%%%%%%%%%%%%%%%%%%%%%
\paragraph{Pregunta inicial:} 
% Pregunta inicial llamativa y motivadora
...

%%%%%%%%%%%%%%%%%%%%%%%%%%%%%%%%%%%%%%%%
\section*{Desarrollo}
%%%%%%%%%%%%%%%%%%%%%%%%%%%%%%%%%%%%%%%%

%%%%%%%%%%%%%%%%%%%%%%%%%%%%%%%%%%%%%%%%
\subsection*{Actividad 1: Nombre de la actividad}

% Resumen de la actividad, incluyendo metodología (clase magistral, clase invertida, retos, clase interactiva con ChatGPT, visualización de videos, investigación en clase etc.)

\paragraph{¿Cómo lo haremos?}  
\begin{itemize}[leftmargin=*]
    \item \textbf{...:}  % Nombre del paso
    % Ejemplo (no tomar todos en todas las clases ni en el mismo orden):
    % \item Motivación: se presenta reto virales de redes sociales de sistemas con dibujos.
    % \item Clase magistral: se exponen los temas de Sistemas de cuaciones, tipos de soluciones, eliminación de Gauss-Jordan (usar \href{https://fcena-puce.github.io/FCENA-PUCE/AlgLinealyGeomAnalitica-05-N0068/2-Resumenes/Resumen02.pdf}{Resumen02.pdf})
    % \item Resolución de ejercicios: los estudiantes resolverán, de manera guiada ejercicios similares a los planteados en \href{https://fcena-puce.github.io/FCENA-PUCE/AlgLinealyGeomAnalitica-05-N0068/2-Ejercicios/Ejercicios02.pdf}{Ejercicios02.pdf}
    % \item Visulización de videos: se presenta y discute el siguiente video \href{youtube.com}{Resolución de Ecuaciones}
    % \item Implementación computacional: los estudiantes utilizarán el cuaderno de Jupyter \href{https://colab.research.google.com/github/FCENA-PUCE/AlgLinealyGeomAnalitica-05-N0068/blob/main/2-Notebooks/Solución de Ecuaciones.ipynb}{Solución de Ecuaciones.ipynb} para resolver sistemas complejos.
    % \item Implementación computacional: los estudiantes utilizarán la aplicación \href{https://matrixcalc.org/es/slu.html}{Matrix calculator} para resolver sistemas complejos.
\end{itemize}


\paragraph{Verificación de aprendizaje:}  
% Indicar actividades cortas que evalúen el aprendizaje, pueden ser una lista de 3 preguntas
...


%%%%%%%%%%%%%%%%%%%%%%%%%%%%%%%%%%%%%%%%
\section*{Cierre}
%%%%%%%%%%%%%%%%%%%%%%%%%%%%%%%%%%%%%%%%

\paragraph{Tarea:}
% Plantear resolución de ejercicio o lectura o similar.
...

\paragraph{Pregunta de investigación:}  
% Dos o tres preguntas que guíen al estudiante a profundizar más en los temas o buscar aplciaciones
\begin{enumerate}[leftmargin=*]
    \item ...
    \item ...
\end{enumerate}
    
\paragraph{Para la próxima clase:}  
% Actividad para la proxima clase
...

\end{document} 