\documentclass[a4,11pt]{aleph-notas}
% Se puede ver la documentación aquí: 
% https://github.com/alephsub0/LaTeX_aleph-notas

% -- Paquetes adicionales 
\usepackage{enumitem}
\usepackage{url}
\hypersetup{
    urlcolor=blue,
    linkcolor=blue,
}

% -- Datos 
\institucion{Facultad de Ciencias Exactas, Naturales y Ambientale}
\carrera{Catálogo STEM}
\asignatura{Álgebra Lineal y Geometría Analítica}
\tema{Clase 06: Independencia Lineal y Conjunto Generador}
\autor{Andrés Merino}
\fecha{Periodo 2025-1}

\logouno[0.14\textwidth]{Logos/logoPUCE_04_ac}
\definecolor{colortext}{HTML}{0030A1}
\definecolor{colordef}{HTML}{0030A1}
\fuente{montserrat}


% -- Comandos para tablas

\begin{document}

\encabezado


%%%%%%%%%%%%%%%%%%%%%%%%%%%%%%%%%%%%%%%%
\section*{Resultado de Aprendizaje}
%%%%%%%%%%%%%%%%%%%%%%%%%%%%%%%%%%%%%%%%

%%%%%%%%%%%%%%%%%%%%%%%%%%%%%%%%%%%%%%%%
\subsection*{RdA de la asignatura:}
\begin{itemize}[leftmargin=*]
    \item \textbf{RdA 1:} Comprender los conceptos básicos del Álgebra Lineal y Geometría Analítica en el campo de la Ingeniería.
    \item \textbf{RdA 2:} Analizar los problemas relacionados al Álgebra Lineal y Geometría Analítica en el campo de la Ingeniería.
    % \item \textbf{RdA 3:} Aplicar distintos tópicos del Álgebra Lineal y la Geometría Analítica en el campo de la Ingeniería.
\end{itemize}

%%%%%%%%%%%%%%%%%%%%%%%%%%%%%%%%%%%%%%%%
\subsection*{RdA de la actividad:}
\begin{itemize}[leftmargin=*]
    \item Reconocer cuándo un conjunto de vectores es linealmente dependiente o independiente.
    \item Determinar si un conjunto de vectores genera un espacio vectorial.
    \item Interpretar la noción de base y dimensión de un espacio vectorial.
\end{itemize}

%%%%%%%%%%%%%%%%%%%%%%%%%%%%%%%%%%%%%%%%
\section*{Introducción}
%%%%%%%%%%%%%%%%%%%%%%%%%%%%%%%%%%%%%%%%

%%%%%%%%%%%%%%%%%%%%%%%%%%%%%%%%%%%%%%%%
\paragraph{Pregunta inicial:} 
¿A qué se refiere cuando decimos que vivimos en un espacio tridimensional?

%%%%%%%%%%%%%%%%%%%%%%%%%%%%%%%%%%%%%%%%
\section*{Desarrollo}
%%%%%%%%%%%%%%%%%%%%%%%%%%%%%%%%%%%%%%%%

%%%%%%%%%%%%%%%%%%%%%%%%%%%%%%%%%%%%%%%%
\subsection*{Actividad 1: Independencia lineal y espacio generado}

\paragraph{¿Cómo lo haremos?}  
\begin{itemize}[leftmargin=*]
    \item \textbf{Puesta en común:} Resolución guiada y discusión de los ejercicios de independencia lineal enviados en la clase invertida.
    \item \textbf{Discusión del video:} Se repasan los conceptos clave del video sobre combinaciones lineales, espacio generado e independencia.
    \item \textbf{Clase magistral:} Exposición formal de los conceptos de conjunto generador, base y dimensión. Se utilizará el \href{https://fcena-puce.github.io/AlgLinealyGeomAnalitica-05-N0068/2-Resumenes/Resumen06.pdf}{Resumen06.pdf}.
    \item \textbf{Resolución de ejercicios:} Práctica de ejercicios similares a los disponibles en \href{https://fcena-puce.github.io/AlgLinealyGeomAnalitica-05-N0068/2-Ejercicios/Ejercicios06.pdf}{Ejercicios06.pdf}.
    \item \textbf{Recomendaciones de lectura:} 
    \begin{itemize}
        \item \href{https://blog.nekomath.com/algebra-lineal-i-combinaciones-lineales/}{Combinaciones lineales}
        \item \href{https://blog.nekomath.com/algebra-lineal-i-conjuntos-generadores-independencia-lineal-y-bases/}{Conjuntos generadores e independencia lineal}
        \item \href{https://blog.nekomath.com/algebra-lineal-i-problemas-de-combinaciones-lineales-generadores-e-independientes/}{Problemas de combinaciones lineales, generadores e independientes}
        \item \href{https://blog.nekomath.com/algebra-lineal-i-bases-y-dimension-de-espacios-vectoriales/}{Bases y dimensión de espacios vectoriales}
        \item \href{https://blog.nekomath.com/algebra-lineal-i-problemas-de-bases-y-dimension-de-espacios-vectoriales/}{Problemas de bases y dimensión de espacios vectoriales}
    \end{itemize}
    \item \textbf{Visualización de video:} \href{https://youtu.be/TgKwz5Ikpc8?si=7z5e-PTXFkTK7szh}{Espacios vectoriales abstractos}
\end{itemize}

\paragraph{Verificación de aprendizaje:}  
\begin{itemize}[leftmargin=*]
    \item ¿Qué significa que un conjunto de vectores sea linealmente dependiente?
    \item ¿Cuál es la relación entre base, conjunto generador e independencia?
    \item ¿Cómo se determina la dimensión de un espacio vectorial?
\end{itemize}

%%%%%%%%%%%%%%%%%%%%%%%%%%%%%%%%%%%%%%%%
\section*{Cierre}
%%%%%%%%%%%%%%%%%%%%%%%%%%%%%%%%%%%%%%%%

\paragraph{Tarea:}  
Resolver del libro \href{https://catalogobiblioteca.puce.edu.ec/cgi-bin/koha/opac-detail.pl?biblionumber=86083}{Álgebra lineal y sus aplicaciones de David C. Lay}, sección 4.3: ejercicios 1, 3, 5, 7, 11 y sección 4.5: ejercicios 1, 3, 5, 7.

\paragraph{Pregunta de investigación:}  
\begin{enumerate}[leftmargin=*]
    \item ¿Qué ocurre si se añaden vectores a una base? ¿Puede seguir siendo base?
    \item ¿Cómo varía el espacio generado si cambiamos un vector por otro dentro del conjunto?
    \item ¿Qué aplicaciones tiene el concepto de base en sistemas físicos o computacionales?
\end{enumerate}

\paragraph{Para la próxima clase:}  
Visualizar el video \href{https://youtu.be/kYB8IZa5AuE?si=gbqAeTULCuLFypRi}{Transformaciones lineales y matrices} y leer la sección 6.1 del libro \href{https://catalogobiblioteca.puce.edu.ec/cgi-bin/koha/opac-detail.pl?biblionumber=86081&query_desc=kw%2Cwrdl%3A%20algebra%20larson}{Fundamentos de Álgebra Lineal} de R. Larson.

\end{document} 