\documentclass[a4,11pt]{aleph-notas}
% Se puede ver la documentación aquí: 
% https://github.com/alephsub0/LaTeX_aleph-notas

% -- Paquetes adicionales 
\usepackage{enumitem}
\usepackage{url}
\hypersetup{
    urlcolor=blue,
    linkcolor=blue,
}

% -- Datos 
\institucion{Facultad de Ciencias Exactas, Naturales y Ambientale}
\carrera{Catálogo STEM}
\asignatura{Álgebra Lineal y Geometría Analítica}
\tema{Clase 07: Aplicaciones Lineales}
\autor{Andrés Merino}
\fecha{Periodo 2025-1}

\logouno[0.14\textwidth]{Logos/logoPUCE_04_ac}
\definecolor{colortext}{HTML}{0030A1}
\definecolor{colordef}{HTML}{0030A1}
\fuente{montserrat}


% -- Comandos para tablas

\begin{document}

\encabezado

%%%%%%%%%%%%%%%%%%%%%%%%%%%%%%%%%%%%%%%%
\section*{Resultado de Aprendizaje}
%%%%%%%%%%%%%%%%%%%%%%%%%%%%%%%%%%%%%%%%

%%%%%%%%%%%%%%%%%%%%%%%%%%%%%%%%%%%%%%%%
\subsection*{RdA de la asignatura:}
\begin{itemize}[leftmargin=*]
    \item \textbf{RdA 1:} Comprender los conceptos básicos del Álgebra Lineal y Geometría Analítica en el campo de la Ingeniería.
    \item \textbf{RdA 2:} Analizar los problemas relacionados al Álgebra Lineal y Geometría Analítica en el campo de la Ingeniería.
\end{itemize}

%%%%%%%%%%%%%%%%%%%%%%%%%%%%%%%%%%%%%%%%
\subsection*{RdA de la actividad:}
\begin{itemize}[leftmargin=*]
    \item Identificar si una función es o no una transformación lineal.
    \item Analizar el efecto geométrico de una transformación lineal.
    \item Determinar el núcleo (kernel) e imagen de una transformación lineal.
\end{itemize}

%%%%%%%%%%%%%%%%%%%%%%%%%%%%%%%%%%%%%%%%
\section*{Introducción}
%%%%%%%%%%%%%%%%%%%%%%%%%%%%%%%%%%%%%%%%

%%%%%%%%%%%%%%%%%%%%%%%%%%%%%%%%%%%%%%%%
\paragraph{Pregunta inicial:}  
¿Sabías que muchas técnicas de predicción en estadística y aprendizaje automático se basan en funciones entre espacios vectoriales? 

%%%%%%%%%%%%%%%%%%%%%%%%%%%%%%%%%%%%%%%%
\section*{Desarrollo}
%%%%%%%%%%%%%%%%%%%%%%%%%%%%%%%%%%%%%%%%

%%%%%%%%%%%%%%%%%%%%%%%%%%%%%%%%%%%%%%%%
\subsection*{Actividad 1: ¿Qué es una transformación lineal?}

\paragraph{¿Cómo lo haremos?}  
\begin{itemize}[leftmargin=*]
    \item \textbf{Lectura previa:} Se discuten preguntas sobre la lectura asignada (sección 6.1 del libro de Larson y la sección de imágenes compartida). Preguntas:
    \begin{itemize}
        \item ¿Qué se entiende por imagen y preimagen de un vector bajo una función?
        \item ¿Cuál es la diferencia entre codominio y rango?
        \item ¿Qué condiciones debe cumplir una función para ser una transformación lineal?
    \end{itemize}
    \item \textbf{Clase magistral:} Se explican conceptos clave de aplicaciones lineales, imagen y preimagen. Se utilizará el \href{https://fcena-puce.github.io/AlgLinealyGeomAnalitica-05-N0068/2-Resumenes/Resumen07.pdf}{Resumen07.pdf}.
    \item \textbf{Visualización:} Se explora la app de GeoGebra \href{https://www.geogebra.org/m/keG7vsZS}{Transformaciones lineales} para observar los efectos geométricos.
    \item \textbf{Resolución de ejercicios:} Se trabajan ejercicios del estilo de \href{https://alephsub0.org/pregrado/ejercicios-de-algebra-lineal/david-escobar/determinar-si-una-funcion-es-aplicacion-lineal-2/}{Ejercicio 1} y \href{https://alephsub0.org/pregrado/ejercicios-de-algebra-lineal/juan-andrade/determinar-si-una-funcion-es-aplicacion-lineal/}{Ejercicio 2}.
    \item \textbf{Recomendaciones de lectura:} \href{https://blog.nekomath.com/lineal-i-transformaciones-lineales/}{Álgebra Lineal I: Transformaciones lineales}
    \item \textbf{Visualización de video:} 
    \begin{itemize}
        \item \href{https://youtu.be/rHLEWRxRGiM?si=NLl2s2yq4yYzMbE3}{Transformaciones lineales tridimensionales}.
        \item \href{https://youtu.be/v8VSDg_WQlA?si=PRo-lCD-QB51Ha1H}{Matrices no cuadradas como transformaciones}.
    \end{itemize}
\end{itemize}

\paragraph{Verificación de aprendizaje:}  
\begin{itemize}
    \item ¿La función \( T(x, y) = (2x, y+1) \) es una transformación lineal?
    \item ¿Qué figura resulta de aplicar una transformación lineal a un cubo?
    \item ¿Cuál es la diferencia entre preimagen y núcleo?
\end{itemize}

%%%%%%%%%%%%%%%%%%%%%%%%%%%%%%%%%%%%%%%%
\subsection*{Actividad 2: Núcleo e imagen de una transformación, Matriz asociada}
\paragraph{¿Cómo lo haremos?}
\begin{itemize}[leftmargin=*]
    \item \textbf{Clase magistral:} Se explica el cálculo del núcleo e imagen de una transformación lineal, además de Matriz asociada. Se utilizará el \href{https://fcena-puce.github.io/AlgLinealyGeomAnalitica-05-N0068/2-Resumenes/Resumen07.pdf}{Resumen07.pdf}.
    \item \textbf{Resolución de ejercicios:} Práctica de ejercicios similares a los disponibles en \href{https://fcena-puce.github.io/AlgLinealyGeomAnalitica-05-N0068/2-Ejercicios/Ejercicios07.pdf}{Ejercicios07.pdf}.
    \item \textbf{Lectura sugerida:} \href{https://blog.nekomath.com/lineal-i-transformaciones-lineales/}{Transformaciones lineales}.
\end{itemize}

\paragraph{Verificación de aprendizaje:}  
\begin{itemize}
    \item ¿La función \( T(x, y) = (2x, y+1) \) es una transformación lineal?
    \item ¿Qué figura resulta de aplicar una transformación lineal a un cubo?
    \item ¿Cuál es la diferencia entre preimagen y núcleo?
    \item \textbf{Kahoot interactivo:} Realizar el siguiente quiz en línea para consolidar conceptos fundamentales sobre aplicaciones lineales: \href{https://create.kahoot.it/share/aplicaciones-lineales/54f82b8b-de9b-4622-a288-e809584d4186}{Kahoot: Aplicaciones Lineales}.
\end{itemize}


%%%%%%%%%%%%%%%%%%%%%%%%%%%%%%%%%%%%%%%%
\section*{Cierre}
%%%%%%%%%%%%%%%%%%%%%%%%%%%%%%%%%%%%%%%%

\paragraph{Tarea:}  
Resolver del libro \href{https://catalogobiblioteca.puce.edu.ec/cgi-bin/koha/opac-detail.pl?biblionumber=86081&query_desc=kw%2Cwrdl%3A%20algebra%20larson}{Fundamentos de Álgebra Lineal} de R. Larson:
\begin{itemize}
    \item Sección 6.1: 1, 3, 5, 7, 9, 11, 17, 25, 29, 35, 69
    \item Sección 6.2: 1, 3, 5, 11, 15, 54
    \item Sección 6.3: 1, 3, 5, 7, 9
\end{itemize}


\paragraph{Pregunta de investigación:}
\begin{enumerate}[leftmargin=*]
    \item ¿Qué utilidad tiene una transformación lineal en el análisis de regresión múltiple?
    \item ¿Cómo se interpreta geométricamente el núcleo de una transformación?
    \item ¿Qué sucede con las áreas y volúmenes al aplicar una transformación lineal?
\end{enumerate}

\paragraph{Para la próxima clase:}  
Visualizar el video \href{https://youtu.be/PFDu9oVAE-g?si=wuTb6PTOu0xJCW-F}{Autovectores (Vectores propios) y Autovalores}. Realizar la actividad de clase invertida: \href{https://fcena-puce.github.io/AlgLinealyGeomAnalitica-05-N0068/2-ClaseInvertida/03Est-ValoresPropios.pdf}{03Est-ValoresPropios.pdf}.


\end{document} 