\documentclass[a4,11pt]{aleph-notas}
% Se puede ver la documentación aquí: 
% https://github.com/alephsub0/LaTeX_aleph-notas

% -- Paquetes adicionales 
\usepackage{enumitem}
\usepackage{url}
\usepackage{aleph-comandos}
\hypersetup{
    urlcolor=blue,
    linkcolor=blue,
}

% -- Datos 
\institucion{Facultad de Ciencias Exactas, Naturales y Ambientale}
\carrera{Catálogo STEM}
\asignatura{Álgebra Lineal y Geometría Analítica}
\tema{Clase 04: El espacio $\R^n$}
\autor{Andrés Merino}
\fecha{Semestre 2025-1}

\logouno[0.14\textwidth]{Logos/logoPUCE_04_ac}
\definecolor{colortext}{HTML}{0030A1}
\definecolor{colordef}{HTML}{0030A1}
\fuente{montserrat}


% -- Comandos para tablas

\begin{document}

\encabezado


%%%%%%%%%%%%%%%%%%%%%%%%%%%%%%%%%%%%%%%%
\section*{Resultado de Aprendizaje}
%%%%%%%%%%%%%%%%%%%%%%%%%%%%%%%%%%%%%%%%

%%%%%%%%%%%%%%%%%%%%%%%%%%%%%%%%%%%%%%%%
\subsection*{RdA de la asignatura:}
\begin{itemize}[leftmargin=*]
    \item \textbf{RdA 1:} Comprender los conceptos básicos del Álgebra Lineal y Geometría Analítica en el campo de la Ingeniería.
    \item \textbf{RdA 2:} Analizar los problemas relacionados al Álgebra Lineal y Geometría Analítica en el campo de la Ingeniería.
    % \item \textbf{RdA 3:} Aplicar distintos tópicos del Álgebra Lineal y la Geometría Analítica en el campo de la Ingeniería.
\end{itemize}

%%%%%%%%%%%%%%%%%%%%%%%%%%%%%%%%%%%%%%%%  
\subsection*{RdA de la actividad:}  
\begin{itemize}[leftmargin=*]
    \item Interpretar geométricamente vectores en \(\mathbb{R}^2\) y \(\mathbb{R}^3\), así como las ecuaciones de rectas y planos en el espacio.
    \item Representar rectas y planos a partir de puntos y vectores de dirección en distintas formas.
    \item Aplicar herramientas computacionales para visualizar objetos en \(\mathbb{R}^3\).
\end{itemize}

%%%%%%%%%%%%%%%%%%%%%%%%%%%%%%%%%%%%%%%%  
\section*{Introducción}  
%%%%%%%%%%%%%%%%%%%%%%%%%%%%%%%%%%%%%%%%  

%%%%%%%%%%%%%%%%%%%%%%%%%%%%%%%%%%%%%%%%  
\paragraph{Pregunta inicial:}  
¿Sabías que vivimos en un espacio de tres dimensiones? ¿Puedes imaginar cómo serían cuatro, cinco o incluso diez dimensiones? ¿Qué formas adoptarían las rectas y planos en esos espacios?

%%%%%%%%%%%%%%%%%%%%%%%%%%%%%%%%%%%%%%%%  
\section*{Desarrollo}  
%%%%%%%%%%%%%%%%%%%%%%%%%%%%%%%%%%%%%%%%  

%%%%%%%%%%%%%%%%%%%%%%%%%%%%%%%%%%%%%%%%  
\subsection*{Actividad 1: Rectas y planos en el espacio}

En esta clase se abordarán los conceptos fundamentales del espacio \(\mathbb{R}^n\), haciendo énfasis en las representaciones geométricas de rectas y planos mediante vectores. Se utilizará una clase magistral combinada con herramientas visuales y ejercicios de aplicación práctica.

\paragraph{¿Cómo lo haremos?}  
\begin{itemize}[leftmargin=*]
    \item \textbf{Discusión inicial:} se formularán preguntas sobre el video asignado la clase anterior para conectar ideas sobre visualización en dimensiones superiores.
    \item \textbf{Clase magistral:} se explicarán los conceptos del conjunto \(\mathbb{R}^n\), y se desarrollarán las formas paramétricas y cartesianas de rectas y planos. Se usa el resumen \href{https://fcena-puce.github.io/AlgLinealyGeomAnalitica-05-N0068/2-Resumenes/Resumen04.pdf}{Resumen04.pdf}.
    \item \textbf{Resolución de ejercicios:} los estudiantes resolverán ejercicios como determinar la ecuación de una recta que pasa por dos puntos o la de un plano que contiene un punto y dos vectores de dirección.
    \item \textbf{Visualización con GeoGebra:} se utilizarán los siguientes tutoriales para representar vectores, rectas y planos:  
    \href{https://www.youtube.com/watch?v=GJYpYoUI104}{Vectores 3D en Geogebra} y
    \href{https://www.youtube.com/shorts/HRGo8-waS9E}{Cómo graficar un punto y un vector en R3  en Geogebra}  
    \item \textbf{Realidad aumentada:} los estudiantes explorarán las visualizaciones 3D en sus dispositivos móviles usando la herramienta de realidad aumentada de GeoGebra.
    \item \textbf{Visualización de video:} se recomendará el estudio con el video: \href{https://youtu.be/wiuEEkP_XuM?si=7Fk0XK8RBzKr8Qtg}{Vectores, ¿qué son?}
\end{itemize}

\paragraph{Verificación de aprendizaje:}  
\begin{itemize}[leftmargin=*]
    \item ¿Cuál es la forma paramétrica de la recta que pasa por los puntos \((1,2,3)\) y \((4,5,6)\)?
    \item ¿Qué condiciones deben cumplir tres puntos para determinar un plano?
    \item ¿Cómo se representa un plano que pasa por el punto \((1,0,2)\) y contiene a los vectores \({v}_1 = (1,1,0)\), \({v}_2 = (0,1,1)\)?
\end{itemize}

%%%%%%%%%%%%%%%%%%%%%%%%%%%%%%%%%%%%%%%%  
\section*{Cierre}  
%%%%%%%%%%%%%%%%%%%%%%%%%%%%%%%%%%%%%%%%  

\paragraph{Tarea:}  
Resolver los ejercicios del libro \href{https://puce.odilo.us/info/precalculo-matematicas-para-el-calculo-03111193}{Precálculo: matemáticas para el cálculo de James Stewart}:  

\begin{itemize}[leftmargin=*]
    \item Sección 9.1: 3, 5, 7, 67, 73
    \item Sección 9.2: 25, 27
    \item Sección 9.3: 3, 5
    \item Sección 9.4: 3, 5, 33, 35
    \item Sección 9.6: 3, 5, 9, 11, 21, 23
\end{itemize}

\paragraph{Pregunta de investigación:}  
\begin{enumerate}[leftmargin=*]
    \item ¿Qué es y qué representa geométricamente el producto vectorial de dos vectores en el espacio?
    \item ¿Podemos extender la idea de planos y rectas al espacio \(\mathbb{R}^4\)? ¿Cómo se representarían?
\end{enumerate}

\paragraph{Para la próxima clase:}  
Leer las páginas 191 a 194 del libro \href{https://catalogobiblioteca.puce.edu.ec/cgi-bin/koha/opac-detail.pl?biblionumber=86083}{Álgebra lineal y sus aplicaciones de David C. Lay}, donde se introduce la noción espacio vectorial.


\end{document} 