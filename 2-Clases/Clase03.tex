\documentclass[a4,11pt]{aleph-notas}
% Se puede ver la documentación aquí: 
% https://github.com/alephsub0/LaTeX_aleph-notas

% -- Paquetes adicionales 
\usepackage{enumitem}
\usepackage{url}
\hypersetup{
    urlcolor=blue,
    linkcolor=blue,
}

% -- Datos 
\institucion{Facultad de Ciencias Exactas, Naturales y Ambientale}
\carrera{Catálogo STEM}
\asignatura{Álgebra Lineal y Geometría Analítica}
\tema{Clase 03: Determinantes y Matriz inversa}
\autor{Andrés Merino}
\fecha{Semestre 2025-1}

\logouno[0.14\textwidth]{Logos/logoPUCE_04_ac}
\definecolor{colortext}{HTML}{0030A1}
\definecolor{colordef}{HTML}{0030A1}
\fuente{montserrat}


% -- Comandos para tablas

\begin{document}

\encabezado


%%%%%%%%%%%%%%%%%%%%%%%%%%%%%%%%%%%%%%%%
\section*{Resultado de Aprendizaje}
%%%%%%%%%%%%%%%%%%%%%%%%%%%%%%%%%%%%%%%%

%%%%%%%%%%%%%%%%%%%%%%%%%%%%%%%%%%%%%%%%
\subsection*{RdA de la asignatura:}
\begin{itemize}[leftmargin=*]
    \item \textbf{RdA 1:} Comprender los conceptos básicos del Álgebra Lineal y Geometría Analítica en el campo de la Ingeniería.
    \item \textbf{RdA 2:} Analizar los problemas relacionados al Álgebra Lineal y Geometría Analítica en el campo de la Ingeniería.
    % \item \textbf{RdA 3:} Aplicar distintos tópicos del Álgebra Lineal y la Geometría Analítica en el campo de la Ingeniería.
\end{itemize}

%%%%%%%%%%%%%%%%%%%%%%%%%%%%%%%%%%%%%%%%
\subsection*{RdA de la actividad:}
\begin{itemize}[leftmargin=*]  
    \item Comprender el significado y cálculo del determinante de una matriz.  
    \item Aplicar propiedades del determinante para identificar matrices invertibles.  
    \item Analizar el concepto de matriz inversa a partir de operaciones matriciales.  
\end{itemize}  


%%%%%%%%%%%%%%%%%%%%%%%%%%%%%%%%%%%%%%%%  
\section*{Introducción}  
%%%%%%%%%%%%%%%%%%%%%%%%%%%%%%%%%%%%%%%%  

%%%%%%%%%%%%%%%%%%%%%%%%%%%%%%%%%%%%%%%%  
\paragraph{Pregunta inicial:}  
¿Qué representa el determinante de una matriz? ¿Tiene un significado geométrico o es solo una fórmula algebraica?  

%%%%%%%%%%%%%%%%%%%%%%%%%%%%%%%%%%%%%%%%  
\section*{Desarrollo}  
%%%%%%%%%%%%%%%%%%%%%%%%%%%%%%%%%%%%%%%%  

%%%%%%%%%%%%%%%%%%%%%%%%%%%%%%%%%%%%%%%%  
\subsection*{Actividad 1: ¿Qué representa el determinante?}  

La clase inicia retomando los conocimientos adquiridos en la clase invertida sobre el cálculo de determinantes. A través de una visión conjunta, se reflexiona sobre su significado. Luego, se presentan las propiedades clave mediante una clase magistral, se resuelven ejercicios guiados y se recomiendan recursos audiovisuales y herramientas digitales para fortalecer el aprendizaje.

\paragraph{¿Cómo lo haremos?}  
\begin{itemize}[leftmargin=*]  
    \item \textbf{Clase invertida – Visión conjunta:} los estudiantes, tras haber estudiado previamente el cálculo de determinantes, reflexionan en grupo sobre qué representa el determinante de una matriz. Se analizan ejemplos concretos (2×2, 3×3, 4×4) y se discute su interpretación geométrica.  
    \item \textbf{Clase magistral:} se explican las propiedades del determinante. Se usa el resumen \href{https://fcena-puce.github.io/AlgLinealyGeomAnalitica-05-N0068/2-Resumenes/Resumen03.pdf}{Resumen03.pdf}.  
    \item \textbf{Resolución de ejercicios:} se guiará a los estudiantes en la solución de ejercicios del documento \href{https://fcena-puce.github.io/AlgLinealyGeomAnalitica-05-N0068/2-Ejercicios/Ejercicios03.pdf}{Ejercicios03.pdf}, aplicando las propiedades vistas.  
    \item \textbf{Visualización de video:} se recomendará el estudio con el video: \href{https://www.youtube.com/watch?v=Bd7XboIfsU4}{¿Qué son LOS DETERMINANTES? ¿Son amigos o enemigos?}.  
    \item \textbf{Implementación computacional:} se explica el uso de la herramienta \href{https://matrixcalc.org/es/slu.html}{Matrix Calculator} para el cálculo automático de determinantes y verificación de ejercicios complejos.  
\end{itemize}  

\paragraph{Verificación de aprendizaje:}  
\begin{enumerate}[leftmargin=*]  
    \item ¿Qué propiedades tiene el determinante?
    \item ¿Cuándo el determinante de una matriz es cero? ¿Qué implica eso?
    \item ¿Cómo cambia el determinante si permutamos dos filas?
\end{enumerate}  


%%%%%%%%%%%%%%%%%%%%%%%%%%%%%%%%%%%%%%%%  
\subsection*{Actividad 2: Explorando la matriz inversa}  

La actividad inicia con una exploración guiada que permite a los estudiantes descubrir la idea de matriz inversa mediante la multiplicación de pares de matrices. A través de esta dinámica, se identifican patrones y se construye la noción de inversa. Luego, mediante una clase magistral, se formaliza el concepto, se introduce el Teorema 15 como herramienta de cálculo, y se practica su uso con ejercicios y herramientas computacionales.

\paragraph{¿Cómo lo haremos?}  
\begin{itemize}[leftmargin=*]  
    \item \textbf{Exploración guiada:} los estudiantes trabajan con dos matrices 2×2 que son inversas entre sí. Se les solicita multiplicarlas en ambos órdenes y reflexionar sobre el resultado. Luego, se plantea el reto de analizar la matriz $\begin{bmatrix}1 & 1\\1 & 1\end{bmatrix}$ y discutir si es posible hallar su inversa.  
    \item \textbf{Clase magistral:} se define formalmente el concepto de matriz inversa, sus propiedades, y su relación con el determinante. Se presenta el Teorema 15 para el cálculo de matrices inversas, apoyándose en el \href{https://fcena-puce.github.io/AlgLinealyGeomAnalitica-05-N0068/2-Resumenes/Resumen03.pdf}{Resumen03.pdf}.  
    \item \textbf{Resolución de ejercicios:} se desarrollan ejercicios del documento \href{https://fcena-puce.github.io/AlgLinealyGeomAnalitica-05-N0068/2-Ejercicios/Ejercicios03.pdf}{Ejercicios03.pdf}, enfocándose en el uso del Teorema 15.  
    \item \textbf{Implementación computacional:} se introduce el uso de \href{https://matrixcalc.org/es/slu.html}{Matrix Calculator} para encontrar la inversa de matrices de forma automática y verificar resultados.  
\end{itemize}  

\paragraph{Verificación de aprendizaje:}  
\begin{enumerate}[leftmargin=*]  
    \item ¿Qué condiciones debe cumplir una matriz para que tenga inversa?  
    \item ¿Cuál es la relación entre la inversa y el determinante?  
    \item ¿Qué sucede si se multiplica una matriz por su inversa en distinto orden?  
\end{enumerate}  


%%%%%%%%%%%%%%%%%%%%%%%%%%%%%%%%%%%%%%%%  
\section*{Cierre}  
%%%%%%%%%%%%%%%%%%%%%%%%%%%%%%%%%%%%%%%%  

\paragraph{Tarea:}  
Resolver del libro \href{https://catalogobiblioteca.puce.edu.ec/cgi-bin/koha/opac-detail.pl?biblionumber=86083}{Álgebra lineal y sus aplicaciones de David C. Lay}, sección 2.2: ejercicios 1, 3, 5, 17, 19; y sección 2.3: ejercicios 1, 3, 7, 15.  
Además, se debe completar y enviar el \href{https://fcena-puce.github.io/AlgLinealyGeomAnalitica-05-N0068/1-Retos/Reto01.pdf}{Reto 1: Factorización $LU$}.  

\paragraph{Pregunta de investigación:}  
\begin{enumerate}[leftmargin=*]  
    \item ¿Cuál es la interpretación geométrica del determinante en 2 y 3 dimensiones?
    \item ¿Puede una matriz no cuadrada tener inversa? ¿Por qué?
    \item ¿Existen métodos alternativos al uso del determinante para verificar si una matriz es invertible?
\end{enumerate}  

\paragraph{Para la próxima clase:}  
Visualizar el video \href{https://youtu.be/eXA4806YuqY?si=8EsnZGPUa_kTr2nw}{VECTORES: ¿Flechas o Espacios Vectoriales?... ¿o ambos?} y reflexionar sobre su relación con el concepto de espacio vectorial y transformación lineal.

\end{document} 

