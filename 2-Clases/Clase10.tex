\documentclass[a4,11pt]{aleph-notas}
% Se puede ver la documentación aquí: 
% https://github.com/alephsub0/LaTeX_aleph-notas

% -- Paquetes adicionales 
\usepackage{enumitem}
\usepackage{url}
\usepackage{amsmath,amssymb}
\hypersetup{
    urlcolor=blue,
    linkcolor=blue,
}

% -- Datos 
\institucion{Facultad de Ciencias Exactas, Naturales y Ambientale}
\carrera{Catálogo STEM}
\asignatura{Álgebra Lineal y Geometría Analítica}
\tema{Clase 09: Cónicas}
\autor{Andrés Merino}
\fecha{Periodo 2025-1}

\logouno[0.14\textwidth]{Logos/logoPUCE_04_ac}
\definecolor{colortext}{HTML}{0030A1}
\definecolor{colordef}{HTML}{0030A1}
\fuente{montserrat}


% -- Comandos para tablas

\begin{document}

\encabezado

%%%%%%%%%%%%%%%%%%%%%%%%%%%%%%%%%%%%%%%%
\section*{Resultado de Aprendizaje}
%%%%%%%%%%%%%%%%%%%%%%%%%%%%%%%%%%%%%%%%

%%%%%%%%%%%%%%%%%%%%%%%%%%%%%%%%%%%%%%%%
\subsection*{RdA de la asignatura:}
\begin{itemize}[leftmargin=*]
    \item \textbf{RdA 1:} Comprender los conceptos básicos del Álgebra Lineal y Geometría Analítica en el campo de la Ingeniería.
    \item \textbf{RdA 2:} Analizar los problemas relacionados al Álgebra Lineal y Geometría Analítica en el campo de la Ingeniería.
    \item \textbf{RdA 3:} Aplicar distintos tópicos del Álgebra Lineal y la Geometría Analítica en el campo de la Ingeniería.
\end{itemize}

%%%%%%%%%%%%%%%%%%%%%%%%%%%%%%%%%%%%%%%%
\subsection*{RdA de la actividad:}
\begin{itemize}[leftmargin=*]
    \item Comprender la noción de producto interno y sus propiedades fundamentales en espacios vectoriales.
    \item Aplicar el concepto de ortogonalidad y norma para resolver problemas de distancia y proyecciones en diferentes contextos.
    \item Implementar el proceso de Gram-Schmidt para obtener bases ortogonales a partir de bases dadas.
\end{itemize}

%%%%%%%%%%%%%%%%%%%%%%%%%%%%%%%%%%%%%%%%
\section*{Introducción}
%%%%%%%%%%%%%%%%%%%%%%%%%%%%%%%%%%%%%%%%

%%%%%%%%%%%%%%%%%%%%%%%%%%%%%%%%%%%%%%%%
\paragraph{Pregunta inicial:} 
¿Sabías que, así como existen vectores ortogonales entre sí, también existen funciones ortogonales o incluso matrices ortogonales entre sí? ¿Qué utilidad crees que tiene esto en la práctica?

%%%%%%%%%%%%%%%%%%%%%%%%%%%%%%%%%%%%%%%%
\section*{Desarrollo}
%%%%%%%%%%%%%%%%%%%%%%%%%%%%%%%%%%%%%%%%

%%%%%%%%%%%%%%%%%%%%%%%%%%%%%%%%%%%%%%%%
\subsection*{Actividad 1: Espacios con producto interno}

\paragraph{¿Cómo lo haremos?}  
\begin{itemize}[leftmargin=*]
    \item \textbf{Clase magistral:} Se introducen los espacios con producto interno y se abordan conceptos como norma, distancia, ortogonalidad y proyección ortogonal, utilizando el \href{https://fcena-puce.github.io/AlgLinealyGeomAnalitica-05-N0068/2-Resumenes/Resumen10.pdf}{Resumen10.pdf}.
    \item \textbf{Aplicaciones prácticas:} Se calcula la distancia entre funciones y entre matrices para ilustrar aplicaciones como comparación de imágenes o señales.
    \item \textbf{Visualización interactiva:} Se explora la proyección ortogonal de vectores usando la herramienta de GeoGebra \href{https://www.geogebra.org/m/b5c9x8ef}{Orthogonal projections of vectors}.
    \item \textbf{Clase magistral:} Se explica el proceso de Gram-Schmidt para construir bases ortogonales y ortonormales.
    \item \textbf{Videos complementarios:} Se recomienda la visualización de los siguientes videos como apoyo:
    \begin{itemize}
        \item \href{https://youtu.be/LyGKycYT2v0?si=AitPzM6HonO3Zltf}{Productos escalares y dualidad}
        \item \href{https://youtu.be/7oO6xXpaTLk?si=2qbLr_N9Z2RR2iE9}{Ortogonalización y ortonormalización}
    \end{itemize}
\end{itemize}

\paragraph{Verificación de aprendizaje:}  
\begin{itemize}[leftmargin=*]
    \item ¿Cómo se define el producto interno en un espacio vectorial? ¿Qué propiedades cumple?
    \item ¿Qué condiciones debe cumplir un conjunto de vectores para ser ortogonal?
    \item ¿En qué consiste el proceso de Gram-Schmidt y para qué se utiliza?
\end{itemize}

%%%%%%%%%%%%%%%%%%%%%%%%%%%%%%%%%%%%%%%%
\section*{Cierre}
%%%%%%%%%%%%%%%%%%%%%%%%%%%%%%%%%%%%%%%%

\paragraph{Tarea:}  
Resolver del libro \href{https://catalogobiblioteca.puce.edu.ec/cgi-bin/koha/opac-detail.pl?biblionumber=86083}{Álgebra lineal y sus aplicaciones de David C. Lay}, los siguientes ejercicios:
\begin{itemize}
    \item Sección 6.1: 1, 3, 5, 7, 13, 15, 17
    \item Sección 6.2: 11, 13
    \item Sección 6.3: 11, 3, 5
    \item Sección 6.4: 1, 3
\end{itemize}

\paragraph{Pregunta de investigación:}  
\begin{enumerate}[leftmargin=*]
    \item ¿Cómo se aplica la ortogonalidad en la compresión de imágenes o señales?
    \item ¿Qué significa que un conjunto de funciones sea ortogonal en un espacio de funciones?
\end{enumerate}


\end{document} 