\documentclass[a4,11pt]{aleph-notas}
% Se puede ver la documentación aquí: 
% https://github.com/alephsub0/LaTeX_aleph-notas

% -- Paquetes adicionales 
\usepackage{enumitem}
\usepackage{url}
\usepackage{aleph-comandos}
\hypersetup{
    urlcolor=blue,
    linkcolor=blue,
}

% -- Datos 
\institucion{Facultad de Ciencias Exactas, Naturales y Ambientale}
\carrera{Catálogo STEM}
\asignatura{Álgebra Lineal y Geometría Analítica}
\tema{Clase 05: Espacios Vectoriales y subespacios}
\autor{Andrés Merino}
\fecha{Periodo 2025-1}

\logouno[0.14\textwidth]{Logos/logoPUCE_04_ac}
\definecolor{colortext}{HTML}{0030A1}
\definecolor{colordef}{HTML}{0030A1}
\fuente{montserrat}


% -- Comandos para tablas

\begin{document}

\encabezado


%%%%%%%%%%%%%%%%%%%%%%%%%%%%%%%%%%%%%%%%
\section*{Resultado de Aprendizaje}
%%%%%%%%%%%%%%%%%%%%%%%%%%%%%%%%%%%%%%%%

%%%%%%%%%%%%%%%%%%%%%%%%%%%%%%%%%%%%%%%%
\subsection*{RdA de la asignatura:}
\begin{itemize}[leftmargin=*]
    \item \textbf{RdA 1:} Comprender los conceptos básicos del Álgebra Lineal y Geometría Analítica en el campo de la Ingeniería.
    \item \textbf{RdA 2:} Analizar los problemas relacionados al Álgebra Lineal y Geometría Analítica en el campo de la Ingeniería.
    % \item \textbf{RdA 3:} Aplicar distintos tópicos del Álgebra Lineal y la Geometría Analítica en el campo de la Ingeniería.
\end{itemize}



%%%%%%%%%%%%%%%%%%%%%%%%%%%%%%%%%%%%%%%%
\subsection*{RdA de la actividad:}
\begin{itemize}[leftmargin=*]
\item Reconocer cuándo un conjunto con operaciones dadas forma un espacio vectorial.
\item Verificar si un subconjunto dado es un subespacio vectorial.
\item Analizar ejemplos concretos y abstractos de espacios y subespacios vectoriales.
\end{itemize}

%%%%%%%%%%%%%%%%%%%%%%%%%%%%%%%%%%%%%%%%
\section*{Introducción}
%%%%%%%%%%%%%%%%%%%%%%%%%%%%%%%%%%%%%%%%

%%%%%%%%%%%%%%%%%%%%%%%%%%%%%%%%%%%%%%%%
\paragraph{Pregunta inicial:}
¿Qué otros tipos de objetos matemáticos sabes sumar? ¿Se cumplen las mismas propiedades que con los números reales? ¿Y con los vectores?

%%%%%%%%%%%%%%%%%%%%%%%%%%%%%%%%%%%%%%%%
\section*{Desarrollo}
%%%%%%%%%%%%%%%%%%%%%%%%%%%%%%%%%%%%%%%%

%%%%%%%%%%%%%%%%%%%%%%%%%%%%%%%%%%%%%%%%
\subsection*{Actividad 1: Explorando definiciones a través de ejemplos}

\paragraph{¿Cómo lo haremos?}
\begin{itemize}[leftmargin=*]
\item \textbf{Discusión guiada:} Se iniciará con preguntas sobre los ejemplos de la lectura asignada:
\begin{itemize}
\item ¿Qué condiciones cumplen las flechas en el espacio tridimensional para formar un espacio vectorial?
\item ¿Qué se suma en el espacio de secuencias infinitas? ¿Cómo se define la multiplicación escalar?
\item ¿Por qué el conjunto de polinomios de grado menor o igual a $n$ forma un espacio vectorial?
\item ¿Cuál es la diferencia entre un espacio vectorial y un subespacio vectorial?
\item ¿Por qué $\mathbb{R}^2$ no es subespacio de $\mathbb{R}^3$?
\end{itemize}
\item \textbf{Clase magistral:} Se presentarán las definiciones formales de espacio vectorial y subespacio vectorial, haciendo énfasis en la verificación de axiomas. Se utilizará el \href{https://fcena-puce.github.io/AlgLinealyGeomAnalitica-05-N0068/2-Resumenes/Resumen05.pdf}{Resumen05.pdf}.
\item \textbf{Resolución de ejercicios:} Ejercicios similares a los del documento \href{https://fcena-puce.github.io/AlgLinealyGeomAnalitica-05-N0068/2-Ejercicios/Ejercicios05.pdf}{Ejercicios05.pdf}.
\item \textbf{Visualización de videos:}
\begin{itemize}
\item \href{https://www.youtube.com/watch?v=fNk\_zzaMoSs\&list=PLZHQObOWTQDPD3MizzM2xVFitgF8hE\_ab}{Vectores - Capítulo 1 - 3Blue1Brown}
\item \href{https://www.youtube.com/watch?v=TgKwz5Ikpc8}{Espacios vectoriales abstractos - 3Blue1Brown}
\end{itemize}
\item \textbf{Lectura adicional recomendada:} \href{https://blog.nekomath.com/algebra-lineal-i-subespacios-vectoriales/}{Álgebra Lineal I: Subespacios vectoriales - El blog de Leo}
\end{itemize}

\paragraph{Verificación de aprendizaje:}
\begin{itemize}
\item ¿Por qué el conjunto de todas las funciones reales es un espacio vectorial?
\item ¿Cuál es la condición para que un subconjunto sea subespacio?
\item ¿Qué propiedad impide que $\mathbb{R}^2$ sea subespacio de $\mathbb{R}^3$?
\end{itemize}

%%%%%%%%%%%%%%%%%%%%%%%%%%%%%%%%%%%%%%%%
\section*{Cierre}
%%%%%%%%%%%%%%%%%%%%%%%%%%%%%%%%%%%%%%%%

\paragraph{Tarea:}
Resolver del libro \href{https://catalogobiblioteca.puce.edu.ec/cgi-bin/koha/opac-detail.pl?biblionumber=86083}{Álgebra lineal y sus aplicaciones de David C. Lay}, sección 4.1, los ejercicios: 1, 3, 5, 7, 9, 11, 23.

\paragraph{Pregunta de investigación:}
\begin{enumerate}[leftmargin=*]
\item ¿Qué relación existe entre subespacios vectoriales y soluciones de sistemas homogéneos?
\item ¿Cómo se representan gráficamente los subespacios vectoriales en $\mathbb{R}^2$ y $\mathbb{R}^3$?
\item ¿Qué aplicaciones prácticas tienen los espacios vectoriales en la ingeniería y la informática?
\end{enumerate}

\paragraph{Para la próxima clase:}
Visualizar el video \href{https://www.youtube.com/watch?v=k7RM-ot2NWY\&list=PLZHQObOWTQDPD3MizzM2xVFitgF8hE\_ab\&index=2}{Combinaciones lineales, espacio generado y vectores base}.
Realizar la actividad de clase invertida: \href{https://fcena-puce.github.io/AlgLinealyGeomAnalitica-05-N0068/2-ClaseInvertida/02Est-Independencia.pdf}{02Est-Independencia.pdf}.



\end{document} 