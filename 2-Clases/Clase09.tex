\documentclass[a4,11pt]{aleph-notas}
% Se puede ver la documentación aquí: 
% https://github.com/alephsub0/LaTeX_aleph-notas

% -- Paquetes adicionales 
\usepackage{enumitem}
\usepackage{url}
\usepackage{amsmath,amssymb}
\hypersetup{
    urlcolor=blue,
    linkcolor=blue,
}

% -- Datos 
\institucion{Facultad de Ciencias Exactas, Naturales y Ambientale}
\carrera{Catálogo STEM}
\asignatura{Álgebra Lineal y Geometría Analítica}
\tema{Clase 09: Cónicas}
\autor{Andrés Merino}
\fecha{Periodo 2025-1}

\logouno[0.14\textwidth]{Logos/logoPUCE_04_ac}
\definecolor{colortext}{HTML}{0030A1}
\definecolor{colordef}{HTML}{0030A1}
\fuente{montserrat}


% -- Comandos para tablas

\begin{document}

\encabezado

%%%%%%%%%%%%%%%%%%%%%%%%%%%%%%%%%%%%%%%%
\section*{Resultado de Aprendizaje}
%%%%%%%%%%%%%%%%%%%%%%%%%%%%%%%%%%%%%%%%

%%%%%%%%%%%%%%%%%%%%%%%%%%%%%%%%%%%%%%%%
\subsection*{RdA de la asignatura:}
\begin{itemize}[leftmargin=*]
    \item \textbf{RdA 1:} Comprender los conceptos básicos del Álgebra Lineal y Geometría Analítica en el campo de la Ingeniería.
    \item \textbf{RdA 2:} Analizar los problemas relacionados al Álgebra Lineal y Geometría Analítica en el campo de la Ingeniería.
\end{itemize}

%%%%%%%%%%%%%%%%%%%%%%%%%%%%%%%%%%%%%%%%
\subsection*{RdA de la actividad:}
\begin{itemize}[leftmargin=*]
    \item Comprender las definiciones geométricas de las cónicas como lugares geométricos.
    \item Identificar las propiedades principales de parábolas, elipses y sus representaciones en el plano.
    \item Interpretar visualmente las cónicas como cortes de un cono doble.
\end{itemize}

%%%%%%%%%%%%%%%%%%%%%%%%%%%%%%%%%%%%%%%%
\section*{Introducción}
%%%%%%%%%%%%%%%%%%%%%%%%%%%%%%%%%%%%%%%%

%%%%%%%%%%%%%%%%%%%%%%%%%%%%%%%%%%%%%%%%
\paragraph{Pregunta inicial:}  
¿Qué tienen en común la trayectoria de un cometa, el reflejo del sonido en un auditorio, y la órbita de un planeta?

%%%%%%%%%%%%%%%%%%%%%%%%%%%%%%%%%%%%%%%%
\section*{Desarrollo}
%%%%%%%%%%%%%%%%%%%%%%%%%%%%%%%%%%%%%%%%

%%%%%%%%%%%%%%%%%%%%%%%%%%%%%%%%%%%%%%%%
\subsection*{Actividad 1: Exploración del lugar geométrico y las secciones cónicas}

\paragraph{¿Cómo lo haremos?}  
\begin{itemize}[leftmargin=*]
    \item \textbf{Clase magistral:} Introducción al concepto de lugar geométrico mediante el problema: «Caracterizar todos los puntos de \(\mathbb{R}^2\) cuya distancia al punto (1,2) sea igual a 3».
    \item \textbf{Transformaciones de lugares geométricos:} Se discute cómo las traslaciones y rotaciones afectan a las curvas.
    \item \textbf{Motivación visual:} Se utilizan animaciones en GeoGebra para presentar las cónicas como cortes de un cono doble: \href{https://www.geogebra.org/m/pCg8NFVT}{Conic sections}.
    \item \textbf{Parábola:} Definición interactiva usando \href{https://www.geogebra.org/m/hyttzxdf}{Parábola}, y análisis de propiedades con \href{https://www.geogebra.org/m/u7psybt6}{Parabolic reflector}.
    \item \textbf{Elipse:} Definición mediante focos con \href{https://www.geogebra.org/m/sbtukqr8}{Elipse Definición}, y propiedades clave usando \href{https://www.geogebra.org/m/RB25jv2a}{Reflexión en la elipse}.
    \item \textbf{Resumen:} Se utilizará el \href{https://fcena-puce.github.io/AlgLinealyGeomAnalitica-05-N0068/2-Resumenes/Resumen09.pdf}{Resumen09.pdf}.
    \item \textbf{Recomendación de lectura:} Se propone el artículo \href{https://www.revistas.espol.edu.ec/index.php/matematica/article/view/538}{«La clasificación de la ecuación cuadrática de dos variables»} para una profundización teórica:\\
\end{itemize}

\paragraph{Verificación de aprendizaje:}  
\begin{itemize}[leftmargin=*]
    \item ¿Qué condiciones debe cumplir un punto para pertenecer a una parábola?
    \item ¿Cuál es la relación entre los focos y cualquier punto de una elipse?
    \item ¿Qué cambia al trasladar una cónica respecto al origen?
\end{itemize}

%%%%%%%%%%%%%%%%%%%%%%%%%%%%%%%%%%%%%%%%
\section*{Cierre}
%%%%%%%%%%%%%%%%%%%%%%%%%%%%%%%%%%%%%%%%

\paragraph{Tarea:}  
Resolver del libro \href{https://catalogobiblioteca.puce.edu.ec/cgi-bin/koha/opac-detail.pl?biblionumber=124311&query_desc=kw%2Cwrdl%3A%20precalculo}{Precálculo: matemáticas para el cálculo de Stewart}, sección 11.5, los ejercicios: 15, 17, 23, 25.

\paragraph{Pregunta de investigación:}  
\begin{enumerate}[leftmargin=*]
    \item ¿Qué ocurre con la ecuación de una cónica si se rota el sistema de coordenadas?
    \item ¿Cómo se puede clasificar una cónica con base en su ecuación general?
    \item ¿Por qué las órbitas planetarias reales se aproximan a una elipse?
\end{enumerate}

\paragraph{Para la próxima clase:}  
Leer el blog \href{https://blog.nekomath.com/algebra-lineal-i-producto-interior-y-desigualdad-de-cauchy-schwarz/}{Álgebra Lineal I: Producto interior y desigualdad de Cauchy-Schwarz} para preparar el estudio del producto escalar y sus aplicaciones.




\end{document} 