\documentclass[a4,11pt]{aleph-notas}
% Se puede ver la documentación aquí: 
% https://github.com/alephsub0/LaTeX_aleph-notas

% -- Paquetes adicionales 
\usepackage{enumitem}
\usepackage{aleph-comandos}
\usepackage{booktabs}
\usepackage{multicol}


% -- Datos 
\institucion{Facultad de Ciencias Exactas, Naturales y Ambientales}
\carrera{Catálogo STEM}
\asignatura{Álgebra Lineal y Geometría Analítica}
\tema{Resumen no. 8: Cónicas}
\autor{Andrés Merino}
\fecha{Periodo 2025-1}

\logouno[0.14\textwidth]{Logos/logoPUCE_04_ac}
\definecolor{colortext}{HTML}{0030A1}
\definecolor{colordef}{HTML}{0030A1}
\fuente{montserrat}


% -- Comandos adicionales
\setlist[enumerate]{label=\roman*.}


\begin{document}

\encabezado

%%%%%%%%%%%%%%%%%%%%%%%%%%%%%%%%%%%%%%
\section{Lugar geométrico}
%%%%%%%%%%%%%%%%%%%%%%%%%%%%%%%%%%%%%%

\begin{defi}[Lugar geométrico]
    Un lugar geométrico es un subconjunto de $\R^n$ de todos los puntos que cumplen una determinada condición. Por lo general, esta condición está dada por una ecuación, a esta se la llama, la ecuación del lugar geométrico.
\end{defi}


%%%%%%%%%%%%%%%%%%%%%%%%%%%%%%%%%%%%%%
\section{Transformaciones del lugar geométrico}
%%%%%%%%%%%%%%%%%%%%%%%%%%%%%%%%%%%%%%


\begin{defi}[Traslaciones en el eje $x$]
    Sea $h\in\R$. Dado un lugar geométrico, para realizar una traslación en el eje $x$ de $h$ unidades hacia la derecha se debe realizar la siguiente transformación en la ecuación del lugar geométrico:
    \[
        (x,y)\mapsto (x-h,y).
    \]
\end{defi}

\begin{advertencia}
    En la definición anterior:
    \begin{itemize}
        \item si $h$ es positivo, la traslación será hacia la derecha;
        \item si $h$ es negativo, la traslación será hacia la izquierda.
    \end{itemize}
\end{advertencia}

\begin{defi}[Traslaciones en el eje $y$]
    Sea $k\in\R$. Dado un lugar geométrico, para realizar una traslación en el eje $y$ de $k$ unidades hacia arriba se debe realizar la siguiente transformación en la ecuación del lugar geométrico:
    \[
        (x,y)\mapsto (x,y-k).
    \]
\end{defi}

\begin{advertencia}
    En la definición anterior:
    \begin{itemize}
        \item si $k$ es positivo, la traslación será hacia arriba;
        \item si $k$ es negativo, la traslación será hacia abajo.
    \end{itemize}
\end{advertencia}


\begin{defi}[Reflexiones respecto al eje $x$]
    Dado un lugar geométrico, para realizar una reflexión respecto al eje $x$ se debe realizar la siguiente transformación en la ecuación del lugar geométrico:
    \[
        (x,y)\mapsto (x,-y).
    \]
\end{defi}

\begin{defi}[Reflexiones respecto al eje $y$]
    Dado un lugar geométrico, para realizar una reflexión respecto al eje $y$ se debe realizar la siguiente transformación en la ecuación del lugar geométrico:
    \[
        (x,y)\mapsto (-x,y).
    \]
\end{defi}

\begin{defi}[Reflexiones respecto al origen]
    Dado un lugar geométrico, para realizar una reflexión respecto al origen se debe realizar la siguiente transformación en la ecuación del lugar geométrico:
    \[
        (x,y)\mapsto (-x,-y).
    \]
\end{defi}

\begin{defi}[Rotación en torno al origen]
    Sea \( \theta \in \mathbb{R} \) un ángulo en radianes. La rotación de un lugar geométrico en sentido horario alrededor del origen, mediante un ángulo \( \theta \), se expresa mediante la transformación:
    \[
        (x, y) \mapsto (x', y') = (x\cos\theta - y\sin\theta,\ x\sin\theta + y\cos\theta).
    \]
    Esta transformación puede escribirse como un producto matricial:
    \[
        \begin{pmatrix}
        x' \\ y'
        \end{pmatrix}
        =
        \begin{pmatrix}
        \cos(\theta) & -\sin(\theta) \\
        \sin(\theta) & \cos(\theta)
        \end{pmatrix}
        \begin{pmatrix}
        x \\ y
        \end{pmatrix}.
    \]
\end{defi}

\begin{advertencia}
    \begin{itemize}
        \item Si \( \theta > 0 \), la rotación es en sentido \textbf{horario}.
        \item Si \( \theta < 0 \), la rotación es en sentido \textbf{antihorario}.
    \end{itemize}
\end{advertencia}


%%%%%%%%%%%%%%%%%%%%%%%%%%%%%%%%%%%%%%
\section{La circunferencia}
%%%%%%%%%%%%%%%%%%%%%%%%%%%%%%%%%%%%%%

\begin{defi}[Circunferencia]
    Dados un punto $(h,k)\in\R^2$ y $r>0$, una circunferencia es el lugar geométrico de todos los puntos cuya distancia al punto $(h,k)$ es igual a $r$. El punto $(h,k)$ es llamado centro de la circunferencia y a $r$ se lo llama radio de la circunferencia.
\end{defi}

\begin{teo}
    La ecuación de una circunferencia puede ser de la forma
    \[
        (x-h)^2 + (y-k)^2 = r^2,
    \]
    con $h,k\in\R$ y $r>0$, donde
    \begin{itemize}
    \item 
        su centro es el punto $(h,k)$; y
    \item   
        su radio es $r$.
    \end{itemize}
\end{teo}


%%%%%%%%%%%%%%%%%%%%%%%%%%%%%%%%%%%%%%
\section{La parábola}
%%%%%%%%%%%%%%%%%%%%%%%%%%%%%%%%%%%%%%

\begin{defi}[Parábola]
    Dados un punto $(x_0,y_0)$ y una recta $\ell$, una parábola es el lugar geométrico de todos los puntos cuya distancia al punto $(x_0,y_0)$ es igual a la distancia a la recta~$\ell$. 
    
    El punto es llamado foco de la parábola y la recta es llamada directriz de la parábola.
    
    A la recta perpendicular a la directriz que pasa por el foco se la llama eje de la parábola y al punto en el que esta recta corta la parábola se la llama vértice de la parábola.
\end{defi}


\begin{teo}
    Si la ecuación de una parábola es de la forma 
    \[
        (x-h)^2=\pm4p(y-k),
    \]
    con $h,k\in\R$ y $p>0$, se dice que la parábola se abre  verticalmente y se tiene que
    \begin{multicols}{2}
    \begin{itemize}
        \item su vértice es $(h,k)$;
        \item su foco es $(h,k\pm p)$;
        \item su directriz es $y=k\mp p$; y
        \item su eje es $x=h$.
    \end{itemize}
    \end{multicols}
\end{teo}

\begin{teo}
    Si la ecuación de una parábola es de la forma 
    \[
        (y-k)^2=\pm 4p(x-h),
    \]
    con $h,k\in\R$ y $p>0$, se dice que la parábola se abre horizontalmente y se tiene que
    \begin{multicols}{2}
    \begin{itemize}
        \item su vértice es $(h,k)$;
        \item su foco es $(h\pm p,k)$;
        \item su directriz es $x=h \mp p$; y
        \item su eje es $y=k$.
    \end{itemize}
    \end{multicols}
\end{teo}

\begin{advertencia}
    En una parábola, la reflexión de cualquier recta perpendicular a la directriz, sobre la parábola, pasa por el foco.
\end{advertencia}


%%%%%%%%%%%%%%%%%%%%%%%%%%%%%%%%%%%%%%
\section{La elipse}
%%%%%%%%%%%%%%%%%%%%%%%%%%%%%%%%%%%%%%

\begin{defi}[Elipse]
    Dados dos puntos $(x_1,y_1)$ y $(x_2,y_2)$, diferentes, una elipse es el lugar geométrico de todos los puntos cuya suma de la distancia al punto $(x_1,y_1)$ y la distancia al punto $(x_2,y_2)$ es constante. 
    
    Los puntos $(x_1,y_1)$ y $(x_2,y_2)$ son llamados focos de la elipse. El punto medio entre estos es llamado centro de la elipse.
    
    La recta que pasa por los focos es llamada eje mayor de la elipse y los puntos en los cuales el eje mayor corta la elipse son llamados vértices de la elipse; la mitad de la distancia entre estos puntos es llamado radio mayor.
    
    La recta que perpendicular a eje mayor que pasa por el centro de la elipse es llamada eje menor y la mitad de la distancia entre los puntos en los que corta a la elipse el llamado radio meyor.
\end{defi}


\begin{teo}
    Si la ecuación de una elipse es de la forma 
    \[
        \dfrac{(x-h)^2}{a^2}+\dfrac{(y-k)^2}{b^2}=1,
    \]
    con $h,k\in\R$ y $a,b>0$ tales que $a>b$, se dice que es una elipse horizontal y se tiene que
    \vspace*{3mm}
    \begin{multicols}{2}
    \begin{itemize}
        \item su centro es $(h,k)$;
        \item sus vértices son $(h\pm a,k)$;
        \item su radio mayor es $a$; 
        \item su radio menor es $b$; y
        \item sus focos son $(h\pm c,k)$,\\ donde $a^2=b^2+c^2$.
    \end{itemize}
    \end{multicols}
\end{teo}


\begin{teo}
    Si la ecuación de una elipse es de la forma 
    \[
        \dfrac{(x-h)^2}{b^2}+\dfrac{(y-k)^2}{a^2}=1,
    \]
    con $h,k\in\R$ y $a,b>0$ tales que $a>b$, se dice que es una elipse vertical y se tiene que
    % \vspace*{-5mm}
    \begin{multicols}{2}
    \begin{itemize}
        \item su centro es $(h,k)$;
        \item sus vértices son $(h,k\pm a)$; 
        \item su radio mayor es $a$; 
        \item su radio menor es $b$; y
        \item sus focos son $(h,k\pm c)$,\\ donde $a^2=b^2+c^2$.
    \end{itemize}
    \end{multicols}
\end{teo}


\begin{advertencia}
    En una elipse, la reflexión de cualquier recta que pase por un foco, sobre la elipse, pasa por el otro foco.
\end{advertencia}

%%%%%%%%%%%%%%%%%%%%%%%%%%%%%%%%%%%%%%
\section{La hipérbola}
%%%%%%%%%%%%%%%%%%%%%%%%%%%%%%%%%%%%%%

\begin{defi}[Hipérbola]
    Dados dos puntos $(x_1,y_1)$ y $(x_2,y_2)$, diferentes, una hipérbola es el lugar geométrico de todos los puntos cuya diferencia de distancias al punto $(x_1,y_1)$ y al punto $(x_2,y_2)$ es constante.

    Los puntos $(x_1,y_1)$ y $(x_2,y_2)$ son llamados focos de la hipérbola. El punto medio entre estos es llamado centro de la hipérbola.

    La recta que pasa por los focos es llamada eje transversal y los puntos en los cuales esta recta corta la hipérbola son llamados vértices; la mitad de la distancia entre estos puntos es llamado radio real.

    La perpendicular al eje transversal que pasa por el centro se llama eje conjugado. A la hipérbola se le asocian dos asíntotas que pasan por el centro y determinan su comportamiento asintótico.
\end{defi}

\begin{teo}
    Si la ecuación de una hipérbola es de la forma 
    \[
        \dfrac{(x-h)^2}{a^2}-\dfrac{(y-k)^2}{b^2}=1,
    \]
    con $h,k\in\R$ y $a,b>0$, se dice que es una hipérbola horizontal y se tiene que
    \vspace*{3mm}
    \begin{multicols}{2}
    \begin{itemize}
        \item su centro es $(h,k)$;
        \item sus vértices son $(h\pm a,k)$;
        \item sus focos son $(h\pm c,k)$,\\ donde $c^2 = a^2 + b^2$.
    \end{itemize}
    \end{multicols}
\end{teo}

\begin{teo}
    Si la ecuación de una hipérbola es de la forma 
    \[
        \dfrac{(y-k)^2}{a^2}-\dfrac{(x-h)^2}{b^2}=1,
    \]
    con $h,k\in\R$ y $a,b>0$, se dice que es una hipérbola vertical y se tiene que
    \vspace*{3mm}
    \begin{multicols}{2}
    \begin{itemize}
        \item su centro es $(h,k)$;
        \item sus vértices son $(h,k\pm a)$;
        \item sus focos son $(h,k\pm c)$,\\ donde $c^2 = a^2 + b^2$.
    \end{itemize}
    \end{multicols}
\end{teo}

%%%%%%%%%%%%%%%%%%%%%%%%%%%%%%%%%%%%%%
\section{Ecuación general de una cónica}
%%%%%%%%%%%%%%%%%%%%%%%%%%%%%%%%%%%%%%

\begin{defi}[Ecuación general]
    La ecuación general de una cónica en el plano es de la forma
    \[
        Ax^2 + Bxy + Cy^2 + Dx + Ey + F = 0,
    \]
    donde \( A, B, C, D, E, F \in \mathbb{R} \) y no todos \( A, B, C \) son cero. Esta expresión representa, dependiendo de los coeficientes, una parábola, una elipse, una hipérbola o una cónica degenerada.
\end{defi}

\begin{defi}[Forma matricial]
    La ecuación general de una cónica puede escribirse de forma matricial como
    \[
        \big(
        x \ \ y \ \ 1
        \big)
        \begin{pmatrix}
        A & B/2 & D/2 \\
        B/2 & C & E/2 \\
        D/2 & E/2 & F
        \end{pmatrix}
        \begin{pmatrix}
        x \\ y \\ 1
        \end{pmatrix}
         = 0.
    \]
    Esta expresión permite estudiar la cónica desde el álgebra lineal, analizando su parte cuadrática mediante la matriz simétrica
    \[
        Q = \begin{pmatrix}
        A & B/2 \\
        B/2 & C
        \end{pmatrix}.
    \]
\end{defi}

\begin{teo}[Clasificación mediante valores propios]
    Sea \( Q \) la matriz simétrica asociada a la parte cuadrática de la cónica. Si \( \lambda_1, \lambda_2 \) son los valores propios de \( Q \), entonces:
    \begin{itemize}
        \item Si \( \lambda_1 \cdot \lambda_2 > 0 \), la cónica es una \textbf{elipse} (si, además \( \lambda_1 = \lambda_2 \), es un \textbf{círculo}).
        \item Si \( \lambda_1 \cdot \lambda_2 < 0 \), la cónica es una \textbf{hipérbola}.
        \item Si uno de los valores propios es cero y el otro no, la cónica es una \textbf{parábola}.
        \item Si ambos son cero, se trata de una \textbf{cónica degenerada} (por ejemplo, un par de rectas coincidentes o vacía).
    \end{itemize}
\end{teo}

\begin{advertencia}
    Los vectores propios de \( Q \) indican la dirección de los ejes principales de la cónica. Por tanto, mediante un cambio de coordenadas ortogonal que diagonalice \( Q \), es posible eliminar el término mixto \( Bxy \).
\end{advertencia}




\end{document}