\documentclass[a4,11pt]{aleph-notas}
% Se puede ver la documentación aquí: 
% https://github.com/alephsub0/LaTeX_aleph-notas

% -- Paquetes adicionales  
\usepackage{enumitem}
\usepackage{aleph-comandos}
\usepackage{booktabs}


% -- Datos 
\institucion{Facultad de Ciencias Exactas, Naturales y Ambientale}
\carrera{Catálogo STEM}
\asignatura{Álgebra Lineal y Geometría Analítica}
\tema{Resumen no. 2: Sistemas lineales}
\autor{Andrés Merino}
\fecha{Semestre 2025-1}

\logouno[0.14\textwidth]{Logos/logoPUCE_04_ac}
\definecolor{colortext}{HTML}{0030A1}
\definecolor{colordef}{HTML}{0030A1}
\fuente{montserrat}


% -- Comandos adicionales
\setlist[enumerate]{label=\roman*.}


\begin{document}

\encabezado

%%%%%%%%%%%%%%%%%%%%%%%%%%%%%%%%%%%%%%
\section{Soluciones de sistemas de ecuaciones lineales}
%%%%%%%%%%%%%%%%%%%%%%%%%%%%%%%%%%%%%%
\begin{defi}[Matriz aumentada]
    Dado un sistema de ecuaciones lineales de $m$ ecuaciones lineales en las cuales figuran $n$ incógnitas:
    \begin{align*}
        a_{11}x_1 + a_{12}x_2 + \cdots + a_{1n}x_n &= b_1,\\
        a_{21}x_1 + a_{22}x_2 + \cdots + a_{2n}x_n &= b_2,\\
            & \hspace{2mm} \vdots\\
        a_{m1}x_1 + a_{m2}x_2 + \cdots + a_{mn}x_n &= b_m.
    \end{align*}
    donde $a_{ij}\in\R$ y $b_i\in \R$ con $i\in I$ y $j\in J$. A la matriz
    \[
        A=\begin{pmatrix}
        a_{11} & a_{12} & \cdots & a_{1n}\\
        a_{21} & a_{22} & \cdots & a_{2n}\\
        \vdots & \vdots & \ddots & \vdots\\
        a_{m1} & a_{m2} & \cdots & a_{mn}
        \end{pmatrix}
    \]
    se la llama matriz de coeficientes del sistema y a 
    \[
        b=\begin{pmatrix}
        b_1 \\ b_2 \\ \vdots \\ b_m
        \end{pmatrix}
        \texty
        x=\begin{pmatrix}
        x_1 \\ x_2 \\ \vdots \\ x_n
        \end{pmatrix}
    \]
    se las llama columnas de constantes y de incógnitas, respectivamente.
\end{defi}

\begin{advertencia}
    Bajo estas definiciones, dado un sistema de ecuaciones lineales, se dice que
    \[
        Ax=b
    \]
    es la representación matricial del sistema de ecuaciones.
\end{advertencia}

\begin{defi}[Matriz ampliada]
    Sean $A\in\Mat[\R]{m}{n}$ y $B\in\Mat[\R]{m}{p}$, se define la matriz ampliada de $A$ y $B$ al elemento de $\Mat[\R]{m}{(m+p)}$ dado por:
    \[
        \begin{pmatrix}
        a_{11}&a_{12}&\cdots&a_{1n}&|&b_{11}&b_{12}&\cdots&b_{1p}\\
        a_{21}&a_{22}&\cdots&a_{2n}&|&b_{21}&b_{22}&\cdots&b_{2p}\\
        \vdots&\vdots&\ddots&\vdots&|&\vdots&\vdots&\ddots&\vdots\\
        a_{m1}&a_{m2}&\cdots&a_{mn}&|&b_{m1}&b_{m2}&\cdots&b_{mp}
        \end{pmatrix}
    \]
    y se la denota por $(A|B)$.
\end{defi}

\begin{defi}
    Dado el sistema de ecuaciones lineales en forma matricial
    \[
        Ax=b
    \]
    con $A\in \Mat[\R]{m}{n}$ y $b\in\R^m$, se dice que 
    \[
        (A|b)
    \]
    es la matriz ampliada asociada al sistema.
\end{defi}

\begin{defi}[Matriz escalonada reducida por filas]
    Se dice que una matriz está en forma escalonada reducida por filas cuando satisface las siguientes propiedades:
    \begin{enumerate}
    \item 
        Todas las filas que constan de ceros, si las hay, están en la parte inferior de la matriz. 
    \item 
        La primera entrada distinta de cero de la fila, al leer de izquierda a derecha, es un $1$. Esta entrada se denomina entrada principal o uno principal de su fila. 
    \item 
        Para cada fila que no consta sólo de ceros, el uno principal aparece a la derecha y abajo de cualquier uno principal en las filas que le preceden. 
    \item 
        Si una columna contiene un uno principal, el resto de las entradas de dicha columna son iguales a cero. 
    \end{enumerate}
    Se dice que una matriz está en forma escalonada por filas si satisface las primeras tres propiedades . 
\end{defi}

\begin{defi}[Operaciones elementales de fila]
    Dada una matriz $A\in\R^{m \times n}$, una operación elemental por filas sobre $A$ es una de las siguientes:
    \begin{itemize}
    \item   
        \textbf{Intercambio de filas:} dados $i\in I$ y $j\in J$, intercambiar la fila $i$ por la fila $j$, denotado por
        \[
            F_i \leftrightarrow F_j,
        \]
        es reemplazar la fila
        \[
            \begin{pmatrix}a_{i1}&a_{i2}&\ldots& a_{in}\end{pmatrix}
        \]
        por la fila
        \[
            \begin{pmatrix}a_{j1}&a_{j2}&\ldots& a_{jn}\end{pmatrix}
        \]
        y viceversa.
    \item 
        \textbf{Multiplicar una fila por un escalar:} dados $i\in I$ y $\alpha\in \R$, con $\alpha\neq 0$, multiplicar la fila $i$ por $\alpha$, denotado por
        \[
            \alpha F_i \rightarrow F_i,
        \]
        es reemplazar la fila
        \[
            \begin{pmatrix}a_{i1}&a_{i2}&\ldots& a_{in}\end{pmatrix}
        \]
        por 
        \[
            \begin{pmatrix}\alpha a_{i1}&\alpha a_{i2}&\ldots&\alpha a_{in}\end{pmatrix}.
        \]
    \item 
        \textbf{Sumar un múltiplo de una fila con otra:} dados $i,j\in I$ y $\alpha\in \R$, multiplicar la fila $i$ por $\alpha$ y sumarlo a la fila $j$, denotado por
        \[
            \alpha F_i + F_j\rightarrow F_j,
        \]
        es reemplazar la fila
        \[
            \begin{pmatrix}a_{j1}&a_{j2}&\ldots& a_{jn}\end{pmatrix}
        \]
        por 
        \[
            \begin{pmatrix}\alpha a_{i1} + a_{j1}&\alpha a_{i2} + a_{j2}&\ldots&\alpha a_{in} + a_{jn}\end{pmatrix}.
        \]
\end{itemize}
\end{defi}




\begin{defi}[Equivalente por filas]
    Sean $A,B\in \Mat[\R]{m}{n}$, se dice que la matriz $A$ es equivalente por filas a una matriz $B$, denotado por $A\sim B$, si $B$ se puede obtener al aplicar a la matriz $A$ una sucesión de operaciones elementales por fila.
\end{defi}

\begin{teo}
    Toda matriz $A\in\Mat[\R]{m}{n}$ es equivalente por filas a una única matriz en forma escalonada reducida por filas. 
\end{teo}

\begin{advertencia}
    El proceso para obtener una matriz escalonada reducida por filas a partir de una matriz cualquiera se conoce como eliminación de Gauss-Jordan.
\end{advertencia}

\begin{defi}
    Sean $A\in\Mat[\R]{m}{n}$ y $B\in\Mat[\R]{m}{n}$ la única matriz escalonada reducida por filas equivalente a $A$. El rango de $A$, denotado por $\rang(A)$, es el número de filas no nulas que tiene la matriz $B$.  
\end{defi}

\begin{prop}
    Sean $A, B\in\Mat[\R]{m}{n}$. Se tiene que si $A\sim B$, entonces 
    \[
        \rang(A) = \rang(B).
    \]
\end{prop}

\begin{prop}
    Sean $A\in\Mat[\R]{m}{n}$. Se tiene que si $A$ es una matriz escalonada, entonces $\rang(A)$ es el número de filas no nulas que tiene $A$.
\end{prop}

\subsection{Resolución de sistemas lineales}

\begin{teo}
    Sean $A,C\in \Mat[\R]{m}{n}$ y $b,d\in \R^m$, se tiene que los sistemas de ecuaciones lineales
    \[
        Ax=b
        \texty
        Cx=d
    \]
    tienen las mismas soluciones si y solo si
    \[
        (A|b)\sim (C|d),
    \]
    es decir, si las matrices aumentadas de los sistemas son equivalentes por filas. 
\end{teo}


\begin{defi}
    Sean $A\in \Mat[\R]{m}{n}$ y $b\in \R^m$, dado el sistema de ecuaciones lineales
    \[
        Ax=b,
    \]
    se dice que
    \begin{itemize}
    \item
        el sistema es \textbf{inconsistente} si no tiene solución;
    \item
        el sistema es \textbf{consistente} si tiene solución.
    \end{itemize}
\end{defi}

\begin{teo}
    Sean $A\in \Mat[\R]{m}{n}$ y $b\in \R^m$, dado el sistema de ecuaciones lineales
    \[
        Ax=b,
    \]
    se tiene una y solo una de las siguientes
    \begin{itemize}
    \item
        el sistema es inconsistente;
    \item
        el sistema es consistente y tiene una única solución; o
    \item
        el sistema es consistente y tiene infinitas soluciones.
    \end{itemize}
\end{teo}


\begin{teo}[Teorema de Rouché–Frobenius]
    Sean $A\in \Mat[\R]{m}{n}$ y $b\in \R^m$, dado el sistema de ecuaciones lineales
    \[
        Ax=b,
    \]
    se tiene que
    \begin{itemize}
    \item
        el sistema es consistente si y solo si $\rang(A)=\rang(A|b)$;
    \item
        en caso de que el sistema sea consistente, la solución es única si y solo si $\rang(A)=n$.
    \end{itemize}
\end{teo}

%%%%%%%%%%%%%%%%%%%%%%%%%%%%%%%%%%%%%%
\subsection{Sistemas homogéneos}

\begin{defi}[Sistema homogéneo]
    Sea $A\in \Mat[\R]{m}{n}$ al sistema
    \[ 
        Ax = 0
    \]
    se lo denomina sistema de ecuaciones lineales homogéneo.
\end{defi}

\begin{defi}
    Sea $A\in \Mat[\R]{m}{n}$, dado el sistema homogéneo
    \[ 
        Ax = 0,
    \]
    entonces
    \begin{itemize}
    \item 
        a $x = 0$ se la llama la solución trivial del sistema;
    \item 
        a $x \neq 0$ tal que $Ax=0$ se la llama una solución no trivial. 
    \end{itemize}
\end{defi}

\begin{teo}
    Un sistema homogéneo de $m$ ecuaciones con $n$ incógnitas siempre tiene una solución no trivial si $m<n$, es decir, si el número de incógnitas es mayor que el número de ecuaciones. 
\end{teo}


\end{document}