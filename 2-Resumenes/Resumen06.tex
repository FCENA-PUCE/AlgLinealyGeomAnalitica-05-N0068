\documentclass[a4,11pt]{aleph-notas}
% Se puede ver la documentación aquí: 
% https://github.com/alephsub0/LaTeX_aleph-notas

% -- Paquetes adicionales 
\usepackage{enumitem}
\usepackage{aleph-comandos}
\usepackage{booktabs}


% -- Datos 
\institucion{Facultad de Ciencias Exactas, Naturales y Ambientales}
\carrera{Catálogo STEM}
\asignatura{Álgebra Lineal y Geometría Analítica}
\tema{Resumen no. 6: Independencia Lineal y Conjunto Generador}
\autor{Andrés Merino}
\fecha{Periodo 2025-1}

\logouno[0.14\textwidth]{Logos/logoPUCE_04_ac}
\definecolor{colortext}{HTML}{0030A1}
\definecolor{colordef}{HTML}{0030A1}
\fuente{montserrat}


% -- Comandos adicionales
\setlist[enumerate]{label=\roman*.}


\begin{document}

\encabezado

%%%%%%%%%%%%%%%%%%%%%%%%%%%%%%%%%%%%%%
\section{Conjunto generador}
%%%%%%%%%%%%%%%%%%%%%%%%%%%%%%%%%%%%%%

\begin{defi}
    Sean $(E,+,\cdot,\R)$ un espacio vectorial y $S\subseteq E$. Al subespacio vectorial más pequeño que contiene a $S$ (es decir, la intersección de todos los subespacios que contienen a $S$) se lo denomina espacio generado por $S$ y se denota por $\gen(S)$.
\end{defi}

\begin{teo}
    Sean $(E,+,\cdot,\R)$ un espacio vectorial y $S\subseteq E$. Se tiene que
    \[
        \spn(S)=\gen(S).
    \]
\end{teo}


\begin{defi}
    Sean $(E,+,\cdot,\R)$ un espacio vectorial y $S=\{v_1,v_2,\ldots,v_k\}\subseteq E$. Se dice que $S$ genera el espacio vectorial $E$ si cada vector en $E$ es una combinación lineal de los elementos de $S$, es decir, si para todo $v\in E$, existen $\alpha_1, \alpha_2, \ldots, \alpha_k\in K$ tales que
    \[
        v = \alpha_1v_1+ \alpha_2v_2+ \cdots+ \alpha_k v_k.
    \]
\end{defi}


\begin{teo}
    Sean $(E,+,\cdot,\R)$ un espacio vectorial y $S=\{v_1,v_2,\ldots,v_k\}\subseteq E$. Se tiene que $S$ genera el espacio vectorial $E$ si y solo si
    \[
        E = \gen(S) = \spn(S).
    \]
\end{teo}

%%%%%%%%%%%%%%%%%%%%%%%%%%%%%%%%%%%%%%
\section{Independencia lineal}
%%%%%%%%%%%%%%%%%%%%%%%%%%%%%%%%%%%%%%

\begin{defi}
    Sean $(E,+,\cdot,\R)$ un espacio vectorial y $S=\{v_1,v_2,\ldots,v_k\}\subseteq E$. Se dice que $S$ es un conjunto linealmente dependiente si existen $\alpha_1, \alpha_2, \ldots, \alpha_k \in \R$, no todos iguales a cero, tales que:
    \[
        \alpha_1 v_1 + \alpha_2 v_2 + \cdots + \alpha_k v_k = 0
    \]
    en caso contrario, se dice que $S$ es un conjunto linealmente independiente. 
\end{defi}

\begin{advertencia}
    De esta definición, se tiene que $\{v_1,v_2,\ldots,v_k\}$ es linealmente independiente si y solo si
    \[
        \alpha_1 v_1 + \alpha_2 v_2 + \cdots + \alpha_k v_k = 0
    \]
    implica que
    \[
        \alpha_1 = \alpha_2 = \cdots = \alpha_k = 0.
    \]
\end{advertencia}

\begin{advertencia}
    Se puede extender esta definición para conjuntos infinitos diciendo que $S$ es linealmente independiente si todo subconjunto finito de $S$ es linealmente independiente.
\end{advertencia}

\begin{teo}
    Sean $(E,+,\cdot,\R)$ un espacio vectorial y $S\subseteq E$. Si $0\in S$, entonces $S$ es linealmente dependiente.
\end{teo}

\begin{teo}
    Sean $(E,+,\cdot,\R)$ un espacio vectorial y $S=\{v_1,v_2,\ldots,v_k\}\subseteq E$. Se tiene que $S$ es un conjunto linealmente dependiente si y sólo si alguno de los vectores $v_j\in S$ es una combinación lineal de otros elementos de $S$.
\end{teo}

%%%%%%%%%%%%%%%%%%%%%%%%%%%%%%%%%%%%%%
\section{Bases}
%%%%%%%%%%%%%%%%%%%%%%%%%%%%%%%%%%%%%%

\begin{defi}
    Sean $(E,+,\cdot,\R)$ un espacio vectorial y $B\subseteq E$. Se dice que $B$ es una base de $E$ si 
    \begin{itemize}
        \item $B$ genera a $E$ y 
        \item $B$ es linealmente independiente. 
    \end{itemize}
\end{defi}


\begin{teo}[Base canónica de $\R^n$]
    En $\R^n$, el conjunto $\{e^1,e^2,\ldots,e^n\}\subset\R^n$, definidos por 
    \[
        e^i_j=
        \begin{cases}
            0& \text{si }i\neq j,\\
            1& \text{si }i= j,
        \end{cases}
    \]
    para todo $i,j\in\{1,2,\ldots,n\}$, es una base de $\R^n$.
\end{teo}

\begin{teo}[Base canónica de {$\R_n[x]$}]
    En $\R_n[x]$, el conjunto $\{1,x,\ldots,x^{n-1},x^n\}$ es una base de $\R_n[x]$. A esta base se la denomina la base canónica de $\R_n[x]$.
\end{teo}

\begin{teo}
    Sean $(E,+,\cdot,\R)$ un espacio vectorial y $B\subseteq E$ una base de $E$. Se tiene que todo elemento de $E$ se puede escribir, de manera única, como combinación lineal de elementos de $B$.
\end{teo}

\begin{teo}
    Sean $(E,+,\cdot,\R)$ un espacio vectorial y $B\subseteq E$. Se tiene que si todo elemento de $E$ se puede escribir, de manera única, como combinación lineal de elementos de $B$, entonces $B$ es una base de $E$.
\end{teo}


\begin{teo}
    Sean $(E,+,\cdot,\R)$ un espacio vectorial y $S\subseteq E$ un conjunto que genera a $E$. Se tiene que algún subconjunto de $S$ es base de $E$.
\end{teo}


\begin{teo}
    Todo espacio vectorial tiene una base.
\end{teo}

%%%%%%%%%%%%%%%%%%%%%%%%%%%%%%%%%%%%%%
\section{Dimensión}
%%%%%%%%%%%%%%%%%%%%%%%%%%%%%%%%%%%%%%


\begin{teo}
    Sean $(E,+,\cdot,\R)$ un espacio vectorial y $B\subseteq E$ una base de $E$. Si $S\subseteq E$ es un conjunto linealmente independiente, entonces 
    \[
        |S|\leq |B|.
    \]
\end{teo}

\begin{teo}
    Sean $(E,+,\cdot,\R)$ un espacio vectorial y $B\subseteq E$ una base de $E$. Si $S\subseteq E$ es un conjunto que genera a $E$, entonces 
    \[
        |B|\leq |S|.
    \]
\end{teo}


\begin{teo}
    Sean $(E,+,\cdot,\R)$ un espacio vectorial y $B,T\subseteq E$ bases de $E$. Se tiene que
    \[
        |B|=|T|.
    \]
\end{teo}

\begin{defi}[Dimensión]
    Sean $(E,+,\cdot,\R)$ un espacio vectorial y $B\subseteq E$ una base de $E$. Se define la dimensión de $E$, denotado por $\dim(E)$, por la cantidad de elementos de $B$.
\end{defi}

\begin{teo}
    Se tiene que
    \begin{itemize}
    \item
        $\dim(\R^n)=n$, con $n\in \N^*$;
    \item
        $\dim(\Mat[\R]{m}{n})=mn$, con $m,n\in \N^*$;
    \item
        $\dim(\R_n[x])=n+1$, con $n\in \N$;
    \item
        $\dim(\{0\}) = 0$.
    \end{itemize}
\end{teo}


\begin{teo}
    Sean $(E,+,\cdot,\R)$ un espacio vectorial y $S\subseteq E$ un conjunto linealmente independiente. Se tiene que existe una base $B$ de $E$ que contiene a $S$.
\end{teo}


\begin{teo}
    Sean $(E,+,\cdot,\R)$ un espacio vectorial, $B\subseteq E$ y $n\in\N^*$ tal que $\dim(E)=n$ y $|B|=n$. Se tiene que
    \begin{itemize}
    \item
        si $B$ es linealmente independiente, entonces $B$ es una base de $E$.
    \item
        si $B$ genera a $E$, entonces $B$ es una base de $E$.
    \end{itemize}
\end{teo}



\end{document}