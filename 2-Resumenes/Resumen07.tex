\documentclass[a4,11pt]{aleph-notas}
% Se puede ver la documentación aquí: 
% https://github.com/alephsub0/LaTeX_aleph-notas

% -- Paquetes adicionales 
\usepackage{enumitem}
\usepackage{aleph-comandos}
\usepackage{booktabs}


% -- Datos 
\institucion{Facultad de Ciencias Exactas, Naturales y Ambientales}
\carrera{Catálogo STEM}
\asignatura{Álgebra Lineal y Geometría Analítica}
\tema{Resumen no. 7: Aplicaciones lineales}
\autor{Andrés Merino}
\fecha{Periodo 2025-1}

\logouno[0.14\textwidth]{Logos/logoPUCE_04_ac}
\definecolor{colortext}{HTML}{0030A1}
\definecolor{colordef}{HTML}{0030A1}
\fuente{montserrat}


% -- Comandos adicionales
\setlist[enumerate]{label=\roman*.}


\begin{document}

\encabezado


\section{Aplicaciones lineales}

\begin{defi}[Aplicación lineal]
    Sean $(E,+_1,\cdot_1,\R)$ y $(F,+_2,\cdot_2,\R)$ espacios vectoriales. A una función $\func{T}{E}{F}$ se la llama una aplicación lineal (transformación lineal) si satisface que para todo $\alpha \in \R$, y todo $u,v \in E$ se cumple
    \begin{enumerate}
    \item 
        $T(u +_1 v) = T(u) +_2 T(v)\quad$ y
    \item 
        $T(\alpha \cdot_1 v ) = \alpha \cdot_2 T(v)$.
    \end{enumerate}
\end{defi}

En adelante, consideraremos $(E,+_1,\cdot_1,\R)$ y $(F,+_2,\cdot_2,\R)$ espacios vectoriales. Notaremos por $\mathcal{L}(E,F)$ el espacio de las aplicaciones lineales de $E$ en $F$.

\begin{advertencia}
    En caso de que no exista riesgo de ambigüedad, dado $T \in \mathcal{L}(E,F)$, se denotará
    \begin{enumerate}
    \item 
        $T(u + v) = T(u) + T(v)\quad$ y
    \item 
        $T(\alpha v ) = \alpha T(v)$,
    \end{enumerate}
    para $\alpha \in \R$ y $u,v \in E$.
\end{advertencia}
    
\begin{teo}
     Sea $T \in \mathcal{L}(E,F)$. Para todo $u,v, v_1, v_2, \ldots, v_n \in E$ y $\alpha_1, \alpha_2, \ldots, \alpha_n \in \R$
     \begin{enumerate}
         \item $T(0_E) = 0_F$;
         \item $T(u-v) = T(u) - T(v)$: y
         \item $T\left(\displaystyle \sum_{k=1}^n \alpha_k v_k \right)= 
         \displaystyle \sum_{k=1}^n \alpha_k T\left(v_k \right)$.
     \end{enumerate}
\end{teo}

\subsection{Ejemplos de transformaciones lineales}
\begin{enumerate}
\item 
    Transformación nula, \[\funcion{T}{F}{F}{v}{0.}\]
\item 
    Transformación identidad, \[\funcion{T}{F}{F}{v}{v.}\]
\item 
    \[
        \funcion{T}{\R^3}{\R^2}
            {(x,y,z)}{(x, y).}
    \]
\item 
    \[
        \funcion{T}{\R^2}{\R^3}
            {(x,y)}{(x, y,0).}
    \]
\item 
    \[
        \funcion{T}{\R^3}{\R^3}
            {(x,y,z)}{(x+y, y + z,x).}
    \]
\item 
    \[
        \funcion{T}{\R^2}{\Mat{2}{2}}
            {(x,y)}{\begin{pmatrix}x& 0\\ 0 &y\end{pmatrix}.}
    \]
\item 
    \[
        \funcion{T}{\R^3}{\R_2[t]}
            {(x,y,z)}{x+(x-y)t + zt^2.}
    \]
\end{enumerate}



\begin{teo}
    Sea $T \in \mathcal{L}(E,F)$. Si $\{u_1,u_2,\ldots,u_n\}$ es una base de $E$, entonces $T$ está completamente determinada por
    \[
        \{T(u_1),T(u_2),\ldots,T(u_n)\}.
    \]
    Es decir, si se conoce el valor de $\{T(u_1),T(u_2),\ldots,T(u_n)\}$, entonces se conoce $T(u)$ para todo $u\in E$.
\end{teo}


%%%%%%%%%%%%%%%%%%%%%%%%%%%%%%%%%%%%%% 14
%% Núcleo e imagen
%%%%%%%%%%%%%%%%%%%%%%%%%%%%%%%%%%%%%%
\subsection{Núcleo e imagen}

\begin{defi}[Núcleo e imagen de una aplicación lineal]
    Sea $T \in \mathcal{L}(E,F)$.
    \begin{enumerate}
    \item 
        El núcleo de $T$, denotado por $\ker(T)$, está definida por:
        \[
            \ker(T) = \{v \in E: T(v) = 0\}.
        \]
    \item 
        La imagen de $T$, denotada por $\img(T)$, está definida por:
        \[
            \img(T) = \{ w \in F: w = T(v) \text{ para algún } v \in E\}.
        \]
    \end{enumerate}
\end{defi}

\begin{teo}
    Sea $T \in \mathcal{L}(E,F)$, entonces
    \begin{enumerate}
    \item 
        $\ker(T)$ es un subespacio vectorial de $E$.
    \item 
        $\img(T)$ es un subespacio vectorial de $F$.
    \end{enumerate}
\end{teo}
    
\begin{teo}
    Sea $T \in \mathcal{L}(E,F)$. Se tiene que $T$ es inyectiva si y solo si $\ker(T)=\{0\}$.
\end{teo}

\begin{defi}[Nulidad y rango de una aplicación lineal]
    Sea $T \in \mathcal{L}(E,F)$.
    \begin{enumerate}
    \item 
        Se llama nulidad de $T$ a $\dim(\ker(T))$.
    \item 
        Se llama rango de $T$ a $\dim(\img(T))$.
    \end{enumerate}
\end{defi}

\begin{teo}
    Sea $T \in \mathcal{L}(E,F)$ con $E$ un espacio de dimensión finita. Se tiene que
    \[
        \dim(E) = \dim(\ker(T)) +  \dim(\img(T)).
    \]
\end{teo}   

\begin{teo}
    Sea $T \in \mathcal{L}(E,F)$ con $E$ un espacio de dimensión finita. Si $\dim(E)=\dim(F)$, entonces se tiene que la siguientes son equivalentes
    \begin{itemize}
        \item $T$ es inyectiva,
        \item $T$ es sobreyectiva.
    \end{itemize}
\end{teo}   

%%%%%%%%%%%%%%%%%%%%%%%%%%%%%%%%%%%%%% 14
%% Aplicaciones lineales
%%%%%%%%%%%%%%%%%%%%%%%%%%%%%%%%%%%%%%
\subsection{Propiedades de aplicaciones lineales}

\begin{teo}
    Se tiene que $\mathcal{L}(E,F)$ es un espacio vectorial.
\end{teo}


\begin{teo}
    Sean $T_1,T_2 \in \mathcal{L}(E,F)$ y $B= \{v_1, v_2, \ldots, v_n\}$ una base para $E$. Si
    \[
        T_1(v_i) = T_2(v_i)
    \]
    para todo $i \in \{1, 2, \ldots, n\}$, entonces se tiene que $T_1(v)=T_2(v)$, para todo $v\in E$, es decir, 
    \[
        T_1 = T_2.
    \]
\end{teo}

\begin{teo}
    Sean $B= \{v_1, v_2, \ldots, v_n\}$ una base para $E$ y $w_1, w_2, \ldots, w_n\in F$. Se tiene que existe una única transformación lineal $T \in \mathcal{L}(E,F)$ tal que
    \[
        T(v_i) = w_i
    \]
    para todo $i\in \{1, 2, \ldots, n\}$.
\end{teo}


\begin{teo}
    Sea $T \in \mathcal{L}(E,F)$. Supongamos que $\dim(E) = n$ y $\dim(F) = m$, se tiene que:
    \begin{enumerate}
        \item si $n > m$, entonces $T$ no es inyectiva; y
        \item si $m > n$, entonces $T$ no es sobreyectiva.
    \end{enumerate}
\end{teo} 

%%%%%%%%%%%%%%%%%%%%%%%%%%%%%%%%%%%%%%
\subsection{Isomorfismos}
%%%%%%%%%%%%%%%%%%%%%%%%%%%%%%%%%%%%%%


\begin{defi}[Isomorfismo]
    Sea $T \in \mathcal{L}(E,F)$. Se dice que $T$ es un isomorfismo de $E$ en $F$ si $T$ es biyectiva.
\end{defi}

\begin{defi}[Espacios isomorfos]
    Sean $(E,+,\cdot,\R)$ y $(F,+,\cdot,\R)$. Se dice que $E$ y $F$ son isomorfos si existe un isomorfismo $T$ de $E$ en $F$, se lo denota por $E \cong F$.
\end{defi}

\begin{teo}
    Sea $T \in \mathcal{L}(E,F)$ un isomorfismo, se tiene que
    \begin{enumerate}
    \item 
        si $\{v_1, v_2, \ldots, v_n\}$ genera a $E$, entonces 
        \[
            \{T(v_1), T(v_2), \ldots, T(v_n)\}
        \]
        genera a $F$;
    \item 
        si $\{v_1, v_2, \ldots, v_n\}$ son linealmente independientes en $E$, entonces 
        \[
            \{T(v_1), T(v_2), \ldots, T(v_n)\}
        \] 
        es linealmente independientes en $F$;
    \item 
        si $\{v_1, v_2, \ldots, v_n\}$ es base de $E$, entonces
        \[
            \{T(v_1), T(v_2), \ldots, T(v_n)\}
        \]
        es base de $F$; 
    \item 
        si $E$ es de dimensión finita, entonces $F$ es de dimensión finita y 
        \[
            \dim(E) = \dim(F).
        \]
    \end{enumerate}
\end{teo} 

\begin{teo}
    Sean $(E,+,\cdot,\R)$ y $(F,+,\cdot,\R)$ espacios de dimensión finita tales que $\dim(E) = \dim(F)$, entonces $E \cong F$.
\end{teo}

\begin{teo}
    Se tiene que
    \begin{enumerate}
        \item $\Mat{m}{n} \cong \R^{mn}$, con $m,n\in\N^*$;
        \item $P_n[x] \cong \R^{n+1}$, con $n\in\N^*$.
    \end{enumerate}
\end{teo}


%%%%%%%%%%%%%%%%%%%%%%%%%%%%%%%%%%%%%% 14
%% Matriz asociada
%%%%%%%%%%%%%%%%%%%%%%%%%%%%%%%%%%%%%%
\subsection{Matriz asociada}

En adelante, consideraremos $E$ y $F$ espacios vectoriales de dimensión finita tales que $\dim(E)=n$ y $\dim(F)=m$.

\begin{defi}
    Sean $T \in \mathcal{L}(E,F)$ y $B$, $D$ bases para $E$ y $F$, respectivamente. Se tiene que existe una única matriz de $\Mat[\R]{m}{n}$, denotada por $[T]_{D,B}$, tal que
    \[
        [T(v)]_{D} = [T]_{D,B} [v]_{B}
    \]
    para todo $v\in E$. A esta matriz se la llama matriz asociada a la aplicación lineal $T$.
\end{defi}


\begin{teo}
    Sean $T \in \mathcal{L}(E,F)$ y 
    \[
        B = \{v_1,v_2,\ldots,v_n\}
        \texty
        D = \{u_1,u_2,\ldots,u_m\}
    \]
    bases para $E$ y $F$, respectivamente. Se tiene que las columnas de la matriz $[T]$ son los vectores de coordenadas de $T(v_j)$ en la base $D$, para $j\in\{1,2,\ldots,n\}$, es decir
    \[
        [T]_{D,B} = \begin{pmatrix}
            [T(v_1)]_D & [T(v_2)]_D & \cdots & [T(v_n)]_D
        \end{pmatrix}.
    \]
\end{teo}

\begin{advertencia}
    En caso de no existir riesgo de confusión en las bases que se utiliza, se nota simplemente $[T]$ a la matriz asociada a la aplicación lineal $T$. 
\end{advertencia}


Se pueden ver ejemplos del procedimiento para encontrar una matriz asociada a una aplicación lineal entre las páginas 522 y 527 del libro de Kolman.




\begin{teo}
    Sea $T \in \mathcal{L}(E,F)$. Se tiene que $T$ es biyectiva si y solo si $[T]$ es invertible.
\end{teo}



\begin{teo}
    Sean $T_1 \in \mathcal{L}(E,F)$ y $T_2\in\mathcal{L}(F,G)$, con $G$ un espacio vectorial de dimensión finita. Se tiene que
    \[
        [T_2\circ T_1] = [T_2][T_1].
    \]
\end{teo}


\begin{teo}
    Sea $T\in \mathcal{L}(E,F)$ una aplicación lineal invertible. Se tiene que
    \[
        [T^{-1}] = [T]^{-1}.
    \]
\end{teo}


\begin{teo}
    Se tiene que $\mathcal{L}(E,F)$ es isomorfo a $\Mat[\R]{m}{n}$, es decir
    \[
        \mathcal{L}(E,F)
        \cong
        \Mat[\R]{m}{n}.
    \]
\end{teo}



\end{document}