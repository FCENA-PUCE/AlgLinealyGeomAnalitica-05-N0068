\documentclass[a4,11pt]{aleph-notas}
% Se puede ver la documentación aquí: 
% https://github.com/alephsub0/LaTeX_aleph-notas

% -- Paquetes adicionales 
\usepackage{enumitem}
\usepackage{aleph-comandos}
\usepackage{booktabs}
\usepackage{multicol}


% -- Datos 
\institucion{Facultad de Ciencias Exactas, Naturales y Ambientales}
\carrera{Catálogo STEM}
\asignatura{Álgebra Lineal y Geometría Analítica}
\tema{Resumen no. 10: Producto interno}
\autor{Andrés Merino}
\fecha{Periodo 2025-1}

\logouno[0.14\textwidth]{Logos/logoPUCE_04_ac}
\definecolor{colortext}{HTML}{0030A1}
\definecolor{colordef}{HTML}{0030A1}
\fuente{montserrat}


% -- Comandos adicionales
\setlist[enumerate]{label=\roman*.}


\begin{document}

\encabezado


%%%%%%%%%%%%%%%%%%%%%%%%%%%%%%%%%%%%%%
%% Espacios con producto interno
%%%%%%%%%%%%%%%%%%%%%%%%%%%%%%%%%%%%%%
\section{Espacios con producto interno}

\begin{defi}[Producto interno]
    Sea $(E,+,\cdot,\R)$ un espacio vectorial. Un producto interno sobre $E$ es una función
    \[
        \funcion{\langle \cdot, \cdot \rangle}{E\times E}{\R}{(u,v)}{\langle u, v \rangle}
    \]
    tales que cumple:
    \begin{enumerate}
        \item 
            $\langle v, v \rangle \geq 0$ para todo $ v \in E$;
        \item 
            $\langle v,v \rangle = 0$ si y solo si $ v = 0$;
        \item 
            $\langle u + v, w \rangle = \langle u,w \rangle + \langle v,w\rangle $ para todo $u,v,w \in E$;
        \item
            $\langle \alpha v,w \rangle = \alpha \langle v,w \rangle$ para todo $v,w \in E$ y $\alpha\in\R$.
        \item
            $\langle v,w \rangle = \langle w, v \rangle$ para todo $v,w \in E$.
    \end{enumerate}
\end{defi}


\begin{advertencia}
    Otra notación para el producto interno es
    \[
        \langle u , v \rangle
        = (u\,|\,v).
    \]
\end{advertencia}


Si se define un producto interno sobre un espacio vectorial $E$, a este se lo denomina espacio con producto interno o pre-Hilbertiano. 

\begin{teo}
    Sea $(E,+,\cdot,\R)$ un espacio vectorial provisto de producto interno $\langle \cdot, \cdot \rangle$, entonces:
    \begin{enumerate}
        \item 
            Para todo $u,v,w \in E$ 
            \[\langle u, v+w\rangle = \langle u,v\rangle + \langle u, w \rangle.\]
        \item 
            Para todo $u,v \in E$ y $\alpha \in \R$
            \[\langle u, \alpha v\rangle =  {\alpha}\langle u,v\rangle.\]
        \item
            Para todo $u\in E$ 
            \[\langle u, 0 \rangle =  \langle 0, u \rangle  = 0.\]
    \end{enumerate}
\end{teo}

\subsection{Productos internos usuales}

\begin{enumerate}
    \item En $(\R^n,+, \cdot, \R)$, para $x,y\in \R^n$:
    \[
        \langle x, y \rangle = \sum_{k=1}^n x_k y_k.
    \]
    
    \item En $(\R_n[x],+, \cdot, \R)$, para $p(x),q(x)\in \R_n[x]$, si 
    \[  
        p(x) = a_0 + a_1x + \cdots + a_nx^n
        \texty
        q(x) = b_0 + b_1x + \cdots + b_nx^n
    \]
    entonces
    \[
        \langle p(x), q(x) \rangle = \sum_{k=0}^n a_k b_k.
    \]
    
    \item En $(\Mat[\R]{m}{n},+, \cdot, \R)$, para $A,B\in \Mat[\R]{m}{n}$:
    \[
        \langle A, B \rangle = \tr(AB^\intercal).
    \]
    
    \item En $(\mathcal{C}([a,b]),+, \cdot, \R)$, para $f,g\in \mathcal{C}([a,b])$:
    \[
        \langle f, g \rangle = \int_a^b f(x) g(x) dx.
    \]
    \end{enumerate}

\begin{defi}
Sea $(E,+,\cdot,\R)$ un espacio vectorial provisto de producto interno y suponga que $u, v \in E$.
Entonces:
\begin{enumerate}
    \item $u$ y $v$ son ortogonales si $\langle u, v \rangle = 0$.
    \item La norma de $u$, denotada por $\|u \|$, está dada por
    \[
        \| u\| = \sqrt{\langle u, u \rangle}
    \]
\end{enumerate}
\end{defi}

\begin{defi}
    Sea $(E,+,\cdot,\R)$ un espacio vectorial provisto de producto interno. La distancia en el espacio se define por
    \[
        \funcion{d}{E\times E}{\R}{(u,v)}{\| u - v\|.}
    \]
\end{defi}


\begin{teo}[Vectores ortogonales]
    Sea $(E,+,\cdot,\R)$ un espacio vectorial provisto de producto interno. Para  $u,v\in E$, se dice que son ortogonales si
    \[
        \langle u , v \rangle = 0.
    \]
\end{teo}


\begin{teo}[Teorema de Pitágoras]
    Sea $(E,+,\cdot,\R)$ un espacio vectorial provisto de producto interno.
    Si $u,v$ son vectores ortogonales de $E$, entonces
    \[
        \| u + v \|^2 = \|u\|^2 + \|v\|^2.
    \]
\end{teo}

\begin{teo}[Desigualdad de Cauchy-Schwartz]
    Sea $(E,+,\cdot,\R)$ un espacio vectorial provisto de producto interno.
    Para todo $u,v\in E$ se cumple que
    \[
        | \langle u , v \rangle | \leq \|u\|  \|v\|.
    \]
\end{teo}

\begin{teo}[Desigualdad triangular]
    Sea $(E,+,\cdot,\R)$ un espacio vectorial provisto de producto interno.
    Para todo $u,v\in E$ se cumple que
    \[
        \| u + v \| \leq \|u\| + \|v\|.
    \]
\end{teo}


\begin{defi}[Conjunto ortogonal]
    Sea $(E,+,\cdot,\R)$ un espacio vectorial provisto de producto interno y
    \[C=\{v_1, v_2, \ldots, v_n\}\subseteq E.\] Se dice que $C$ es un conjunto ortogonal en $E$ si
    \[
        \langle v_i, v_j \rangle = 0
    \]
    para todo $i \neq j$.
\end{defi}

\begin{defi}[Conjunto ortonormal]
    Sea $(E,+,\cdot,\R)$ un espacio vectorial provisto de producto interno y
    \[C = \{v_1, v_2, \ldots, v_n\} \subseteq E.\] Se dice que $C$ es un conjunto ortonormal en $E$ si es ortogonal y 
    \[
        \| v_k\| =  1
    \]
    para todo $k \in \{1, \ldots, n\}$.
\end{defi}


\begin{teo}
    Sea $(E,+,\cdot,\R)$ un espacio vectorial provisto de producto interno. Si $C \subseteq E$ es un conjunto ortogonal de vectores no nulos, entonces es linealmente independiente.
\end{teo}

%%%%%%%%%%%%%%%%%%%%%%%%%%%%%%%%%%%%%%
%% Bases ortogonales
%%%%%%%%%%%%%%%%%%%%%%%%%%%%%%%%%%%%%%

A partir de aquí, siempre consideraremos espacios vectoriales provistos con un producto interno.


\subsection{Bases ortogonales}

\begin{defi}[Base ortogonal (ortonormal)]
    En un espacio vectorial, una base ortogonal (ortonormal) es una base cuyos vectores forman un conjunto ortogonal (ortonormal). 
\end{defi}

\begin{teo}
    Sean $(E,+,\cdot,\R)$ un espacio vectorial y $\{u_1, u_2, \ldots, u_n\}$ una base ortogonal para $E$. Se tiene que, para $v \in E$,
    \[
        v = \alpha_1 u_1 + \alpha_2 u_2 +\cdots + \alpha_n u_n
    \]
    donde
    \[
        c_k = \dfrac{\langle v, u_k \rangle}{\langle u_k, u_k \rangle}
    \]
    para todo $k \in \{1, 2, \ldots, n\}$.
\end{teo}


\begin{teo}
    Sean $(E,+,\cdot,\R)$ un espacio vectorial y $\{u_1, u_2, \ldots, u_n\}$ una base ortonormal para $E$. Se tiene que, para $v \in E$,
    \[
        v = \alpha_1 u_1 + \alpha_2 u_2 +\cdots + \alpha_n u_n
    \]
    donde
    \[
        c_k = \langle v, u_k \rangle 
    \]
    para todo $k \in \{1, 2, \ldots, n\}$.
\end{teo}


\begin{teo}[Proceso de Gram-Schmidt]
    Sean $(E,+,\cdot,\R)$ un espacio vectorial y $\{u_1, u_2, \ldots, u_n\}$ un conjunto linealmente independiente de $E$. Definamos
    \begin{enumerate}
    \item 
        $v_1 = u_1$ y
    \item
        $\displaystyle v_k = 
            u_k 
            - \sum_{i=1}^{k-1} \dfrac{\langle u_k, v_i\rangle}{\langle v_i,v_i\rangle} v_i$, para $k=2,\ldots,n$.
    \end{enumerate}
    Se tiene que el conjunto
    \[
        \{v_1, v_2, \ldots, v_n\}
    \]
    es un conjunto ortogonal. Además, si definimos
    \[
        w_k = \dfrac{v_k}{\|v_k\|}
    \]
    para $k=1,\ldots,n$, se tiene que el conjunto
    \[
        \{w_1, w_2, \ldots, w_n\}
    \]
    es un conjunto ortonormal.
\end{teo}


\begin{teo}
    Todo espacio vectorial de dimensión finita tiene una base ortogonal y una base ortonormal.
\end{teo}


\begin{ejem}
    Considere la base $S = \{u_1, u_2, u_3\}$ de $\R^3$, donde 
    \[
        u_1 = (1,1,0), \qquad u_2 = (0, 1,1), \qquad u_3 = (1,0,1)
    \]
    mediante el proceso de Gram-Schmidt, se obtiene la base ortonormal $T=\{w_1, w_2, w_2\}$ para $\R^3$ donde
    \[
        w_1 = \left(\dfrac{1}{\sqrt{2}},\dfrac{1}{\sqrt{2}},0\right), \qquad
        w_2 = \left(-\dfrac{1}{\sqrt{6}},\dfrac{1}{\sqrt{6}},
        \dfrac{2}{\sqrt{6}}\right), \qquad
        w_3 = \left(\dfrac{1}{\sqrt{3}},-\dfrac{1}{\sqrt{3}},
        \dfrac{1}{\sqrt{3}}\right)
    \]
\end{ejem}

\subsection{Complemento ortogonal}

A partir de aquí, siempre consideraremos espacios vectoriales provistos con un producto interno.

\begin{defi}[Complemeto ortogonal]
    Sea $(E,+,\cdot,\R)$ un espacio vectorial y
    $H \subset E$. El complemento ortogonal de $H$, denotado por $H^\perp$, se define por:
    \[
        H^\perp = \{ x \in E: \langle x, h \rangle = 0, \text{ para todo } h \in H\}.
    \]
\end{defi}

\begin{teo}
    Sea $(E,+,\cdot,\R)$ un espacio vectorial y
    $H$ subespacio vectorial de $H$, entonces:
    \begin{enumerate}
        \item $H^\perp$ es un subespacio vectorial de $E$.
        \item $H \cap H^\perp = \{0\}$.
        \item Si $\dim(E) = n$, entonces $\dim(H^\perp) = n - \dim(H)$
    \end{enumerate}
\end{teo}


\begin{teo}
    Sea $(E,+,\cdot,\R)$ un espacio vectorial de dimensión finita, $W$ un subespacio vectorial de $E$, entonces 
    \[
    E = W \oplus W^\perp.
    \]
\end{teo}

\begin{teo}
    Sea $(E,+,\cdot,\R)$ un espacio vectorial de dimensión finita, $W$ un subespacio vectorial de $E$, entonces 
    \[
    \left(W^\perp\right)^\perp =  W.
    \]
\end{teo}

\begin{defi}[Proyección ortogonal]
    Sea $(E,+,\cdot,\R)$ un espacio vectorial y
    $H$ un subespacio vectorial de $E$, con base ortogonal $\{u_1, u_2, \ldots, u_n \}$. Para $v \in E$, la proyección ortogonal de $v$ sobre $H$, denotado por $\proy_H v$, se define por:
    \[
       \proy_H( v) = 
       \dfrac{\langle v,u_1 \rangle}{\langle u_1, u_1 \rangle}u_1 +
       \dfrac{\langle v,u_2 \rangle}{\langle u_2, u_2 \rangle}u_2 +
       \cdots +
       \dfrac{\langle v,u_n \rangle}{\langle u_n, u_n \rangle}u_n,
    \]
    donde $\proy_H (v) \in H$.
\end{defi}


\begin{teo}[Teorema de la proyección]
    Sea $(E,+,\cdot,\R)$ un espacio vectorial de dimensión finita, $H$ un subespacio vectorial de $E$ y $v \in E$. Se tiene que
    \[
        v = \proy_H( v) + \proy_{H^\perp}( v).
    \]
\end{teo}

\begin{teo}
    Sea $(E,+,\cdot,\R)$ un espacio vectorial de dimensión finita, $H$ un subespacio vectorial de $E$ y $v \in E$. Se tiene que el vector en $H$ más cercano a $v$ es $\proy_H(v)$, es decir,
    \[
    \|v-w\| \quad \text{ es mínima cuando } \quad w = \proy_W(v).
    \]
\end{teo}

\end{document}