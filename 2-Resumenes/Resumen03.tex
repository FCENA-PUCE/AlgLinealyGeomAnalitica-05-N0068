\documentclass[a4,11pt]{aleph-notas}
% Se puede ver la documentación aquí: 
% https://github.com/alephsub0/LaTeX_aleph-notas

% -- Paquetes adicionales 
\usepackage{enumitem}
\usepackage{aleph-comandos}
\usepackage{booktabs}


% -- Datos 
\institucion{Facultad de Ciencias Exactas, Naturales y Ambientale}
\carrera{Catálogo STEM}
\asignatura{Álgebra Lineal y Geometría Analítica}
\tema{Resumen no. 3: Determinantes y Matriz inversa}
\autor{Andrés Merino}
\fecha{Semestre 2025-1}

\logouno[0.14\textwidth]{Logos/logoPUCE_04_ac}
\definecolor{colortext}{HTML}{0030A1}
\definecolor{colordef}{HTML}{0030A1}
\fuente{montserrat}


% -- Comandos adicionales
\setlist[enumerate]{label=\roman*.}


\begin{document}

\encabezado

%%%%%%%%%%%%%%%%%%%%%%%%%%%%%%%%%%%%%%
\section{Determinantes}
%%%%%%%%%%%%%%%%%%%%%%%%%%%%%%%%%%%%%%

En esta sección tomaremos $n\in \N$, con $n>0$, e $I = \{ 1, 2, \ldots, n\}$.

\begin{defi}[Menor]
    Sean $A\in \Mat[\R]{n}{n}$ e $i,j\in I$. A la matriz de $\Mat[\R]{(n-1)}{(n-1)}$ que se obtiene eliminar la fila $i$ y la columna $j$ de $A$ se la llama el menor $ij$ de $A$, denotado por $A_{ij}$.
\end{defi}

\begin{advertencia}
    En la literatura se puede encontrar la notación de $M_{ij}$ para el menor de $ij$ de $A$.
\end{advertencia}

\begin{defi}[Menor principal]
    Sean $A\in \Mat[\R]{n}{n}$ y $k\in I$. A la matriz de $\Mat[\R]{k}{k}$ que se obtiene eliminar las $n-k$ últimas filas y columnas de $A$, se la llama el menor principal $k$ de $A$, denotado por $M_{k}$.
\end{defi}

\begin{advertencia}
    En la literatura se puede encontrar la notación de $A_{k}$ para el menor principal $k$ de $A$.
\end{advertencia}


\begin{defi}[Determinantes] 
    Sea $A\in \Mat[\R]{n}{n}$ se define el determinante de $A$, denotado por $\det(A)$ (o por $|A|$), de manera inductiva, como sigue:
    \begin{itemize}
    \item 
        Si $n=1$ y $A=(a_{11})$, entonces $\det(A) = a_{11}$.
    \item
        Si $n>1$, entonces
        \begin{align*}
            \det(A) 
            & = \sum_{k=1}^n a_{1k}(-1)^{1+k}\det(A_{1k})\\
            & = a_{11}\det(A_{11}) - a_{12}\det(A_{12}) 
            + \ldots +  (-1)^{1+n} a_{1n}\det(A_{1n}).
        \end{align*}
    \end{itemize}
\end{defi}

Ejemplos:
\begin{itemize}
\item 
    Sea $A$ una matriz de orden $2 \times 2$ de la forma
    \[ 
        A=
        \begin{pmatrix}
         a_{11} & a_{12}\\
         a_{21} & a_{22}
        \end{pmatrix},
    \]
    se tiene que
    \[
        A_{11} = (a_{22})
        \texty
        A_{12} = (a_{21}),
    \]
    por lo tanto
    \[
        \det(A_{11}) = a_{11}
        \texty
        \det(A_{12}) = a_{12},
    \]
    de esta forma,
    \[
        \det(A) = a_{11}\det(A_{11}) - a_{12}\det(A_{12})
            = a_{11}a_{22} - a_{12}a_{21},
    \]
    es decir,
    \[
        \det(A) = \begin{vmatrix}
        a_{22} & a_{23} \\
        a_{32} & a_{33}
        \end{vmatrix}
        = a_{11}a_{22} - a_{12}a_{21}.
    \]
\item Sea $A$ una matriz de orden $3 \times 3$ de la forma
         \[ A=
        \begin{pmatrix}
         a_{11} & a_{12} & a_{13}\\
         a_{21} & a_{22} & a_{23}\\
         a_{31} & a_{32} & a_{33}\\
        \end{pmatrix}
        \]
        \noindent el determinante de la matriz $A$ está dado por:
        \[
        \det(A) = a_{11}
        \begin{vmatrix}
        a_{22} & a_{23} \\
        a_{32} & a_{33}
        \end{vmatrix} - a_{12}
        \begin{vmatrix}
        a_{21} & a_{23} \\
        a_{31} & a_{33} \\
        \end{vmatrix}+ a_{13}
        \begin{vmatrix}
        a_{21} & a_{22} \\
        a_{31} & a_{32}
        \end{vmatrix}.
        \]
\end{itemize}

%%%%%%%%%%%%%%%%%%%%%%%%%%%%%%%%%%%%%%
\subsection{Propiedades de los determinantes}
%%%%%%%%%%%%%%%%%%%%%%%%%%%%%%%%%%%%%%

\begin{teo}
    Sea $A \in \Mat[\R]{n}{n}$. Si una fila o columna de $A$ contiene solo ceros, entonces $\det (A) =0$.
\end{teo}


\begin{teo}
    Sea $A \in \Mat[\R]{n}{n}$ una matriz triangular superior o triangular inferior, entonces
    \[
        \det(A) = a_{11}a_{22} \cdots a_{nn},
    \]
    es decir, el determinante de una matriz triangular es el producto de los elementos de su diagonal principal.
\end{teo}

\begin{teo}
    Sea $A \in \Mat[\R]{n}{n}$. El determinante de una matriz $A$ y de su transpuesta son iguales, es decir, 
    \[
        \det(A^\intercal) = \det(A).
    \]
\end{teo}

\begin{teo}
    Sean $A, B\in \Mat[\R]{n}{n}$, se tiene que:
    \[
        \det(AB) = \det(A)\det(B).
    \]
\end{teo}


\begin{teo}
    Sean $A, B \in \Mat[\R]{n}{n}$. Si la matriz $B$ se obtiene intercambiando dos filas o columnas de $A$ entonces
    \[  
        \det(B) = - \det(A).
    \]
\end{teo}

\begin{teo}
    Sea $A \in \Mat[\R]{n}{n}$. Si dos filas o columnas de $A$ son iguales, entonces 
    \[
        \det(A) = 0
    \]
\end{teo}

\begin{teo}
    Sean $A, B \in \Mat[\R]{n}{n}$. Si $B$ se obtiene al multiplicar una fila o columna de $A$ por un escalar $\alpha \in \R$, entonces
    \[
        \det(B) = \alpha \det(A).
    \]
\end{teo}


\begin{teo}
    Sean $A\in \Mat[\R]{n}{n}$ y $\alpha \in\R$. Se tiene que
    \[
        \det(\alpha A) = \alpha^n \det(A).
    \]
\end{teo}

\begin{teo}
    Sean $A, B \in \Mat[\R]{n}{n}$, $\alpha\in\R$ e $i,j\in I$, con $i\neq j$. Si $B$ se obtiene al aplicar una operación de fila $\alpha F_i + F_j \to F_j$, entonces 
    \[
        \det(B) = \det(A).
    \]
\end{teo}


%%%%%%%%%%%%%%%%%%%%%%%%%%%%%%%%%%%%%%
\subsection{Cofactores}
%%%%%%%%%%%%%%%%%%%%%%%%%%%%%%%%%%%%%%

\begin{defi}[Cofactores]
    Sean $A\in \Mat[\R]{n}{n}$ e $i,j\in I$. El cofactor $ij$ de $A$, denotado $C_{ij}$, está dado por
    \[
        C_{ij} = (-1)^{i+j} \det (A_{ij})
    \]
     donde $A_{ij}$ es el menor $ij$ de $A$.
\end{defi}

\begin{advertencia}
    En la literatura, también se suele llamar menor al determinante de $A_{ij}$ en lugar de a la matriz, como lo haremos en este texto. Además, al cofactor, se lo suele denotar por $A_{ij}$.
\end{advertencia}

\begin{teo}
    Sea $A \in \Mat[\R]{n}{n}$. Se tiene que para todo $i\in I$,
    \[
        \det(A) 
        = \sum_{k=1}^n a_{ik}C_{ik} 
        = a_{i1}C_{i1} + a_{i2}C_{i2} + \ldots + a_{in}C_{in}
    \]
    y
    \[
        \det(A) 
        = \sum_{k=1}^n a_{ki}C_{ki} 
        = a_{1i}C_{1i} + a_{2i}C_{2i} + \ldots + a_{ni}C_{ni}.
    \]
    El lado derecho de las igualdades toma el nombre de expansión por cofactores del determinante de $A$.
\end{teo}

%%%%%%%%%%%%%%%%%%%%%%%%%%%%%%%%%%%%%%
\section{Inversa de una matriz}
%%%%%%%%%%%%%%%%%%%%%%%%%%%%%%%%%%%%%%

\begin{defi}
    Sea $A\in \Mat[\R]{n}{n}$ es no singular o invertible si existe una matriz $B\in \Mat[\R]{n}{n}$ tal que 
    \[
        AB = BA = I_n.
    \]
    A la matriz $B$ se la denomina inversa de $A$ y se la denota por $A^{-1}$. Si no existe tal matriz, entonces se dice que $A$ es singular o no invertible. 
\end{defi}

\begin{teo}
    Si una matriz tiene inversa, la inversa es única. 
\end{teo}

\begin{teo}
    Sea $A\in \Mat[\R]{n}{n}$. 
    \begin{itemize}
    \item 
        Si $A$ es una matriz no singular, entonces $A^{-1}$ es no singular y 
        \[ 
            (A^{-1})^{-1} = A.
        \]
    \item 
        Si $A$ y $B$ son matrices no singulares, entonces $AB$ es no singular y 
        \[ 
            (AB)^{-1} = B^{-1}A^{-1}.
        \]
    \item 
        Si $A$ es una matriz no singular, entonces
        \[
            {(A^\intercal)}^{-1} = {(A^{-1})}^\intercal.
        \]
    \end{itemize}
\end{teo}

\begin{teo}
    Sean $p\in\N^*$ y $A_1, A_2, \ldots, A_p \in \Mat[\R]{n}{n}$ matrices no singulares. Se tiene que $A_1 A_2 \cdots A_p$ es no singular y 
    \[
        (A_1 A_2 \cdots A_p)^{-1} = A_p^{-1} A_{p-1}^{-1} \cdots A_1^{-1}.
    \]
\end{teo}

\begin{teo}
    Sean $A, B \in \Mat[\R]{n}{n}$. Se tiene que si $AB=I_n$, entonces $BA=I_n$.
\end{teo}

\begin{teo}
    Sea $A \in \Mat[\R]{n}{n}$, se tiene que $A$ es no singular si y solo si es equivalente por filas a $I_n$. Es más
    \[
        (A|I_n) \sim (I_n|A^{-1}).
    \]
\end{teo}

\begin{teo}
    Sea $A \in \Mat[\R]{n}{n}$, el sistema homogéneo 
    \[
        Ax = 0
    \]
    tiene una solución no trivial si y solo si $A$ es singular. 
\end{teo}

\begin{teo}
    Sea $A \in \Mat[\R]{n}{n}$. Se tiene que $A$ es no singular si y solo si el sistema lineal $Ax=b$ tiene una solución única para cada vector $b \in \R^{n}$.
\end{teo}

\begin{teo}
    Sea $A \in \Mat[\R]{n}{n}$, se tienen que las siguientes son equivalentes:
    \begin{enumerate}
    \item 
        $A$ es no singular;
    \item 
        el sistema $Ax=0$ solamente tiene la solución trivial;
    \item 
        $A$ es equivalente por filas a $I_n$;
    \item 
        $\rang(A)=n$; y 
    \item 
        el sistema lineal $Ax=b$ tiene una solución única para cada vector $b \in \R^{n}$. 
    \end{enumerate}
\end{teo}


\begin{defi}[Matriz de cofactores]
    Sea $A \in \Mat[\R]{n}{n}$. La matriz de cofactores de $A$, que se denota por $\cof(A)$, es la matriz de $\Mat[\R]{n}{n}$ que está formada por los cofactores de $A$, es decir, 
    \[
        \cof(A) = (C_{ij}) = 
        \begin{pmatrix}
          C_{11} & C_{12} & \cdots & C_{1n} \\
          C_{21} & C_{22} & \cdots & C_{2n} \\
          \vdots & \vdots & \ddots &\vdots  \\
          C_{n1} & C_{n2} & \cdots & C_{nn}
        \end{pmatrix}.
    \]
\end{defi}


\begin{defi}
    Sea $A \in \Mat[\R]{n}{n}$. La matriz adjunta de $A$, que se denota por $\adj(A)$, es la matriz de $\Mat[\R]{n}{n}$ que está formada por la transpuesta de la matriz de los cofactores de $A$, es decir, 
    \[
        \adj(A) = \cof(A)^\intercal.
    \]
\end{defi}

\begin{teo}
    Sea $A \in \Mat[\R]{n}{n}$, entonces
    \[
        A (\adj(A)) = (\adj(A))A = \det(A)I_n.
    \]
\end{teo}

\begin{cor}
    Sea $A \in \Mat[\R]{n}{n}$. Si $\det(A) \neq 0$, entonces $A$ es invertible y
    \[
        A^{-1} = \dfrac{1}{\det(A)} \adj(A).
    \]
\end{cor}

\begin{teo}
    Sea $A \in \Mat[\R]{n}{n}$. Una matriz $A$ es no singular si y sólo si $\det(A) \neq 0$.
\end{teo}


\begin{teo}
    Sea $A \in \Mat[\R]{n}{n}$. Si $A$ es no singular, entonces $\det(A) \neq 0$ y 
    \[
        \det(A^{-1}) = \dfrac{1}{\det(A)}.
    \]
\end{teo}

\begin{teo}
    Sea $A \in \Mat[\R]{n}{n}$. El sistema homogéneo $Ax = 0$ tiene una solución no trivial si y sólo si $\det(A) = 0$.
\end{teo}

\begin{teo}
    Sean $A \in \Mat[\R]{n}{n}$ y $b\in\R^n$. El sistema $Ax = b$ tiene una solución única si y sólo si $\det(A) \neq 0$ y su solución es
    \[
        A^{-1}b.
    \]
\end{teo}

\begin{teo}[Equivalencias no singulares]
    
    Sea $A \in \Mat[\R]{n}{n}$, se tienen que las siguientes son equivalentes:
    \begin{enumerate}
    \item 
        $A$ es no singular;
    \item 
        el sistema $Ax=0$ tiene solamente la solución trivial;
    \item 
        $A$ es equivalente por filas a $I_n$;
    \item 
        $\rang(A)=n$; 
    \item 
        el sistema lineal $Ax=b$ tiene una solución única para cada vector $b \in \R^{n}$; y
    \item 
        $\det(A) \neq 0$.
    \end{enumerate}
\end{teo}


\end{document}