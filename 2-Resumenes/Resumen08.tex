\documentclass[a4,11pt]{aleph-notas}
% Se puede ver la documentación aquí: 
% https://github.com/alephsub0/LaTeX_aleph-notas

% -- Paquetes adicionales 
\usepackage{enumitem}
\usepackage{aleph-comandos}
\usepackage{booktabs}


% -- Datos 
\institucion{Facultad de Ciencias Exactas, Naturales y Ambientales}
\carrera{Catálogo STEM}
\asignatura{Álgebra Lineal y Geometría Analítica}
\tema{Resumen no. 8: Valores y vectores propios}
\autor{Andrés Merino}
\fecha{Periodo 2025-1}

\logouno[0.14\textwidth]{Logos/logoPUCE_04_ac}
\definecolor{colortext}{HTML}{0030A1}
\definecolor{colordef}{HTML}{0030A1}
\fuente{montserrat}


% -- Comandos adicionales
\setlist[enumerate]{label=\roman*.}


\begin{document}

\encabezado

%%%%%%%%%%%%%%%%%%%%%%%%%%%%%%%%%%%%%% 14
%% Valores y vectores propios
%%%%%%%%%%%%%%%%%%%%%%%%%%%%%%%%%%%%%%
\section{Valores y vectores propios}

\begin{defi}
    Sea $A\in \Mat[\R]{n}{n}$, se tiene que $\lambda\in \R$ es un valor propio de $A$ si existe $v\in \R^n$, con $v\neq 0$, tal que
    \[
        Av = \lambda v,
    \]
    además, $v$ es un vector propio de $A$ asociado al valor propio $\lambda$.
\end{defi}


\begin{teo}
   Sean $A\in \Mat[\R]{n}{n}$ y $\lambda \in \R$. Se tiene que $\lambda$ es un valor propio de $A$ si y solo si
   \[
        \det(A - \lambda I) = 0,
   \]
   donde $I$ es la matriz identidad.
\end{teo}

\begin{teo}
    Los valores propios de una matriz triangular son los elementos de la diagonal de la matriz. 
\end{teo}


\begin{defi}
    Sea $A\in \Mat[\R]{n}{n}$. Al polinomio
    \[
        p_A(\lambda) = \det(A - \lambda I)
    \]
    se lo denomina polinomio característico de $A$.
\end{defi}


\begin{teo}[Teorema de Cayley–Hamilton]
    Sea $A\in \Mat[\R]{n}{n}$ y $p_A(\lambda)$ su polinomio característico. Se tiene que
    \[
        p_A(A) = 0.
    \]
\end{teo}


%%%%%%%%%%%%%%%%%%%%%%%%%%%%%%%%%%%%%% 14
%% Diagonalización de matrices
%%%%%%%%%%%%%%%%%%%%%%%%%%%%%%%%%%%%%%
\subsection{Diagonalización de Matrices}

\begin{defi}
    Sean $A,B\in \Mat[\R]{n}{n}$. Se dice que $A$ es semejante a $B$ si existe una matriz invertible $P\in \Mat[\R]{n}{n}$ tal que
    \[
        A = P^{-1} B P.
    \]
\end{defi}


\begin{teo}
    Sean $A,B,C\in \Mat[\R]{n}{n}$. Se tiene que
    \begin{enumerate}
    \item 
        $A$ es semejante a $A$.
    \item
        Si $A$ es semejante a $B$, entonces $B$ es semejante a $A$.
    \item
        Si $A$ es semejante a $B$ y $B$ es semejante a $C$, entonces $A$ es semejante a $C$.
    \end{enumerate}
\end{teo}

\begin{defi}
    Sea $A\in \Mat[\R]{n}{n}$, se dice que $A$ es diagonalizable si $A$ es semejante a una matriz diagonal. Es decir, si existe una matriz invertible $P\in \Mat[\R]{n}{n}$ y una matriz diagonal $D\in \Mat[\R]{n}{n}$ tal que
    \[
        A = P^{-1} D P.
    \]
\end{defi}

\begin{teo}
    Sean $A,B\in \Mat[\R]{n}{n}$. Si $A$ y $B$ son semejantes, entonces $A$ y $B$ tienen los mismos valores propios, es decir, tienen el mismo espectro.
\end{teo}

\begin{teo}
    Sea $A\in \Mat[\R]{n}{n}$. Se tiene que $A$ es diagonalizable si y solo si tiene $n$ vectores propios linealmente independientes.
\end{teo}

\begin{teo}
    Sea $A\in \Mat[\R]{n}{n}$. Si $A$ tiene $n$ valores propios distintos entre sí, entonces $A$ es diagonalizable.
\end{teo}

\begin{teo}
    Sea $A\in \Mat[\R]{n}{n}$. Si $A$ es diagonalizable, entonces tomando $D$ la matriz diagonal formada por los valores propios de $A$, y $P$ la matriz cuyas columnas son los vectores propios de $A$, se tiene que
    \[
        A = P^{-1} D P.
    \]
\end{teo}

\begin{teo}
    Sea $A\in \Mat[\R]{n}{n}$. Si $A$ es simétrica, entonces $A$ todos sus valores propios son reales; además, $A$ es diagonalizable.
\end{teo}

\begin{defi}
    Sea $A\in \Mat[\R]{n}{n}$. Se dice que $A$ es ortogonal si $A^\intercal = A^{-1}$.
\end{defi}


\begin{teo}
    Sea $A\in \Mat[\R]{n}{n}$. Si $A$ es simétrica, y $D,P\in \Mat[\R]{n}{n}$ son tales que
    \[
        A = P^{-1} D P,
    \]
    entonces $P$ es ortogonal.
\end{teo}




\end{document}