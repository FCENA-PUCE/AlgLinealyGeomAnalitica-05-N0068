\documentclass[a4,11pt]{aleph-notas}
% Se puede ver la documentación aquí: 
% https://github.com/alephsub0/LaTeX_aleph-notas

% -- Paquetes adicionales 
\usepackage{enumitem}
\usepackage{aleph-comandos}
\usepackage{booktabs}


% -- Datos 
\institucion{Facultad de Ciencias Exactas, Naturales y Ambientales}
\carrera{Catálogo STEM}
\asignatura{Álgebra Lineal y Geometría Analítica}
\tema{Resumen no. 5: Espacios vectoriales}
\autor{Andrés Merino}
\fecha{Periodo 2025-1}

\logouno[0.14\textwidth]{Logos/logoPUCE_04_ac}
\definecolor{colortext}{HTML}{0030A1}
\definecolor{colordef}{HTML}{0030A1}
\fuente{montserrat}


% -- Comandos adicionales
\setlist[enumerate]{label=\roman*.}


\begin{document}

\encabezado

%%%%%%%%%%%%%%%%%%%%%%%%%%%%%%%%%%%%%%
\section{Espacios vectoriales}
%%%%%%%%%%%%%%%%%%%%%%%%%%%%%%%%%%%%%%

\begin{defi}[Espacio Vectorial]
    Dados un campo $\R$, un conjunto no vacío $E$ y dos operaciones
    \[
        \funcion{\oplus}{E\times E}{E}{(x,y)}{x\oplus y,}
        \texty
        \funcion{\odot}{\R\times E}{E}{(\alpha,x)}{\alpha\odot x}
    \]
    llamadas suma y producto, respectivamente; se dice que $(E,\oplus,\odot,\R)$ es un espacio vectorial si cumplen las siguientes propiedades
    \begin{enumerate}
    \item \textbf{asociativa de la suma:}
        para todo $x,y,z\in E$ se tiene que
        \[
            (x \oplus y) \oplus z = x \oplus (y \oplus z);
        \]
    \item \textbf{conmutativa de la suma:}
        para todo $x,y\in E$ se tiene que
        \[
            x \oplus y = y \oplus x;
        \]
    \item \textbf{elemento neutro de la suma:}
        existe un elemento de $E$, denotado por $0_E$ o simplemente $0$, tal que para todo $x\in E$ se tiene que 
        \[
            x \oplus 0 = 0 \oplus x = x;
        \]
    \item \textbf{inverso de la suma:}
        para todo $x\in E$, existe un elemento de $E$, denotado por $-x$, tal que
        \[
            x \oplus (-x)= 0;
        \]
    \item \textbf{distributiva del producto I:}
        para todo $x,y\in E$ y todo $\alpha\in \R$ se tiene que
        \[
            \alpha\odot (x \oplus y)=\alpha\odot x + \alpha\odot y
        \]
    \item \textbf{distributiva del producto II:}
        para todo $x\in E$ y todo $\alpha,\beta\in \R$ se tiene que
        \[
            (\alpha+\beta)\odot x=\alpha\odot x \oplus \beta\odot x;
        \]
    \item \textbf{asociativa del producto:}
        para todo $x\in E$ y todo $\alpha,\beta\in \R$ se tiene que
        \[
            (\alpha\beta)\odot x=\alpha\odot(\beta\odot x);
        \]
    \item \textbf{elemento neutro del producto:}
        para todo $x\in E$ se tiene que
        \[
            1\odot x=x,
        \]
        donde $1\in\R$ es el elemento neutro multiplicativo de $\R$
    \end{enumerate}
\end{defi}

\begin{advertencia}
    Utilizamos los símbolos $\oplus$ y $\odot$ para enfatizar el hecho de que, en general, las operaciones definidas no son la suma y el producto estándar que utilizamos. Si no existe riesgo de confusión, utilizaremos la notación
    \[
        x\oplus y = x+y
        \texty
        \alpha \odot x = \alpha x
    \]
    y diremos que el espacio vectorial es $(E,+,\cdot,\R)$, es más, en caso de que no exista ambigüedad en las operaciones utilizadas se dirá simplemente que $E$ es un espacio vectorial.
\end{advertencia}

\begin{teo}
    Los siguientes conjuntos son espacios vectoriales en el campo $\R$:
    \begin{itemize}
    \item
        $\R^n$, con $n\in \N^*$;
    \item
        $\Mat[\R]{m}{n}$, con $m,n\in \N^*$;
    \item
        $\mathcal{F}(I)=\{\func{f}{I}{\R}: f\text{ es una función}\}$, con $I\subseteq \R$.
    % \item
    %     $\mathcal{C}(I)=\{\func{f}{I}{\R}: f\text{ es continua en }I\}$, con $I\subseteq \R$.
    % \item
    %     $\mathcal{C}^k(I)=\{\func{f}{I}{\R}: f\text{ es $k$ veces derivable en }I\text{ y }f^{k}\in\mathcal{C}^k(I)\}$, con $I\subseteq \R$.
    \item
        $\R_n[x]$ el conjunto de todos los polinomio de grado menor igual que $n$ en la variable $x$, con $n\in \N$;
    \end{itemize}
\end{teo}

\begin{teo}
    Sea $(E,+,\cdot,\R)$ un espacio vectorial, se tiene que para todo $u \in E$ y todo $\alpha \in \R$, se tiene que
    \begin{enumerate}
        \item $0u = 0$;
        \item $\alpha 0 = 0$;
        \item si $\alpha u=0$ entonces $\alpha  = 0$ o $u = 0$;
        \item $(-1) u = -u$.
    \end{enumerate}
\end{teo}

%%%%%%%%%%%%%%%%%%%%%%%%%%%%%%%%%%%%%%
\subsection{Subespacios}
%%%%%%%%%%%%%%%%%%%%%%%%%%%%%%%%%%%%%%

\begin{defi}[Subespacio Vectorial]
    Sea $(E,+,\cdot,\R)$ un espacio vectorial y $W \subset E$ un conjunto no vacío de $E$. Si $W$ es un espacio vectorial con respecto a las operaciones de $E$, entonces se dice que $W$ es un subespacio de $E$.
\end{defi}

\begin{teo}
    Sean $(E,+,\cdot,\R)$ un espacio vectorial y $W$ un subconjunto no vacío de $E$. Entonces $W$ es un subespacio de $E$ si y sólo si se cumplen las siguientes condiciones:
    \begin{itemize}
    \item 
        si $u,v \in W$, entonces $u + v \in W$; u
    \item 
        si $\alpha \in \R$ y $u \in W$, entonces $\alpha u \in W$.
    \end{itemize}
\end{teo}

\begin{defi}[Combinación lineal]
    Sean $(E,+,\cdot,\R)$ un espacio vectorial, $k\in\N^*$ y $v_1, v_2, \ldots, v_k \in E$. Se dice que un vector $v \in E$ es una combinación lineal de 
    $v_1, v_2, \ldots, v_k$ si 
    \[
        v = \alpha_1 v_1 + \alpha_2 v_2 + \cdots + \alpha_k v_k
    \]
    para algunos $\alpha_1, \alpha_2, \ldots, \alpha_k \in \R$.
\end{defi}

\begin{defi}
    Sean $(E,+,\cdot,\R)$ un espacio vectorial y $S = \{v_1, v_2, \ldots, v_k\} \subset E$. Al conjunto conjunto de todos los vectores en $E$ que son combinaciones lineales de los vectores de $S$ se lo llama cápsula de $S$ y se denota por $\spn(S)$, es decir
    \[
        \spn(S) = \{\alpha_1 v_1 + \alpha_2 v_2 + \cdots + \alpha_k v_k:\alpha_1, \alpha_2, \ldots, \alpha_k \in \R\}.
    \]
\end{defi}

\begin{advertencia}
    A este conjunto también se lo conoce como \emph{clausura lineal} y su notación viene de su nombre en inglés, \emph{linear span}. También se utiliza la notación $\langle S\rangle$.
\end{advertencia}

\begin{teo}
   Sean $(E,+,\cdot,\R)$ un espacio vectorial y $S\subset E$. Se tiene que $\spn(S)$ es un subespacio vectorial de $E$.
\end{teo}

\end{document}