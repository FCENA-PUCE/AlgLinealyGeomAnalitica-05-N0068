\documentclass[a4,11pt]{aleph-notas}
% Se puede ver la documentación aquí: 
% https://github.com/alephsub0/LaTeX_aleph-notas

% -- Paquetes adicionales 
\usepackage{enumitem}
\usepackage{aleph-comandos}
\usepackage{booktabs}


% -- Datos 
\institucion{Facultad de Ciencias Exactas, Naturales y Ambientale}
\carrera{Catálogo STEM}
\asignatura{Álgebra Lineal y Geometría Analítica}
\tema{Resumen no. 4: El espacio $\mathbb{R}^n$}
\autor{Andrés Merino}
\fecha{Semestre 2025-1}

\logouno[0.14\textwidth]{Logos/logoPUCE_04_ac}
\definecolor{colortext}{HTML}{0030A1}
\definecolor{colordef}{HTML}{0030A1}
\fuente{montserrat}


% -- Comandos adicionales
\setlist[enumerate]{label=\roman*.}


\begin{document}

\encabezado

%%%%%%%%%%%%%%%%%%%%%%%%%%%%%%%%%%%%%%
\section{El espacio $\R^n$}
%%%%%%%%%%%%%%%%%%%%%%%%%%%%%%%%%%%%%%

En esta sección, consideramos $n\in\N$ con $n\geq 1$.

\begin{defi}[El conjunto $\R^n$]
    El conjunto $\R^n$ es
    \[
        \R^n=\underbrace{\R\times\R\times\cdots\times\R}_{n\text{ veces}},
    \]
    es decir
    \[
        \R^n=\{(x_1,x_2,\ldots,x_n):x_i\in\R\text{ para todo }i=1,2,\ldots,n\}.
    \]
\end{defi}

\begin{advertencia}
    Recordemos que, por notación, si $y\in\R^n$, se asumirá $y=(y_1,y_2,\ldots,y_n)$.
\end{advertencia}

\begin{advertencia}
    Recordemos que podemos identificar cada elemento de $\R^n$ con una matriz de $\Mat{n}{1}$ de la siguiente manera: si
    \[
        a = (a_1,a_2,\ldots,a_n) \in \R^n,
    \]
    entonces, visto como matriz es
    \[
        a = 
        \begin{pmatrix}
            a_1\\a_2\\
            \vdots\\a_n
        \end{pmatrix}
        \in \Mat{n}{1}.
    \]
\end{advertencia}

\begin{teo}
    En $\R^n$ se cumplen las siguientes propiedades:
    \begin{enumerate}
    \item \textbf{asociativa de la suma:}
        para todo $x,y,z\in \R^n$ se tiene que
        \[
            (x + y) + z = x + (y + z);
        \]
    \item \textbf{conmutativa de la suma:}
        para todo $x,y\in \R^n$ se tiene que
        \[
            x + y = y + x;
        \]
    \item \textbf{elemento neutro de la suma:}
        existe un elemento de $\R^n$, denotado por $0$, tal que para todo $x\in \R^n$ se tiene que 
        \[
            x + 0 = 0 + x = x;
        \]
    \item \textbf{inverso de la suma:}
        para todo $x\in\R^n$, existe un elemento de $\R^n$, denotado por $-x$, tal que
        \[
            x + (-x)= 0;
        \]
    \item \textbf{distributiva del producto I:}
        para todo $x,y\in\R^n$ y todo $\alpha\in \R$ se tiene que
        \[
            \alpha (x + y)=\alpha x + \alpha y
        \]
    \item \textbf{distributiva del producto II:}
        para todo $x\in\R^n$ y todo $\alpha,\beta\in \R$ se tiene que
        \[
            (\alpha+\beta) x=\alpha x + \beta x;
        \]
    \item \textbf{asociativa del producto:}
        para todo $x\in\R^n$ y todo $\alpha,\beta\in \R$ se tiene que
        \[
            (\alpha\beta) x=\alpha(\beta x);
        \]
    \item \textbf{elemento neutro del producto:}
        para todo $x\in\R^n$ se tiene que
        \[
            1 x=x.
        \]
    \end{enumerate}
\end{teo}

\begin{defi}[Base canónica]
    En $\R^n$, el conjunto $\{e^1,e^2,\ldots,e^n\}\subset\R^n$, definidos por 
    \[
        e^i_j=
        \begin{cases}
            0& \text{si }i\neq j,\\
            1& \text{si }i= j,
        \end{cases}
    \]
    para todo $i,j\in\{1,2,\ldots,n\}$, se lo denomina \emph{base canónica} de $\R^n$.
\end{defi}

\begin{teo}
    Sea $x\in\R^n$, se tiene que existen únicos
    \[
        \alpha_1,\alpha_2,\ldots,\alpha_n\in\R
    \]
    tales que
    \[
        x = \alpha_1e^1 + \alpha_2 e^2 + \cdots + \alpha_n e^n.
    \]
\end{teo}

\begin{defi}[Producto punto]
    La función 
    \[
        \funcion{\cdot}{\R^n\times \R^n}{\R}
        {(x,y)}{\sum_{i=1}^n x_iy_i}
    \]
    se denomina producto punto de $\R^n$ o producto interno de $\R^n$.
\end{defi}

\begin{teo}[Propiedades del producto punto]
    Sean $x,y,z\in \R^n$ y $\alpha\in\R$, se tiene que
    \begin{itemize}
    \item
        $x\cdot x \geq 0$;
    \item
        $x\cdot x = 0$ si y solo si $x=0$;
    \item
        $x \cdot y = y \cdot x$;
    \item 
        $(x + y)\cdot z = (x\cdot z) + (y\cdot z)$; y
    \item
        $(\alpha x)\cdot y = x\cdot(\alpha y) = \alpha(x\cdot y)$.
    \end{itemize}
\end{teo}

\begin{defi}[Norma]
    La función
    \[
        \funcion{\|\cdot\|}{\R^n}{\R}
        {x}{\sqrt{x\cdot x}}
    \]
   se la llama la norma de $\R^n$. Para $x\in\R^n$ a $\|x\|$ se le llama norma, módulo o longitud de $x$.
\end{defi}


\begin{teo}[Propiedades de la norma]
    Sean $x,y\in \R^n$ y $\alpha\in\R$, se tiene que
    \begin{itemize}
    \item
        $\|x\| \geq 0$;
    \item
        $\|x\| = 0$ si y solo si $x=0$;
    \item 
        $\|\alpha x\| = |\alpha|\|x\|$; y
    \item
        $\|x + y\| \leq \|x\|+\|y\|$
    \end{itemize}
\end{teo}

\begin{advertencia}
    Dados $x,y\in\R^n$, se define la distancia entre $x$ y $y$ por $\|x-y\|$.
\end{advertencia}

\begin{defi}[Vector unitario]
    Dado $x\in\R^n$, se dice que $x$ es un vector unitario si $\|x\|=1$.
\end{defi}

    

\begin{defi}[Ángulo entre vectores]
    Sean $x,y\in\R^n$, ambos diferentes de 0, se define el ángulo entre estos vectores por
    \[
       \theta =  \arccos\left( \frac{x\cdot y}{\|x\|\,\|y\|} \right).
    \]
\end{defi}

\begin{defi}
    Sean $x,y\in\R^n$, se dice que 
    \begin{itemize}
    \item 
        son ortogonales si $x\cdot y =0$;
    \item
        son paralelos si $|x\cdot y| = \|x\|\,\|y\|$.
    \end{itemize}
\end{defi}

%%%%%%%%%%%%%%%%%%%%%%%%%%%%%%%%%%%%%%
\subsection{Geometría de $\R^n$}
%%%%%%%%%%%%%%%%%%%%%%%%%%%%%%%%%%%%%%

En esta sección, a menos que se indique lo contrario, asumiremos $n\in\N$ con $n\geq 2$.
%%%%%%%%%%%%%%%%%%%%%%%%%%%%%%%%%%%%%%

\begin{defi}[Puntos de $\R^n$]
    Se llama punto de $\R^n$ a cualquier elemento de $\R^n$. 
\end{defi}


\begin{defi}[Recta de $\R^n$]
    Dados $a,b\in\R^n$, con $b\neq0$, la recta que pasa por $a$ con dirección $b$ es el conjunto
    \[
        L(a;b)=\{a+t b:t\in\R\}.
    \]
\end{defi}

\begin{proof}[Ejemplo]
    En $\R^2$, tenemos la recta que pasa por $(1,-2)$ con dirección $(3,2)$ es
    \begin{align*}
        L((1,-2);(3,2)) 
            & = \{(1,-2)+t(3,2):t \in\R\} \\
            & = \{(1+3t ,-2+2t ): t \in\R\} \\
            & = \{(x,y)\in\R^2: x = 1+3t ,\ y=-2+2t ,\ t \in\R\} \\
            & = \{(x,y)\in\R^2: 2x-3y=4\}
    \end{align*}
    Con esto, tenemos que la recta está definida por
    \begin{align*}
        x& = 1+3t ,\\
        y& = -2+2t ,
    \end{align*}
    con $t \in \R$; a esta se la llama la ecuación paramétrica de la recta. Por otro lado, también tenemos que la recta está definida por 
    \[
        2x-3y=4
    \]
    con $(x,y)\in\R^2$; a esta se la llama la ecuación cartesiana de la recta.
\end{proof}

\begin{defi}[Plano de $\R^n$]
    Dados $a,b,c\in\R^n$, con $b$ y $c$ no paralelos y diferentes de $0$, el plano que pasa por $a$ con dirección $b$ y $c$ es el conjunto
    \[
        P(a;b,c)=\{a+s  b+t c:s ,t \in\R\}.
    \]
\end{defi}

\begin{proof}[Ejemplo]
    En $\R^3$, tenemos la recta que pasa por $(1,-2,-1)$ con direcciones $(3,2,1)$ y $(1,0,2)$ es
    \begin{align*}
        L&((1,-2,-1);(3,2,1),(1,0,2)) \\
            & = \{(1,-2,-1)+s (3,2,1) + t (1,0,2) :s ,t \in\R\} \\
            & = \{(1+3s +t ,-2+2s ,-1+s +2t ): s ,t \in\R\} \\
            & = \{(x,y,z)\in\R^3: x = 1+3s +t ,\ y=-2+2s ,\ z=-1+s +2t ,\ s ,t \in\R\} \\
            & = \{(x,y,z)\in\R^3: 4x-5y-2z=8\}
    \end{align*}
    Con esto, tenemos que la recta está definida por
    \begin{align*}
        x& = 1+3s +t ,\\
        y& = -2+2s ,\\
        z& = -1+s +2t 
    \end{align*}
    con $s ,t \in \R$; a esta se la llama la ecuación paramétrica del plano. Por otro lado, también tenemos que la recta está definida por 
    \[
        4x-5y-2z=8
    \]
    con $(x,y,z)\in\R^3$; a esta se la llama la ecuación cartesiana de la recta.
\end{proof}

\begin{advertencia}
    Podemos pasar de la ecuación paramétrica a la ecuación cartesiana si, al verlas como un sistema de ecuaciones en sus parámetros, analizamos cuándo el sistema es consistente.
    
    Además, podemos pasar de la ecuación cartesiana a la ecuación paramétrica si, al verla como un sistema de ecuaciones, resolvemos el sistema.
\end{advertencia}

\begin{advertencia}
    En $\R^2$, dados dos puntos $(x_1,y_1),(x_2,y_2)\in\R^2$, la ecuación cartesiana de la recta que pasa por estos dos puntos es
    \[
        \begin{vmatrix}
            x & y & 1\\
            x_1 & y_1 & 1\\
            x_2 & y_2 & 1
        \end{vmatrix}
        = 0,
    \]
    para $(x,y)\in\R^2$.
\end{advertencia}

\begin{advertencia}
    En $\R^3$, dados tres puntos $(x_1,y_1,z_1),(x_2,y_2,z_2),(x_3,y_3,z_3)\in\R^3$, la ecuación cartesiana del plano que pasa por estos tres puntos es
    \[
        \begin{vmatrix}
            x & y & z & 1\\
            x_1 & y_1 & z_1 & 1\\
            x_2 & y_2 & z_2 & 1\\
            x_3 & y_3 & z_3 & 1
        \end{vmatrix}
        = 0,
    \]
    para $(x,y,z)\in\R^3$.
\end{advertencia}

\end{document}